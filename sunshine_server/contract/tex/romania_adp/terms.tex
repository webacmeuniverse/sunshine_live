\begin{multicols}{2}
[\section{GENERAL TERMS AND CONDITIONS}]

\subsection{DEFINITIONS}
\begin{itemize}[label={}]
  	\item\textbf{Agreement:} this Energy Performance Contract concluded between the Client and the Contractor including the Specific Conditions and its Annexes and the General Terms and Conditions and developed and managed via the Sunshine platform - \url{https://sunshine.acme.universe} (also known as sunshine).
	\item\textbf{Apartment:} an apartment property is a legally partitioned autonomous real property in a residential building.
	\item\textbf{Apartment Owner:} an apartment owner is a person who has acquired the apartment property and corroborated the title with the Land Register.
	\item\textbf{Banking Day:} is a day when in a place where pursuant to provisions of the Contract bank transfers are to be executed the commercial banks perform general banking transactions.
	\item\textbf{Baseline:} means the consumption of heat energy and domestic hot water of the Building, expressed as yearly average value, occurring during the Baseline Period.
	\item\textbf{Baseline Period:} is a mutually agreed time period representing the functioning of the Building prior to the implementation of the Measures.
	\item\textbf{Building:} the multi-apartment residential building where the Contractor delivers the Renovation Works and provides the Services under this Agreement.
	\item\textbf{Business Day:} an official working day that is not an official holiday or an official work day off under Romanian Law.
	\item\textbf{Client:} The Apartment Owner(s) of the Building or the authorized person acting on behalf of the Apartment Owner(s).
	\item\textbf{Comfort Standards:} the set of indoor climate conditions and parameters which the Contractor guarantees to the Client under this Agreement.
	\item\textbf{Commencement Date:} the date on which the Construction Period starts.
	\item\textbf{Commissioning Date:} the date on which the Parties sign the Delivery and Acceptance Protocol for the Measures and the date when the Service Period of the Agreement starts.
	\item\textbf{Construction Period:} the period planned by the Contractor for the implementation of the Measures. The Construction Period starts at the Commencement Date and ends at the Commissioning Date.
	\item\textbf{Contractor:} a legal entity that undertakes this contract, the Renovation Works and provides the Services based on the provisions of the Agreement.
	\item\textbf{Delivery and Acceptance Protocol:} the protocol prepared by the Contractor in accordance to Romanian regulation and Norms for final commissioning of the Measures implemented in the Building by the Contractor.
	\item\textbf{Domestic Hot Water Fee:} the Fee paid by the Client top the Contractor which is due for the actual domestic hot water consumption, at current Heat Energy Tariff.
	\item\textbf{Energy:} a product of certain value – fuel, thermal energy, renewable energy, electricity, or any other type of energy.
	\item\textbf{Energy Audit:} actions performed for the purpose to obtain information about the pattern of energy consumption in buildings or groups of buildings, procedures, or equipment, as well as to measure and verify possibilities of economically feasible energy savings, and findings of which are compiled in a report.
	\item\textbf{Energy Consumption Guarantee:} the amount of heat energy consumed in the Building for space heating and circulation losses during the Service Period reached based on the Energy Savings Guarantee provided by the Contractor and used for the determination of the Flat Heat Energy Consumption.
	\item\textbf{Energy Efficiency Service (Services):} a set of actions performed by the Contractor including the implementation of the Measures in the Building, operation, and maintenance of the implemented Measures, energy consumption data analyses, monitoring and assessment of energy consumption, in particular, related to the fulfillment of the Energy Saving Guarantee.
	\item\textbf{Energy Savings:} volume of the saved energy, which is established by measuring and verifying consumption before and after implementation of one or more energy performance measures and achieved in the Building through the implementation of the Measures and the provision of energy performance.
	\item\textbf{Energy Savings Guarantee:} the minimum amount of Energy Savings resulting from the implementation of the Measures and from the provision of Energy Efficiency Services, guaranteed in the Agreement by the Contractor and determined according to a Measurement and Verification Plan.
	\item\textbf{Energy Tariff:} charge for energy unit where the Building is located.
	\item\textbf{Sunshine platform - } \url{https://sunshine/acme.universe}: the multi-sided multi-stakeholder online platform for Energy Performance Contracting available at sunshineplatform.eu which supports the development and management of buildings renovation projects based on Energy Performance Contracting.
	\item\textbf{Fee(s):} monthly recurring Fee(s) paid by the Client to the Contractor for rendering the Services prescribed in the Agreement for the duration of the Service Period, which includes the Heat Energy Fee, the Domestic Hot Water Fee the Renovation Fee and the Operation and Maintenance Fee along with any taxes due (such as VAT).
	\item\textbf{Flat Heat Energy Consumption:}  the amount of thermal energy calculated by the Contractor for charging to the Client a fixed monthly amount of heat energy for each Settlement period during the Service Period.
	\item\textbf{Financial Contribution:} the share of the investment costs for the Renovation Works financed directly with equity or indirectly arranged as third party financing by the Contractor and for which the Contractor charges the Renovation Fee.
	\item\textbf{Heat Energy Fee:} the Fee paid by the Client to the Contractor due for the energy consumed by the Building during the Service Period subject to adjustments and balance once per year at the end of the Settlement Period to account for the actual weather conditions during the Settlement Period and for Measurement and Verification of the Energy Savings Guarantee.
	\item\textbf{Heat Supply:} the supply of thermal energy to the Building for the needs of space heating and domestic hot water preparation.
	\item\textbf{Heating Season:} the period in the year when the Contractor shall comply with the Comfort Standards guarantees set out in this Agreement starting from the 1st October and ending the 30th April of each settlement year during the Service Period.
	\item\textbf{Invoice:} an invoice issued to the Client (Apartment Owners or a representative of the Apartment owners) for the Services received and for other payments due to the Contractor arising out of the Agreement and issued in full compliance with all applicable statutory requirements under Romanian law.
	\item\textbf{IPMVP:} the International Protocol of Measurement and Verification of the effect of energy conservation prepared by the EVO (Efficiency Valuation Organization), (1629 K Street NW, Suite 300, Washington, DC 20006, USA), which is applied for the purposes of measurement and verification within the framework of the Agreement.
        \item\textbf{National Energy Regulator - ANRE} (Autoritatea Națională de Reglementare în domeniul Energiei) - is an autonomous administrative authority, with legal personality, under parliamentary control, fully financed from its own revenues, with decision-making, organizational and functional independence, whose objects are the preparation, approval and monitoring of the application of mandatory regulations at national level necessary for the operation of the electricity, heat and natural gas sector and market under conditions of efficiency, competition, transparency and consumer protection.
	\item\textbf{Latent Condition(s):} faults and defects of the Building, or adjacent to the Building, of which the Client did not have knowledge and the Contractor could not identify by reasonable observations and ordinary inspections when the Agreement was prepared.
	\item\textbf{Manager:} an individual or a legal entity, who in line with the applicable provisions of the Romanian Law on Management of Residential Housing and based on a management agreement, carries out managerial and maintenance activities assigned by the Client and established by the Agreement.
	\item\textbf{Measure(s)} also referred as Energy Efficiency Measures: such actions that result in achievement of verifiable, measurable or estimable increase of energy performance and other construction and installation works aiming at the refurbishment and improvement of the Building both for structural and aesthetical reasons.
	\item\textbf{Measurement and Verification:} the process and activities conducted in order to determine the Energy Savings attributable to the Building as a result of the implemented Measures and provided Services.
	\item\textbf{Operation and Maintenance Fee:} the Fee paid by the Client to the Contractor due for the Services related to Operation and Maintenance of the Measures and subject to annual indexation with the applicable Romanian Consumer Price Index for the respective year as published by Centrala statistikas parvalde (Central Statistical Bureau).
	\item\textbf{Operational and Maintenance Manual:}  a manual indicating the maintenance schedule for the Measures implemented under this Agreement and the operational activities covered by the Agreement.
	\item\textbf{Parties:} the Client and the Contractor collectively.
	\item\textbf{Party:} the Client and the Contractor each individually.
	\item\textbf{Payment Schedule:} a document prepared by the Contractor for the Client showing the Renovation Fee for the repayment of the Financial Contribution, calculated for each interest calculation period in accordance with this Agreement.
	\item\textbf{Proper Functioning:} the functioning of the Measures in such a manner as to ensure the achievement of their full functionalities and efficiency and includes all necessary maintenance actions undertaken by and at the expense of the Contractor.
	\item\textbf{Regulator:} the \textbf{National Energy Regulator} ANRE or another applicable authority established in the laws and regulations effective in Romania which approves tariffs for trading thermal energy in the respective local government where the Building is situated.
	\item\textbf{Renovation Fee:} the Euribor indexed Fee paid by the Client to the Contractor for the Contractor’s Financial Contribution.
	\item\textbf{Renovation Work:} activities undertaken by the Contractor necessary for the implementation of the Measures in the Building, including engineering, procurement, supply, installation, start-up, commissioning and financing of the Measures.
        \item\textbf{ROMBEEF:} Romanian Building Energy Efficiency Facility, operating as a joint stock company duly registered in the Commercial register of Romania under corporate No. xx
	\item\textbf{Service Period:} the period during which the Contractor provides the Services to the Client. The Service Period starts at the Commissioning Date.
	\item\textbf{Settlement Period:} the period of one calendar year recurring on an annual basis during the Service Period.
	\item\textbf{Statement:} a document signed by the Parties to evidence various parameters present at the Building as recorded at the time of execution of such document.
	\item\textbf{VAT:} the value-added tax payable in accordance with laws and regulations effective in Romania and provisions of the Agreement.
\end{itemize}

\subsection{ACCEPTANCE OF AGREEMENT TERMS}
\begin{enumerate}
	\item	The Client is of the opinion that the Contractor has the necessary qualifications, experience, and abilities to carry out the Renovation Works and provide the Services to the Client. For this reason, the Client authorizes the Contractor and will assist the Contractor, to undertake, at the Contractor’s expense all legal and factual actions to execute the Agreement, without the need for an express power of attorney to the benefit of the Contractor.
	\item	The Contractor will undertake the Renovation Works and provide the Services to the Client on the terms and conditions set out in this Agreement. The Contractor acknowledges it has satisfied itself regarding the nature, situation, and location of the Building and all other matters which could in any way affect the performance of its obligations under the Agreement. Any failure by the Contractor to acquaint itself with the Building or any Building site conditions under this Clause will not relieve it from responsibility for performing its obligations under the Agreement.
	\item	The Contract confirms that the budget included in the Specific Terms of this Agreement includes all construction works, material and equipment which are necessary for delivering the Renovation Works in accordance with to the project technical specifications and conditions of this Agreement.
	\item	The definitions for all purposes of the Agreement, its Specific Conditions, its Annexes and this General Terms and Conditions shall have the respective meanings indicated in Article 1 of the General Terms and Conditions of this Agreement.
	\item	In cases of discrepancy between the General Terms and Conditions and the Specific Conditions and its Annexes, the provisions of the latter take precedence.
\end{enumerate}

\subsection{SAFETY, QUALITY, AND COMFORT}
\begin{enumerate}
	\item	The Services rendered by the Contractor under this Agreement shall:
	\begin{enumerate}
		\item	be delivered with the highest standard of skill and care as expected by experienced and professional contractors regularly undertaking work and services of the same or similar scope and complexity as in this Agreement;
		\item	be developed using material and equipment of suitable quality, new, fit for the purpose;
		\item	conform with the construction legislation and any other applicable legal rules, regulations or norms effective in the Romania at the time of rendering the Services;
		\item	be performed to cause as little inconvenience as possible to the use of the Building by the Client and other occupants of the Building;
	\end{enumerate}
	\item	The Comfort Standards meet or exceed the level outlined in the Specific Conditions of this Agreement during the Service Period of the Agreement.
	\item	During the time when the windows in an Apartment of the Building are open and for 2 (two) hours after the windows have been closed, the Contractor does not guarantee the indoor temperature levels agreed in the Specific Conditions of the Agreement for the specific Apartment where windows have been opened.
	\item	The Contractor shall ensure an adequate level of ventilation in the Apartments according to relevant Romanian regulations and norms.
	\item	The Contractor shall take all necessary actions to ensure safety and health protection of the employees at work in accordance with the Labour Protection Law and all relevant Romanian regulation and norms.
	\item	The Contractor shall implement suitable protection measures to protect all people from death or injury which may be caused by default or gross negligence of the Contractor, its employees, agents or subcontractors during the Construction works and Service Period. The Contractor shall also protect the entire Building from damages related to the implementation of the Measures.
	\item	The Contractor shall ensure that all utility services provided to the Building are not disconnected or disrupted due to a default or negligence of the Contractor at any time without prior notice. Any utility services disrupted or disconnected due to a default or negligence of the Contractor shall be promptly reinstated by the Contractor, at Contractor’s costs. The Contractor is not liable for cases when such disruptions are beyond the Contractor’s control and/or are due to acts or omissions by the maintenance company, energy and water utilities, or any third parties not linked to the Contractor.
	\item	The Contractor during the Construction Period shall ensure proper protection of the Building from the impact of weather by preventing infiltration of rainwater and damages to the Building. Groundwater infiltrations and Force Majeure events are excluded.
	\item	The Contractor abides by the European Code of Conduct for Energy Performance Contracting (\url{http://transparense.eu}), which is a set of values and principles that are considered fundamental for the successful, professional and transparent implementation of Energy Performance Contracting in European countries.
\end{enumerate}

\subsection{GUARANTEES}
\begin{enumerate}
	\item	The Contractor during the Service Period shall provide to the Client an Energy Savings Guarantee as part of this Agreement, which on a yearly basis is subject to Measurement and Verification.
	\item	The Contractor during the Service Period shall guarantee the agreed Comfort Standards under this Agreement.
	\item	The Contractor shall guarantee during the Service Period, at its own expenses, the Proper Functioning of the Measures installed or introduced by the Contractor for the heating system, domestic hot water supply systems, Heating Ventilation and Air cooling systems, junctions and pipelines, in line with their specifications and normal wear and tear, during the term of this Agreement, inter alia, by repairing or replacing the Measures, if necessary.
	\item	The Contractor shall guarantee during the Service Period, at its own expenses, the effect and efficiency of the insulating materials installed or introduced by the Contractor in line with their specifications and normal wear and tear, during the entire validity of the Agreement, inter alia by repairing them or replacing them, if necessary.
	\item	The Contractor shall ensure at the end of the Service Period the Proper Functioning of all implemented Measures in line with their specifications and normal wear and tear and considering proper maintenance. The Contractor at the end of the Service Period shall provide to the Client all use, care and maintenance manuals, records, instructions, other documentation, software, intellectual property licenses, special tools and protocols and procedures necessary or convenient for the continued good performance of the Measures to achieve the Comfort Standards under this Agreement.
	\item	The Contractor before the starting of the Construction Period shall submit to the Client a performance guarantee of a credit institution or insurance company for the performance of its obligations of 10\% of the total Investment Costs (excluding VAT) as follows:
	\begin{enumerate}
		\item	when the Contractor is also the general construction company, this performance guarantee is provided by the Contractor in favor of the Client against the provisions of this Agreement;
		\item	when the Contractor is procuring the general construction company, this performance guarantee is provided by the general construction company in favors of the Contractor and based on the provision of the construction contract between the Contractor and the general construction company;
		\item	in case the Contractor fails to provide such original performance guarantee for the Construction Period securing the execution of activities in the Construction Period, the Contractor has no right to initiate the construction works.
		\item	this performance guarantee shall be valid during all Construction Period. In case the Construction Period is extended, the Contractor shall extend this guarantee by the same period of time.
	\end{enumerate}
	\item	The Contractor not later than 10 (ten) days after the signature of the Delivery and Acceptance Protocol shall submit to the Client a performance guarantee of a credit institution or insurance company for the performance of the its obligations of at least 5\% of the Investment Costs (excluding VAT) as follows:
	\begin{enumerate}
		\item	when the Contractor is also the general construction company, this performance guarantee is provided by the Contractor in favour of the Client against the provisions of this Agreement;
		\item	when the Contractor is procuring the general construction company, this performance guarantee is provided by the general construction company in favours of the Contractor and based on the provision of the construction contract between the Contractor and the general construction company;
		\item	this guarantee shall be valid for 36 (thirty-six) months;
	\end{enumerate}
	\item	The Client has the right to call the performance guarantee referred in Article 4.6. and Article 4.7. for the liquidation of the financial obligations of the Contractor or regulatory enactments. %chktex 12
	\item	The performance guarantee referred in this Article shall be issued by a credit institution or insurance company registered in the Republic of Romania or any other member state of European Union or European Economic Area, which according to the procedure set by the legal acts of the Republic of Romania has commenced the provision of services in the territory of the Republic of Romania.
\end{enumerate}

\subsection{RIGHTS AND OBLIGATIONS OF THE CONTRACTOR}
\begin{enumerate}
        \item The Contractor has the necessary professional qualifications, experience, and abilities in supplying the equipment, material, and services under the Agreement.
        \ item The Contractor shall obtain all necessary permits and approvals from the government, municipal institutions and authorities for the implementation of the Renovation Works and the delivery of the Services without the involvement of the Client, unless so needed under applicable laws or regulations.
	\item The Contractor shall start the implementation of the construction and installation works of the Measures on the Commencement Date and shall complete them within the provided Construction Period. The Contractor shall inform the Client of the provisional Commencement Date, not later than within 20 Business days after the signing of this Agreement.
	\ item The Contractor shall notify the Client in writing at least 10 (ten) Business days in advance about the Date of Commencement of the of the construction and installation works of the Measures, giving an opportunity for the Client to clear the common areas of the Building (including stairways, basement space, attic, roof, coal/wood and gas storage areas, electricity and telecommunication panels, and boiler plant rooms), from waste, abandoned properties and any other object there located. If the Client fails to clear the common areas in due time, the Contractor has the right to procure clearing of the common areas of the Building and issue an invoice to the Client for the payment of these works in compensation for the expenses incurred. The Client shall promptly pay such invoice not later than within 20 Business days.
	\item During the Construction Period the Contractor shall provide all necessary labour for the implementation of the Measures, including the necessary supervision, tools, materials, and equipment of suitable nature, quality, and quantities.
	\item During the Service Period the Contractor shall provide all necessary labour for the operation and maintenance of the Measures, including the necessary supervision, tools, materials, and equipment of suitable nature, quality, and quantities.
	\item During the Construction Period, the Contractor shall arrange electricity supply with separate metering and pay for the electricity consumed for the implementation and installation works of the Measures. The Contractor has the right to have access to the common electricity supply system of the Building.
	\item The Contractor shall properly clean the construction site (the common areas of the Building, windows, entrances, and surrounding) on the completion of the construction and installation works of the Measures before the Commissioning Date.
	\item The Contractor shall invite the Client for commissioning the Measures implemented in the Building. The Contractor shall provide to the Client the Delivery and Acceptance Protocol of the Building at the end of the Construction Period.
	\item The Contractor during the Service Period shall notify the Client in writing in case waste and/or objects are stored and abandoned by Apartment Owners or other third parties not related to the Contractor in common area of the Building, which could cause problems to the Contractor in operation and maintenance activities under the Agreement. If the Client fails to clean the area in accordance with this Agreement, the Contractor has the right to procure cleaning of the premises and issue an invoice to the Client for payment in compensation for carrying out such cleaning. The Client shall promptly pay such invoice not later than within 20 Business days.
	\item The Contractor during the Service Period shall notify the Client in writing about any identified fault, theft, vandalism or sabotage made to the Measures.
	\item The Contractor shall ensure sufficient Heat Energy Supply to the Building during the Heating Season in a manner appropriate to fulfil the Comfort Standards of this Agreement. The Contractor under this Agreement is not liable for disruptions in, or lack of, Heat Energy Supply to the Building in cases that are beyond the control of the Contractor, including the failure of the heating supply company to supply Heat Energy or due to Force Major Events.
	\item The Contractor before the beginning of Construction Period shall present to the Client the project for the renovation of the Building in accordance with the Parliament Law no. 10 from 18 January 1995 - updated,r egarding quality in construction and Law no. 50 from 29 July 1991 regarding authorization of execution in construction updated, and coordinate with the Client and the construction supervisor.
	\item The Contractor during the Construction Period shall notify and invite the Client to participate in the weekly monitoring meetings on the status of the construction works. The Contractor shall then submit to the Client 2 (two) progress reports per month on the status and progress of the construction works. These reports can be submitted electronically at the online EPC SUNShINE - Platform \url{https://sunshine.acme.universe}.
	\item In cases when the Contractor is the intermediary between the Client and the heat energy supplier, the Contractor shall pay in the name and on behalf of the Client the invoices due to the Heating Company upon receipt of the corresponding Heat Energy Fee component due under this Agreement to the Contractor. Notwithstanding the above, invoices for heating during the Construction Period shall be at the expense of the Client.
	\item The Contractor has the right to assign or subcontract to third parties (subcontractors) the performance of works and services set forth in the Agreement. The Contractor is fully liable for subcontractors with respect to the obligation of this Agreement.
	\item The Contractor has the right to modify the Energy Savings Guarantee in case of a change in use of the Building (Article 17). Any change shall be agreed between the Parties with a written amendment of this Agreement.
	\item The Contractor has the right to invite, at its own expenses, a properly qualified and experienced independent expert having been pre-approved by the Client (such approval not to be unreasonably withheld) to evaluate the conformity of the proposed Measures with the laws and regulations effective in the Romania or decisions of the local governments binding on the Client in case the Client shall exercise its veto rights. The opinion of such expert shall be binding on the Parties.
	\item The Contractor shall not later than within 5 (five) Business days, notify the Client about a change of the address specified in the Agreement or other changes in its legal status, administration and legal situation; in particular, if the Contractor undergoes any merger or acquisition, enters into liquidation or bankruptcy proceedings.
\end{enumerate}

\subsection{SETTLEMENT PROCEDURE}
\begin{enumerate}
	\item	The Client shall pay the monthly Fees to the Contractor set forth in this Agreement.
	\item	The period for the mutual settlement of accounts between the Parties is a calendar month. The Settlement Period for the calculation of the Balance Payment between the Invoices based on the flat Heat Energy Consumption and the actual metered Heat Energy Consumption and for Measurement and Verification of the Energy Savings Guarantee is one year.
	\item	Every month the Contractor, or an appointed third party representing the Contractor, shall calculate the payment to be made by the Client under the Agreement. The total sum of all Fees calculated shall be deemed due and payable by the Client to the Contractor for the rendered Services under this Agreement for the mutual settlement of the accounts.
	\item	Every 12 (twelve) months on the anniversary from the starting of the Service Period the Contractor shall conduct an annual settlement based on the results of the Measurement and Verification of the Energy Savings Guarantee.
	\item	The Contractor or its assigned third party shall issue not later than the 10th (tenth) day of each month an Invoice itemizing all Fees components as expressly outlined in the Agreement and deliver it to the Client or the Manager representing the Client.
	\item	The Contractor shall ensure that the information included in the Invoices to each Apartment Owners for the rendered Services is presented in a clear and comprehensive manner, separately indicating, the Renovation Fee and the Contractor’s Operational and Maintenance Fee.
	\item	The first payment of the Fees shall be calculated as due 1 (one) month following the signing of Delivery and Acceptance Protocol. Until then, the Client remains obligated to cover all its utility and communal expenses as due.
	\item	The Client (each Apartment Owner) shall pay the Fees to the Contractor (or to a third party indicated by the Contractor) directly or via the facilitation of the Manager based on the Invoices issued by the Manager for all utilities and other operational expenses for the maintenance of the Building, part of which shall include the Fees due to the Contractor. The Client shall pay the Fees according to established practices by the Manager, but not later than within 15 (fifteen) days as of the date of receipt of the Invoice by transferring the necessary funds to the bank account specified by the Manager.
	\item	The Contractor or its assigned Manager shall administer information associated with the settlement of accounts in relation to this Agreement, inter alia, by:
	\begin{enumerate}
		\item	recording all information about the Invoices issued to each of the Apartment Owners, including their amounts;
		\item	keeping records about payment of the Invoices and constantly updating the amount of each Apartment Owner’s debt, if there is such;
		\item	the Client upon request of the Contractor or any of its assignees shall provide the Contractor or such assignee (as applicable) with the current report on payments made by the Apartment Owners of the Building and the list of debtors.
	\end{enumerate}
\end{enumerate}

\subsection{TERM OF THE AGREEMENT}
\begin{enumerate}
	\item	The Term of this Agreement begins on the date of this Agreement and will remain in full force and effect until the completion of the Service Period, subject to early termination as provided in this Agreement.
	\item	The Tem of this Agreement may be extended by written consent of the Parties, in particular, the Parties have the right to anticipate or postpone the Commencement Date and Commissioning Date by written consent of the Parties.
	\item	The Contractor shall start the implementation of the construction and installation works of the Measures on the Commencement Date and shall complete them within the provided Construction Period. At the conclusion of the Construction Period, the Client and Contractor shall sign the Delivery and Acceptance Protocol.
	\item	The Service Period and payment term begin at the signature of the Delivery and Acceptance Protocol.
	\item	In case of default or negligence of the Client, such as if the Contractor is not furnished with all due documents and/or access to the Building and/or in case of Force Majeure events, the Commencement Date, the Construction Period and the Service Period shall be considered automatically postponed by the period of delay. These changes shall be agreed in writing by the Parties, and the Agreement amended.
\end{enumerate}

\subsection{LATENT CONDITIONS}
\begin{enumerate}
	\item  If during the Construction Period, the Contractor becomes aware of any Latent Conditions that will affect the completion of the Measures, the Contractor must (as a condition precedent to any entitlement to additional time or money) give a written Notice to the Client within 5 (five) Business days specifying:
	\begin{enumerate}
		\item	the Latent Conditions encountered and in what respects they differ materially from the condition of the Building, which should reasonably have been anticipated by a competent and experienced Contractor exercising good industry practice at the date of the Agreement;
		\item	the additional work and additional resources, which the Contractor estimates to be necessary to deal with the Latent Conditions;
		\item	the time the Contractor estimates will be required to deal with the Latent Conditions and the expected delay in completion;
		\item	the Contractor’s reasonable estimate of the cost of the measures necessary to deal with the Latent Conditions; and
		\item	any other details that may be reasonably required by the Client.
	\end{enumerate}
	\item The delay caused by a Latent Condition may justify an extension of the Construction Period if the latter causes the Contractor to:
	\begin{enumerate}
		\item	carry out additional work;
		\item	use additional materials; or
		\item	incur additional costs (including, but not limited to, the cost of delay or disruption); which the Contractor did not and could not reasonably have anticipated at the time of signature of the Agreement exercising good industry practice.
	\end{enumerate}
	\item	The Client shall cover all actual expenses which arise in connection with Latent Conditions and agreed between the Parties. If the Client does not wish the Contractor to proceed as notified, the Client must promptly notify the Contractor not to proceed, and the Contractor must comply with such a notification. The Client and the Contractor may negotiate and agree on some other way of overcoming the Latent Condition, including, but not limited to, having additional necessary work performed by others entities, and to their payment by the Client.
	\item	In case of Latent Conditions, the Contractor is entitled to claim an extension of time for the Construction Period corresponding to the time needed to solve the Latent Conditions. The Contractor is entitled to recover the additional costs incurred directly or indirectly stemming from any Latent Conditions.
	\item	The Contractor is not entitled to claim any adjustment to the Energy Savings Guarantee, to reduce the scope of the Renovation Works, or to modify the agreed Fees in this Agreement due to Latent Conditions.
\end{enumerate}

\subsection{MEASUREMENT AND VERIFICATION AND DATA MANAGEMENT}
\begin{enumerate}
	\item	The Contractor must perform all the Measurement and Verification activities based on the IPMVP (International Performance Measurement and Verification Protocol) adherent Measurement and Verification plan available on the Sunshine platform - sunshineplatform.eu.
	\item	All Measurement and Verification activities should be clearly and fully disclosed and transparent to all the Parties.
	\item	The Contractor during the Service Period shall submit to the Client an Annual report. This Annual report shall document the calculation of the Fees, the Operation and Maintenance Activities carried out by the Contractor and whether the Energy Savings Guarantee has been fulfilled based on Measurement and Verification during the Settlement Period. The report shall provide sufficient information on the Energy Savings resulted from the implemented Measures and on the calculation made for the determination of the Energy Savings. The Annual report shall be sent by the Contractor to the Client every year no later than 20 (twenty) Business days after the end of the Settlement Period. This process can be delivered via the Sunshine platform - sunshineplatform.eu.
	\item	If the Client has objections on the conclusions made in the Annual report, the Client shall inform the Contractor accordingly within 15 (fifteen) Business days upon receipt of the report or receipt of the Notification from the Sunshine platform - sunshineplatform.eu. The Client shall provide the Contractor with reasons for its objections; The Contractor shall, within the following 15 (fifteen) Business days, as of the receipt of the Assignor’s objections, make the necessary amendments and report and inform the Client accordingly.
	\item	Any unjustified interference or tampering by the Client with the Measures implemented in the Building resulting in a decrease in the level of Energy Savings shall be taken into account during Measurement and Verification of the Energy Savings Guarantee of the Agreement and serve to readjust the Guarantee on a pro rata basis.
	\item	The Client acknowledges and consents to the use, by the Contractor or any other third party assigned by the latter with the rights and obligations under this Agreement:
	\begin{enumerate}
		\item	of any anonymous data and information relating to the energy consumption of the Building, whether provided by the Client or obtained by the Contractor, for the purposes of benchmarking and compilation of national, regional or international wide database or for the purposes of use by the Contractor as a reference or for any internal purpose agreed with the Client;
		\item	of personal data provided by the Client, or its Manager, for the purposes of rendering its Services and to transfer the latter to any third party that may be assigned with rights or obligations stemming from this Agreement, including to any party forfaiting the receivables stemming under this Agreement or managing or being in charge of the development, implementation, operation and maintenance of an online Sunshine platform - sunshineplatform.eu keeping track of the performance of the implemented Measures.
	\end{enumerate}
	\item	For any reasonable period of time, the Contractor, at its discretion, alone or through its assignees, is entitled to install, operate, service and introduce an energy management system or in general measurement instrumentation and access such installed equipment at all reasonable times in accordance with the Agreement.
	\item	The Contractor is entitled to install temperature loggers in Apartments if complaints are received about nonconformity with the Comfort Standards. If the Apartment Owners of the Building do not agree with the installation of the said logger in their Apartments or fail to provide reasonable access to such installation, the Contractor shall not be liable for alleged improper fulfillment of the Agreement with regard to such Apartments.
	\item	Data collected by the Contractor measurement instrument are of informative nature and cannot be recognized as the basis for establishing a breach of the Agreement or conformity with the Comfort Standards in case of disputes.
\end{enumerate}

\subsection{DISPUTE RESOLUTION PROCEDURE}
\begin{enumerate}
	\item	Any disagreements between the Parties shall first be negotiated. To this end, the Parties shall provide written acknowledgment to any letter from another Party concerning a disagreement, as well as devote reasonable time to resolve the disagreement in person or through a senior representative of the Parties in disagreement.
	\item	If the Client (or a single Apartment Owner) has complaints about the Contractor (for example on the Comfort Standards or on the amount of Energy Savings, or in general on the implemented Measures and provided Energy Efficiency Services), the latter shall directly or via the Manager notify the Contractor thereof. The Contractor shall verify and prepare a Statement of the complaint and remedy the problems that have occurred. If the problem persists for more than 20 (twenty) Business days from the notification the Client shall arrange a committee comprised of duly authorized representatives of the Contractor, the Manager and the Client to meet and to prepare a draft Statement based on the complaint and facts verified on the site or undertake any other fact-finding procedure according to the mediation rules available on the Sunshine platform - sunshineplatform.eu.
	\item	If the Contractor has complaints about the Client (for example for damaging installed equipment), the latter shall notify the Client and the Manager thereof. The Client shall verify and possibly identify the perpetrator and prepare a Statement of the complaint and remedy the problems that have occurred. If the problem persists for more than 30 days from notification, the Contractor shall arrange a committee comprised of duly authorized representatives of the Contractor, the Manager and the Client to meet and to prepare a draft Statement based on the complaint and facts verified on the site or undertake any other fact-finding procedure according to the mediation rules available on the Sunshine platform - sunshineplatform.eu.
	\item	For the fact-finding procedure applicable in case of disputes pertaining to facts, the following shall apply:
	\begin{enumerate}
		\item	The actual Comfort Standards (ambient temperature of the individual Apartments in the Building) shall be considered duly recorded if the temperature measurements are conducted by an independent certified energy auditor (according to MK Nr. 382) and in compliance with LVS EN 12599 standard. The Statement shall be executed based on the findings measured by the independent certified energy auditor;
		\item	General problems with the implemented Measures, like for example malfunctioning equipment and/or defect and damages made to the Measures, or with the calculation of the Energy Savings, shall be considered duly recorded if reported by an independent expert like a certified energy auditor (according to MK Nr. 382);
		\item	All Parties shall be notified at least 5 (five) Business days before any measurement made by a third party. An authorized representative of the Parties has the right to participate in the measurement process for the preparation of the Statement. The absence of the Authorized representatives of any of the Parties is not an obstacle to the preparation and execution of the Statement by the Parties;
		\item	The signing of the Statement by any of the Parties shall not be considered as an acknowledgment of a breach under this Agreement and/or shall not be deemed as a waiver of any of the Parties’ rights and obligation hereunder. The costs for the independent third parties shall be evenly shared between the Parties;
		\item	One copy of each Statement executed shall be delivered to the Contractor, the Manager and the Apartment Owner having filed the complaint.
	\end{enumerate}
	\item	If the Parties fail to reach an agreement, the Parties shall enter into a formal mediation process according to the mediation rules available on the Sunshine platform - sunshineplatform.eu, effective during the validity period of the Agreement and as existing at the time of the dispute. If there is a dispute between the Parties regarding technical issues, any Party may request the dispute on established facts to be resolved pursuant to the procedural rules of the fact-finding committee available on the Sunshine platform - sunshineplatform.eu.
	\item	If the Parties fail to reach a mutual agreement after the mediation process and/or fact-finding process the dispute shall be resolved by a court of general jurisdiction of Romania in accordance with applicable laws and regulations effective in Romania. An application shall be lodged at the court according to the jurisdiction of the defendant’s place of residence or registered address; however, if it is not Romania, then at the Riga City Centre District Court or the Riga Regional Court.
\end{enumerate}

\subsection{MAINTENANCE OF THE MEASURES IMPLEMENTED BY THE CONTRACTOR}
\begin{enumerate}
	\item	The Contractor shall replace, or repair, or overhaul the equipment (or any part thereof) installed as part of the Renovation Works upon expiry of the period of their useful life (as determined by the Operation and Maintenance Manual) during the Service Period of the Agreement.
	\item	The Contractor shall implement maintenance procedures for the Measures that corresponds or exceeds manufacturer’s requirements and recommendations for their respective maintenance and in accordance with the Specific Conditions of this Agreement.
\end{enumerate}

\subsection{INSURANCE}
\begin{enumerate}
	\item	Upon commencement of the Construction Period, the Contractor shall insure the Building for the amount not less than the restoration value of the Building with minimum insurance coverage against fire, earthquake, flood, water damage, any other natural disasters having an impact on the Building, structural damage caused by subsidence and fallen trees. For this insurance the following provisions apply:
	\begin{enumerate}
		\item	The Contractor shall conclude this insurance with an insurer rated minimum A+ according to relevant ratings applicable to Romania.
		\item	The Contractor shall submit a copy of this insurance policy and the documents confirming payment of the insurance premium to the Client before the commencement date of the Construction Period
		\item	Client shall be indicated as the beneficiary in case of payment of insurance coverage in an amount at least sufficient to recover the restoration value of the Building.
		\item	The construction works in the Building shall not start until the Contractor provides a validly conclude insurance policy.
		\item	The Contractor shall maintain the insurance policy during the validity of the Agreement and upon request of the Client shall present to the Client the original of such insurance policy or submit a copy of the certificate of the policy, or give access to the document via the Sunshine platform - sunshineplatform.eu or other conclusive documents confirming currency and payment of the insurance premium.
		\item	The Contractor provides the Building insurance at its own cost and expenses for the entire Construction Period. After the completion of the Renovation Works and upon signing of the Delivery and Acceptance protocol, the insurance costs for the remaining period of validity of the Agreement shall be divided between the Apartment Owners in proportion to the Apartment area owned in the Building and included in the Contractor’s invoices for operation and maintenance. The Manager appointed by the Parties shall ensure the Client’s invoices issued to Apartment Owners include such insurance costs.
	\end{enumerate}
	\item	Additionally, for the Construction Period, the Contractor shall maintain a valid insurance policy for civil and professional liability insurance in an amount not less than 110\% of the total investment costs for the Renovation Works.
\end{enumerate}

\subsection{ASSIGNMENT OF CLAIMS}
\begin{enumerate}
	\item	The Contractor is entitled to the unlimited right to assign to third persons its rights and claims to any of the receivables due by the Client under this Agreement. In particular, the Contractor is entitled to assign the receivables from the Renovation Fee to any Assignee that has entered into a financing, forfaiting, cession or any other arrangement with the Contractor.
	\item	The Assignment of Claims does not relieve the Contractor from its obligation and liabilities under this Agreement. However, the Assignee has step-in rights in this Agreement in case the Contractor is not fulfilling its obligations under this Agreement. Step-in rights shall only aim to replace a Contractor defaulting with another entity able to fulfill all obligations and liabilities under this Agreement to the benefit of the Client and Assignee.
	\item	In the case of such an assignment, the Contractor shall send a Notice to the Client within 5 (five) Business days from conclusion of such assignment agreement.
	\item	This Agreement is personal to the Client and may not be assigned or transferred by the Client without the prior notification to the Contractor.
	\item	If the Contractor undergoes any merger or acquisition, enters into liquidation or bankruptcy proceedings, the Agreement remains valid, and its provisions are binding on legal successors and assignor of the Contractor.
\end{enumerate}

\subsection{TITLE TO THE MEASURES INSTALLED IN THE BUILDING AS PART OF THE RENOVATION WORKS}
\begin{enumerate}
	\item	The title to the Measures that can be separated from the Building without causing material damages belongs to the Contractor, if the Contractor provides a Financial Contribution as part of the Renovation Works. If the Renovation Works are fully financed by the Client, the title of the Measures belongs to the Client.
	\item	The Client may not, and shall take all reasonable measures to ensure that none of the Apartment Owners or other visitors, remove, encumber (lease or rent among other), pledge or destroy, damage or tamper with the Measures implemented as part of the Renovation Works independently on the Party with title on the Measure.
	\item	The Contractor is entitled, without the approval of the Client (namely without the need to obtain the consent of each of the Apartment Owners), to pledge and encumber the Measures (or parts of) to its sole or third-party benefit, if:
	\begin{enumerate}
		\item	the Contractor has title over them;
		\item	technically it is possible to dismantle them without materially damaging the Building;
		\item	the pledge and/or encumbrance is necessary as collateral for the Contractor Financial Contribution under this Agreement. The Contractor is not entitled to pledge the Measures to raise financial resources for purposes other than for the performance of this Agreement; and
		\item	the duration of the pledge and/or encumbrance does not exceed the term of the Agreement.
	\end{enumerate}
	\item In case the Contract has title on the Measures, upon receipt of all payments due under this Agreement to the Contractor, the title of all Measures implemented as part of the Renovation Works in the Agreement shall be considered automatically transferred to the Client. This eventually against a non-refundable price of 1 EUR (one euro) payable in advance at the signature of this Agreement. Transfer of title to the Client shall be verified by a Statement of transfer signed by the Contractor and the Client.
\end{enumerate}

\subsection{SOFTWARE AND INTELLECTUAL PROPERTY RIGHTS}
\begin{enumerate}
	\item	The Contractor shall ensure that all intellectual and industrial property rights (IPR) for the Measures implemented in the Building, including equipment, materials, systems, software or any other thing or document supplied by the Contractor to the Client under this Agreement, are owned by or licensed to the Contractor. The Parties agree that such rights remain the property of the Contractor and do not pass to the Client. The Contractor grants the Client a perpetual, irrevocable, non-exclusive royalty-free licence (with a right to sub-licence) to use the said IPR in connection with the use of the Building and not for any other use.
	\item	The Client must not modify, copy or reverse engineer any software or merge it with any other software, which the Contractor provided as part of the Renovation Works. During the term of this Agreement, the Contractor will provide to the Client user manuals, technical information and all updates and revisions of provided software.
	\item	The Contractor indemnifies the Client against any claims, which the Client is legally liable for, in respect of any infringement of third party intellectual property rights relating to any part of the IPR supplied by the Contractor. The Contractor’s obligation to indemnify the Client against such claims is subject to the Client:
	\begin{enumerate}
		\item	giving the Contractor prompt written notice of the claim;
		\item	not making any admission or prejudicing the Contractor’s defence of the claim or the Contractor’s ability to negotiate a satisfactory settlement;
		\item	allowing the Contractor the opportunity to control at the Contractor’s expense the conduct of the defence and any negotiations for the settlement of the claim; and
		\item	giving the Contractor (at the Contractor’ expense) such assistance and information as may reasonably be required by the Contractor to assist the Contractor with the conduct of the defence and any negotiations for the settlement of the claim.
	\end{enumerate}
	\item	The Contractor shall, at its option, either replace or modify the IPR infringing part with a non-infringing part or procure for the Client the right to use such infringing part. The remedies set out in this Clause shall be the sole and exclusive remedy for infringement of intellectual property rights.
\end{enumerate}

\subsection{CHANGES IN USE OF THE BUILDING}
\begin{enumerate}
	\item	The Building is described in the Specific Conditions of the Agreement including its use, area and size. If any circumstances, on which calculations of the Contractor were based, change at the initiative of the Client, or with the Client’s consent or allowance, the change shall not affect the Contractor and the performance of the Agreement. Changes in use of the Building and modification of the Building shall be evaluated in view of economic considerations (particularly changes in costs), and the Agreement shall be adjusted to the new circumstances accordingly.
	\item	Changes in use of the Building include:
	\begin{enumerate}
		\item	Expansion or reduction of the surface area of the Building;
		\item	Assembly, damaging or dismantling of respective equipment or other installations if it results in a material increase or decrease of the energy consumption or other technical parameters of the Building;
		\item	Changes in the use of the Building (for example apartment area is turned into shops, stores, restaurants, and offices or unused / uninhabited apartments start being used) affecting energy consumption of the Building.
	\end{enumerate}
\end{enumerate}


\subsection{DISPOSAL OF DISCONNECTED AND/OR DISMANTLED EQUIPMENT and MATERIALS}
\begin{enumerate}
	\item	The Contractor shall arrange, at its own costs, the disposal of waste generated under the scope of this Agreement in compliance with relevant law and regulations on waste disposal of the Republic of Romania.
	\item	The Contractor shall notify in writing the Client not later than within 5 (five) Business days prior to the first planned disposals activity. Such notification covers all equipment, material and other assets installed in the Building and subject to be dismantled and replaced for the implementation and installation of the Measures during the Construction Period.
	\item	If the Client wishes to use any of the equipment, material or other assets disconnected or dismantled by the Contractor during the Construction Period it should notify the Contractor and arrange at its own costs pick up and transport.
\end{enumerate}

\subsection{LIABILITIES}
\begin{enumerate}
	\item	The Contractor is liable for the timely implementation of the Measures in the agreed Construction Period. Failures of the Contractor to comply with this liability entitles the Client to liquidated damages at the rate of 0.02\% per day of the total planned investment costs. Liquidated damages may not exceed 10\% (ten percent) of the planned investment costs.
	\item	The Client is liable for timely payment of the due costs and Fees under this Agreement. The Contractor is entitled to claim compensation for late payment. Compensation for late payments correspond to 0.1\% per day of the overdue amount.
	\item	If the Client fails to make any payment due and arising out of the Agreement for more than 90 (ninety) days, in the course of which Dispute Resolution Procedures under this Agreement were duly and effectively deployed, the Contractor is entitled to terminate the contract due to the Client’s default and breach of the Agreement.
	\item	The Contractor is liable for keeping the Building at the Comfort Standards set in this Agreement. If during the Heating Season the indoor temperature has been on average 2 (two) degrees on the Celsius scale (considering the accuracy of the instrumentation) below the Comfort Standards set in this Agreement in any of the Apartments, the Contractor will be responsible to instruct the Manager to reduce the Client’s invoice for the respective Apartment Owner as follows:
	\begin{enumerate}
		\item	The discount of 5\% (five percent) for each degree Celsius of the Energy Fee for each month of the Heating Season when the temperature was below the agreed Comfort Standards;
		\item	The determination of the indoor temperature and whether the temperature level was below the Comfort Standards shall be determined in accordance with the Dispute Resolution Procedures of this Agreement.
		\item	The Contractor shall not apply the discount if the decrease in indoor temperature in the Apartment has occurred: (i) as a result of actions or omissions of the occupants or of the Apartment owners in breach of this Agreement; (ii) the decrease is a consequence of non-performance of the Client’s obligations; or (iii) as the result of other reasons not attributable to the fault of the Contractor.
	\end{enumerate}
	\item	The Client is liable for damages, manipulation or tampering, vandalism, sabotage, theft (unless by the Contractor or by those for whom the Contractor is responsible) with the Measures, in particular if affecting the level of Energy Savings, agreed Comforts Standards or the safety of the persons living and using the Building. In this case, the Client shall:
	\begin{enumerate}
		\item	fully compensate the Contractor for the costs of restoring the relevant Measure;
		\item	pay compensation to the Contractor at the rate of 10\% (ten percent) of the costs of restoration to cover the administration costs of the Contractor;
		\item	The costs for restoration shall be calculated on the basis of the applicable market prices relevant at the time of calculations;
		\item	the determination of the Client liability for damages, manipulation or tampering with the Measures shall be determined in accordance with Dispute Resolution Procedure of this Agreement.
	\end{enumerate}
	\item	The Contractor shall hold the Client harmless from any liability, costs, expenses, damages, fees it may incur as a result of a claim or complaint, administrative or court actions brought against the Client by state or administrative authorities or third parties arising from the actions of the Contractor or stemming from any potential breaches of intellectual property rights pertaining to the Measures having been executed by the Contractor. The Client shall receive from the Contractor reimbursement for all costs and expenses reasonably required for the reparation of all direct damages resulting from the actions of the Contractor undertaken in breach of any applicable legislation. The reimbursement shall be due documented and payable within 30 Business Days as of the receipt of the respective demand from the Client to the Contractor expressly indicating the amount due.
	\item	Charges for the provision of services of general economic interest (including heat supply) and sanctions applicable to any Party set out in the laws and regulations effective in the Republic of Romania shall not limit the obligations of the Parties under this Agreement, including liability of the Parties and payments of applicable contractual liquidated damages, compensations and penalties under this Agreement.
	\item	Payment of liquidated damages, compensations and penalties does not release the incurring Party from fulfilling its obligations under the Agreement.
\end{enumerate}

\subsection{TERMINATION OF THE AGREEMENT}
\begin{enumerate}
	\item	In the event that either Party breaches a material provision of this Agreement, the non-defaulting Party may terminate this Agreement immediately and require the defaulting Party to indemnify the non-defaulting Party according to this Agreement.
	\item	Termination of the Agreement before the Commencement Date and investment for construction and installation works have not occurred:
	\begin{enumerate}
		\item	In case of unilateral termination of the Agreement by the Client due to a material default or breach of the Agreement by the Contractor, the Client is entitled to compensation in the amount of 1\% of the Investment Costs (excluding VAT) planned in the Agreement.
		\item	In case of unilateral termination of the Agreement by the Contractor due to default or a breach of the Agreement by the Client, the Contractor is entitled to compensation in the amount of 1\% of the Investment Costs (excluding VAT) planned in the Agreement.
		\item	In case of unilateral termination of the Agreement by the Client due to other business or commercial reasons not necessarily linked with this Agreement, the Contractor is entitled to compensation in the amount of 1\% of the Investment Costs (excluding VAT) planned in the Agreement.
		\item	In case of unilateral termination of the Agreement by the Contractor due to other business or commercial reasons not necessarily linked with this Agreement, the Client is entitled to compensation in the amount of 1\% Investment Costs (excluding VAT) planned in the Agreement.
	\end{enumerate}
	\item	Termination of the Agreement after that investment costs for construction and installation works of the Measures have occurred and have been covered by the Contractor Financial Contribution:
	\begin{enumerate}
		\item	In case of unilateral termination of the Agreement by the Client due to a material default or a breach of the Agreement by the Contractor, the Client shall only reimburse the outstanding amount of the Financial Contribution made by the Contractor discounted by 3\%, which is functioning properly in accordance with the performance criteria set out in the Agreement. In addition to the above, the Client is entitled to receive the entire project documentation providing in detail the work having been executed thus far, along with all permits, licenses or other documents having been obtained by the Contractor pursuant to this Agreement and completion of most urgent works, all manufacturers warranties, any sub-licenses (and transfer of any licenses) for the use of necessary intellectual property rights and software (including, software having been installed and as applicable, any accompanying documentation, information related to code, any source code, data files, calculations, electronic media, print outs, or related information), additional training to any third party having been expressly appointed by the Client in case the implementation of the Renovation Works has been completed.
		\item	In case of unilateral termination of the Agreement by the Contractor due to a material default or a breach of the Agreement by the Client, the Client shall reimburse the outstanding amount of the Financial Contribution plus a compensation at a rate of 3\% of the amount being repaid. The Client is entitled to receive the entire project documentation providing in detail the work having been executed thus far, along with all permits, licenses or other documents having been obtained by the Contractor pursuant to this Agreement, all manufacturers warranties, any sub-licenses (and transfer of any licenses) for the use of necessary intellectual property rights and software (including, software having been installed and as applicable, any accompanying documentation, information related to code, any source code, data files, calculations, electronic media, print outs, or related information).
		\item	In case of unilateral termination of the Agreement by the Client due to other business or commercial reasons not necessarily linked with this Agreement, the Contractor shall be entitled to a compensation corresponding to the outstanding principal of the Financial Contribution made by the Contractor plus a compensation at a rate of 3\% of the amount being repaid.
		\item	In case of unilateral termination of the Agreement by the Contractor due to other business or commercial reasons not necessarily linked with this Agreement, the Contractor shall be entitled to a compensation corresponding to the outstanding principal of the Financial Contribution made by the Contractor discounted by 3\% of the calculated amount.
	\end{enumerate}
	\item	The Contractor or any of its assignees shall issue an invoice to the Client for the calculated compensation indicating clearly, the information used for such calculation based on the Payment Schedules issued during the Service Period or the invoices paid during the execution of the Renovation works before the termination date of the Agreement. The Client shall pay within 60 (sixty) days of the date of issuance of the invoice for the compensation due to the Contractor or any of its receivers, assignees or such other entities unilaterally identified as being legally entitled to all or some of the rights of the Contractor under this Agreement.
	\item	Early termination of the Agreement shall be notified by the Party terminating the Agreement in writing (notice of termination) at least 20 (twenty) Business days in advance. In case the Termination of the agreement is due to default or a breach of the Agreement by a Party, a valid written notice of termination must include the steps undertaken under the Dispute Resolution Procedures of this Agreement and the related supportive documentation.
	\item	The Client is entitled at any time to claim and to receive from the Contractor a calculation of the amount to be compensated to the Contractor in case of early termination of the Agreement.
	\item	In general, the termination of the Agreement does not release the Parties from the fulfillment of the obligations set forth in the Agreement which have occurred before the time of termination of the Agreement unless the Parties have agreed in writing on other provisions or the Agreement provides otherwise. In particular the unilateral termination of the Agreement by the Client in case of a material default or a breach of the Agreement by the Contractor does not by itself release the Client from the obligations for payments of the Invoices issued for the periods preceding the date of termination of the Agreement.
	\item	Reorganization, change of shareholders and/or ownership, change in management of the Parties including changes in Apartment Owners of the Building, does not serve as grounds for termination of the Agreement or non-performance of the obligations contained in the Agreement.
	\item	In addition to the provisions of the Agreement, the Parties may terminate the Agreement at any time upon mutual written arrangement on the conditions for termination.
	\item	The Party entitled to compensation shall seek compensation either exercising its rights under this Agreement or under relevant laws and regulations effective in the Republic of Romania. However, the entitled Party shall not receive double compensation for the same default or breach.
	\item	The Parties may agree to dismantle the Measures owned or partly owned by the Contractor from the Building if the Agreement is terminated early due to any circumstances and if the value of the respective Measures is agreed and counted as compensation by the affected Party. The possibility to dismantle the Measures in the circumstances set out in this paragraph, is without prejudice to any claims for damages, costs, and expenses that any of the Parties may be entitled to for early termination of this Agreement.
\end{enumerate}

\subsection{FORCE MAJEURE EVENTS}
\begin{enumerate}
	\item	Any emergency situation or event unforeseeable in advance characterized by all below features shall be defined as force majeure:
	\begin{enumerate}
		\item	the Parties are not able to predict and influence it;
		\item	it interferes with the performance by the Parties of their obligations;
		\item	it cannot be qualified as an error or negligence made by any of the Parties;
		\item	it can be proven or recognized as insurmountable although the Party/ies has/have made reasonable efforts to prevent it.
	\end{enumerate}
	\item	Force majeure events shall include, but not be limited, to: warfare, natural disaster and legal acts of state administration.
	\item	Are NOT Force Majeure Events: defects of the Measures, services lacking the agreed quality or quantity, equipment or materials used, provided by or installed by the Contractor or delay in operation thereof (if not caused by force majeure events), Client disputes, strikes, financial difficulties or such others specific to the party relying on a force majeure event.
	\item	The Parties shall not be liable for complete or partial non-performance of obligations under the Agreement if due to force majeure events. The Party referring to the force majeure event shall provide evidence of it to the other Party.
	\item	The Party (the “Affected Party”) prevented from carrying out its obligations hereunder shall give notice immediately and no later than within 3 (three) Business Day to the other Party of an Event of Force Majeure upon it being foreseen by, or becoming known to, the Affected Party describing the situation that will occur, or has occurred providing a description of the event, possible duration, expected consequences and prospective solution thereof.
	\item	The Parties shall perform any necessary activities jointly or severally to mitigate the impact of the force majeure event and take reasonable steps to alleviate any damage caused.
	\item	If the force majeure event continues for more than 6 (six) uninterrupted months and cessation thereof is not expected for another 3 (three) months, the Contractor or the Client has the right to terminate the Agreement unilaterally.
\end{enumerate}

\subsection{CONFIDENTIALITY}
\begin{enumerate}
	\item	Information obtained in the course of conclusion or during the performance of the Agreement that is not generally available to third parties and the disclosure of which the receiving Party is aware, or should have been aware, may damage lawful rights or interests of the disclosing Party, shall be deemed to be confidential.
	\item	The Parties agree not to disclose Confidential Information of the other Party to third parties as well as not to disclose data of the other Party which might be used for the purpose of competition or committal of unlawful activities both while the Agreement is valid as well as for 3 (three) years after the Agreement loses its validity.
	\item	Information which has come into the public domain by third parties without the Parties breaching provisions of the Agreement shall not be deemed confidential.
	\item	Parties may disclose the confidential information to third parties for to fulfill the obligations of the Agreement. If the Parties divulge Confidential Information based this Clause, they shall ensure that the third party will comply with the same confidentiality obligations as established in this Agreement.
	\item	Disclosure of confidential information required in accordance with laws and regulations effective in the Republic of Romania shall not be deemed a breach of the Agreement.
	\item	For advertising purposes and for the purpose of informing the general public, the Contractor, all its assignees, and the Client are entitled to disclose general information about mutual cooperation, including inter alia: divulge information already in the public domain about the Parties, nature of cooperation, achieved Energy Savings and Energy Consumption data. This as far as the disclosure of the information does not infringe lawful rights and interests of the other Party concerning the protection of confidential information. If a Party has doubts about the nature of the specific information, before disclosure thereof, such nature of information shall be approved by the Party(ies) whose lawful rights and interests might be infringed by disclosure of this information if such Party considers such information is not within the obligation of confidentiality in the Agreement. %chktex 36
	\item	The above is without prejudice to the Client express obligation not to request or advise any Client, potential Client or business contact of the Contractor and/or any developing entities to curtail, cancel, withdraw, limit, reduce or otherwise restrict their business from the Contractor.
	\item	The above provisions are without prejudice to the right of the Contractor to collect, process, store, transform, transfer to its assignees of funding partners and disseminate all of the collected data from the Client for the purposes of improving the quality of its Services and for development, operation and maintenance of the online Sunshine platform - sunshineplatform.eu supporting all phases and participating parties in a typical EPC Project.
\end{enumerate}

\subsection{CONCLUSION AND AMENDMENT OF THIS AGREEMENT}
\begin{enumerate}
	\item	The Agreement comes into effect on the day when signed by all Parties under the present General Terms and Conditions and shall remain in force until complete fulfillment of all its obligations by the Parties.
	\item	All modifications, supplements, and amendments to the Agreement shall be made in writing upon mutual agreement of all the Parties and, shall take effect after signature by all Parties and shall be attached to the present Agreement in the form of annexes.
	\item	All remaining provisions of the terms and conditions or the respective Annexes shall remain in full force and effect. Any deviations agreed shall apply only to the part of the Agreement for which said deviations have been agreed.
	\item	The Agreement shall be deemed terminated after the Client has discharged himself of all payments of the Fees to the Contractor in full and the Contractor has fulfilled its obligations.
	\item	If during the validity period of the Agreement, amendments to the laws and regulations of Romania come into effect that renders the fulfillment of any obligations under this Agreement completely or partially impossible, it shall not affect the validity of the remaining obligations under this Agreement. In this case, the Parties shall introduce suitable amendments to the Agreement with the intent and purpose to minimise the economic impact of the Parties in the Agreement.
\end{enumerate}

\subsection{REPRESENTATION OF THE PARTIES}
\begin{enumerate}
	\item	For this Agreement, the Parties shall be represented by their legitimate representatives (for legal entities) or persons specifically indicated in this Agreement. Only the persons referred under the Specifics Conditions of this Agreement are entitled to represent the Client or the Contractor respectively.
	\item	The Agreement is drawn up and signed in 3 (three) original copies in Romanian with equal legal effect. The Parties certify with its signing that they understand the content, meaning and consequences of the Agreement; they acknowledge this Agreement to be correct, mutually beneficial, entailing all provisions, promises, conditions, and representations of intentions between the Parties and that they voluntarily wish to execute it without exertion of any duress over the will of any of them.
\end{enumerate}

\subsection{NOTICES}
\begin{enumerate}
	\item	Form of Notice: all notices, requests, claims, demands and other communications required or permitted by the terms of this Agreement shall be given and delivered to the Parties addresses.
	\item	Method of Notice: all notices shall be given (i) by delivery in person (ii) by a next day courier service, (iii) by registered or certified post mail, (iv) by facsimile, (v) by electronic mail (email) with request of Delivery Receipt to the address of the Party specified in this Agreement or such other address as either Party may specify in writing, (vi) by the notification services given by the Sunshine platform - sunshineplatform.eu. (vii) for Notices with a small amount of information SMS messages with receipt confirmation sent to the mobile number of the Party specified in this Agreement or such other numbers as either Party may specify in writing.
	\item	Receipt of Notice: all notices shall be effective upon (i) receipt by the Party to which notice is given, or (ii) on the 7th (seventh) day following mailing, whichever occurs first
\end{enumerate}

\end{multicols}
