{{template "preamble.tex"}} % chktex 18
\begin{document}

\begin{center}
	\begin{tabu}{|X[2]|X|}\tabucline{}
		Номер на договора: \iffalse input fields.contract_id value="{{.Contract.Fields.contract_id}}" \fi {{.Contract.Fields.contract_id}} & Дата: {{index .Contract.Fields "date"}} \\\tabucline{} %chktex 26
	\end{tabu}
\end{center}

\section{ДОГОВОР С ГАРАНТИРАН РЕЗУЛТАТ}

\textbf{Клиент}:

\begin{center}
	\begin{tabu}{|X|X[2]|}\tabucline{}
		Име & {{.Contract.Fields.client_name}} \iffalse input fields.client_name value="{{.Contract.Fields.client_name}}" \fi \\\tabucline{}
		ЕИК / ЕГН & {{.Contract.Fields.client_id}} \iffalse input fields.client_id value="{{.Contract.Fields.client_id}}" \fi \\\tabucline{}
		Адрес & {{.Asset.Address}} \iffalse input fields.client_address value="{{.Contract.Fields.client_address}}" \fi \\\tabucline{}
	\end{tabu}
\end{center}

\textbf{Изпълнител}:

\begin{center}
	\begin{tabu}{|X|X[2]|}\tabucline{}
		Име & {{.ESCo.Name}} \\\tabucline{}
		ЕИК & {{.ESCo.VAT}} \\\tabucline{}
		Адрес & {{.LEAR.Name}} \\\tabucline{}
    Упълномощено лице & {{index .Contract.Fields "legal-representative"}} \\\tabucline{}
	\end{tabu}
\end{center}

Сключиха настоящия договор с гарантиран резултат.

\section{СПЕЦИФИЧНИ УСЛОВИЯ}
\subsection{ПРЕДМЕТ}
\begin{enumerate}
\item Клиентът възлага, а Изпълнителят приема да изпълни за
  многофамилната жилищна сграда, разположена на адрес: {{.Asset.Address}},
  кадастрален идентификатор {{.Asset.Cadastre}} следните дейности.
  \begin{enumerate}[label=\arabic*.]
  \item Изпълнение (проектиране и изпълнение на строително-монтажни
    работи) на енергоспестяващи мерки в съответствие с доклада от
    извършеното обследване за енергийна ефективност, Приложение 1 {-}
    Доклад и резюме от обследване за ЕЕ за сградата (вж. Приложение 1);
    към настоящия договор в обхват, съгласно Приложение 2 {-} Обхват на
    мерките и КСС (вж. т. (1) и (2) ); към настоящия договор;
  \item Обслужване и поддръжка на осъществените енергоспестяващи мерки
    за целия срок на договора и мониторинг на енергийното потребление,
    с оглед отчитане на икономиите от енергия и постигането на
    гарантирания резултат.
  \end{enumerate}
\item Изпълнителят се задължава да изпълни всички ЕСМ в следствие на
  които, сградата ще има специфичен разход на енергия, който ще
  отговоря на изискванията за клас на енергопотребление „[●]”.
\item Икономиите на енергия и постигането на гарантирания резултат се
  отчитат чрез Методика за отчитане на гарантирания резултат,
  Приложение 4 {-} Методика за отчитане на гарантирания резултат
  (вж. Приложение 4);. Методиката е разработена от Изпълнителя.
\item Подробният обхват на ремонтните дейности и свързаните с тях
  мерки са описани в Приложение 2 {-} Обхват на мерките и КСС (вж. т. (1) и (2)); към договора.
\end{enumerate}

\subsection{ГАРАНТИРАН РЕЗУЛТАТ}
\begin{enumerate}
\item Страните по договора договорят следните числени стойности на
  БГПЕ, ГГПЕ и съответните им ГГИЕ и екологичен еквивалент на
  спестената енергия:
  \begin{enumerate}
  \item БГПЕ: [●]kWh/м2, в това число: (а) топлинна енергия {–} [●]
    kWh/година и (б) електрическа енергия {–} [●] kWh/година
  \item ГГПЕ: [●] kWh/m2 и [●] kWh/година, включително топлинна
    енергия {-} [●] kWh/година, електрическа енергия {-} [●] kWh/година,
  \item ГГИЕ: [●] kWh/година, включително: топлинна енергия {-} [●]
    kWh/година, електрическа енергия {-} [●] kWh/година,
  \item Екологичен еквивалент на ГГИЕ от сградата: [●] тона въглероден
    диоксид/година.
  \item Цена на топлинна енергия към датата на подписване на договора
    [●].
  \item Цена на електрическа енергия към датата на подписване на
    договора [●].  При отчитане на СГГИЕ и СДГПЕ за целия срок на
    договора се прилагат цените на енергийните ресурси по т. (5) и (6)
    (вж. Приложение 4).
  \end{enumerate}

\item Посочените в т. 1) числени стойности на енергийния разход са
  определени съгласно доклада от извършеното енергийно обследване и
  информацията, предоставена от клиента, както и съобразно състоянието
  на външните елементи на сградната конструкция, характеристики на
  сградната инсталация, режим на експлоатация, брой обитатели и
  инсталирани енергийни консуматори и събраните данни от Изпълнителя
  при извършения оглед на сградата и при фактори на климатичната среда
  характерни за Климатична зона № [●], съгласно Наредба №
  Е{-}РД{-}04{-}2/22.01.2016 г. за показателите за разход на енергия и
  енергийните характеристики на сградите.
\item Изпълнителят за периода на мониторинг гарантира стандартите за
  комфорт, описани в Приложение 3 {-} Стандарти за комфорт (вж. Приложение 3);.
\item Изпълнителят за периода на мониторинг осъществява обслужване и
  поддръжка на мерките в съответствие с разпоредбите, подробно описани
  в Приложение 5 {-} Ръководство за експлоатация и поддръжка
  (вж. т. (1) , (2) и (3) );.
\end{enumerate}

\subsection{СРОК НА ДОГОВОРА}
\begin{enumerate}
\item Настоящият договор влиза в сила от датата на подписването му от
  страните. Срокът за изпълнение на договора е XXXXX години /месеци/ и
  включва срок на изпълнение на Етап 1 и срок на изпълнение на Етап 2
  по-долу.
\item Етап 1 започва от подписване на настоящия договор и включва
  времето, необходимо за проектиране на мерките, и времето през
  периода на строителството, и приключва при въвеждане на мерките в
  експлоатация.
  \begin{enumerate}
  \item Периодът на строителството е XXXXX месеца.
  \item Срокът за изпълнение на етап 1 е XXXX месеца.
  \end{enumerate}
\item Етап 2 започва от датата на въвеждане на експлоатация и включва
  периода на мониторинг. Периодът на мониторинг е XXXX.
\end{enumerate}

\subsection{ЦЕНА}
\begin{enumerate}
\item Клиентът заплаща на Изпълнителя цена за изпълнение на ЕСМ, цена
  за обслужване и поддръжка и цена за мониторинг за целия срок на
  договора, както следва:
  \begin{enumerate}
  \item Цена за изпълнение на ЕСМ {-} [●] лева без ДДС и [●] лева с
    ДДС. Върху цената за изпълнение на ЕСМ се начислява лихва съгласно
    Приложение 7 {-} Погасителен план и правила за определяне размер на
    дължими суми от всеки собственик на самостоятелен обект
    (вж. 1.1 и 1.2);
  \item Цена за обслужване и поддръжка {-} [●] лева без ДДС и [●] лева с
    ДДС, дължима на месечна база всеки месец, считано от въвеждане в
    експлоатация до изтичане на срока на договора.
  \item Цена за мониторинг {-} [●] лева без ДДС и [●] лева с ДДС,
    дължима на месечна база всеки месец, считано от въвеждане в
    експлоатация до изтичане на срока на договора.
  \item Цените по подт. (ii) и (iii) подлежат на годишна индексация по
    правила съгласно Приложение 8 {-} Правила за индексация на цената за
    обслужване и поддръжка и цената за мониторинг (вж. т. (4) и (5) );
  \end{enumerate}
\item Цената по т. 1) подт. (i) се заплаща на [●] вноски, считано от
  завършването на Етап 1. Плащането се извършва съгласно Погасителен
  план {-} Приложение 7 {-} Погасителен план и правила за определяне
  размер на дължими суми от всеки собственик на самостоятелен обект
  (вж. т. (1) ); до 15-то число на месеца.
\item Сумите по т. 1) подт. (ii) и (iii) се заплащат в срок от {{.Contract.Fields.invoiced_days}} дни
  след издаване на съответните фактури.
\item Конкретното разпределение на задълженията на отделните
  собственици в Сградата и начин на изчисляването му се съдържа в
  Приложение 7 {-} Погасителен план и правила за определяне размер на
  дължими суми от всеки собственик на самостоятелен обект
  (вж. т. (1) );
\end{enumerate}

\begin{center}
	\begin{tabu}{|X|X|X|X|}\tabucline{}\rowfont[c]\bfseries
	{{with translate "bg" .Contract.Tables.summary}} % chktex 26
	{{.Columns | column}} \\\tabucline{}
	{{range .Rows}} % chktex 26
	{{.|row}} \\\tabucline{}
	{{end}}
	\bfseries {{total .}} \\\tabucline{} % chktex 26
	{{end}}
	\end{tabu}
\end{center}

\subsection{ДРУГИ РАЗПОРЕДБИ}
\begin{enumerate}
\item Настоящият договор включва Общите условия {-} Приложение 11 {-} Общи
  условия (вж. 1.1)., специфичните условия и приложенията към
  специфичните условия.
\item Клиентът дава изричното си съгласие Изпълнителят да прехвърли
  и/или възложи за събиране на трето лице всички вземания, произтичащи
  от настоящия договор или част от тях.
\item Неразделна част от този договор са следните приложения,
  подписани от клиента, като с полагане на подписа си по-долу,
  клиентът декларира, че е получил и екземпляр от приложенията,
  включително и общите условия:
  \begin{enumerate}
  \item Приложение 1 {-} Доклад и резюме от обследване за ЕЕ за сградата;
  \item Приложение 2 {-} Обхват на мерките и КСС;
  \item Приложение 3 {-} Стандарти за комфорт;
  \item Приложение 4 {-} Методика за отчитане на гарантирания резултат;
  \item Приложение 5 {-} Ръководство за експлоатация и поддръжка;
  \item Приложение 6 {-} Правила за определяне на балансово плащане след
    измерване и отчитане на резултата;
  \item Приложение 7 {-} Погасителен план и правила за определяне размер
    на дължими суми от всеки собственик на самостоятелен обект;
  \item Приложение 8 {-} Правила за индексация на цената за обслужване и
    поддръжка и цената за мониторинг;
  \item Приложение 9 {-} Определяне на лица за контакт съгласно
    настоящия договор;
  \item Приложение 10 {-} Протоколи от проведени общи събрания на ЕС за
    вземане на решение за сключване и изпълнение на настоящия договор,
    както и всички други протоколи от решения, имащи отношение към
    този договор, включително и проведени след неговото подписване;
  \item Приложение 11 {-} Общи условия.
  \end{enumerate}
\end{enumerate}

\vspace{2cm}
{{template "sign.tex" .}} % chktex 18

{{template "annex1.tex" .}} % chktex 18
{{template "sign.tex" .}} % chktex 18

{{template "annex2.tex" .}} % chktex 18 chktex 26
{{template "sign.tex" .}} % chktex 18

{{template "annex3.tex" .}} % chktex 18 chktex 26
{{template "sign.tex" .}} % chktex 18

{{template "annex4.tex" .}} % chktex 18 chktex 26
{{template "sign.tex" .}} % chktex 18

{{template "annex5.tex" .Contract.Tables}} % chktex 18 chktex 26
{{template "sign.tex". }} % chktex 18

{{template "annex6.tex" .}} % chktex 18 chktex 26
{{template "sign.tex" .}} % chktex 18

{{template "annex7.tex" .}} % chktex 18 chktex 26
{{template "sign.tex" .}} % chktex 18

{{template "annex8.tex" .}} % chktex 18 chktex 26
{{template "sign.tex" .}} % chktex 18

{{template "annex9.tex" .Contract.Fields}} % chktex 18 chktex 26
{{template "sign.tex" .}} % chktex 18

\pagebreak
\section{ПРИЛОЖЕНИЕ 10 {-} ПРОТОКОЛИ ОТ ПРОВЕДЕНИ ОБЩИ СЪБРАНИЯ НА ЕС
  ЗА ВЗЕМАНЕ НА РЕШЕНИЕ ЗА СКЛЮЧВАНЕ И ИЗПЪЛНЕНИЕ НА НАСТОЯЩИЯ
  ДОГОВОР, КАКТО И ВСИЧКИ ДРУГИ ПРОТОКОЛИ ОТ РЕШЕНИЯ, ИМАЩИ ОТНОШЕНИЕ
  КЪМ ТОЗИ ДОГОВОР, ВКЛЮЧИТЕЛНО И ПРОВЕДЕНИ СЛЕД НЕГОВОТО ПОДПИСВАНЕ}

Прилагат се протоколите от общите събрания на собствениците, касаещи
обновяване на сградата и сключване на настоящия договор, плащане на
дължимото на изпълнителя възнаграждение, както и произнасяне на
собствениците по другите съществени условия на договора,
упълномощаване на управителя да извършва определени дейности и
пр. Решенията трябва да бъдат взети при спазване на всички правила в
ЗУЕС и с единодушие от всички собственици съгласно казаното в Анализа,
а протоколите да са с нотариална заверка на подписа.

\begin{center}
\begin{tabu}{ |X|X| }
 \hline
 Aquisition protocol meeting & \url{ {{.Attachments.acquisition_meeting }} } \iffalse attachment value="acquisition meeting" \fi \\
 \hline
 Commitment protocol meeting & \url{ {{.Attachments.commitment_meeting }} } \iffalse attachment value="commitment protocol meeting" \fi \\
 \hline
 Kickoff protocol meeting & \url{ {{.Attachments.kickoff_meeting }} } \iffalse attachment value="kickoff protocol meeting" \fi \\
 \hline
\end{tabu}
\end{center}


\vspace{2cm}
{{read .Markdown}} % chktex 26
{{template "en_sign.tex"}} % chktex 18
\FloatBarrier{}\mbox{}\vfill\pagebreak % make sure no floats (e.g. images) go past here.

{{template "terms.tex"}} % chktex 18
{{template "sign.tex"}} % chktex 18

\end{document}
