\documentclass[a4paper]{article}
\usepackage[T1]{fontenc}
\usepackage{fontspec}
\setmainfont{NotoSans-Regular}
\usepackage{lmodern}
\usepackage{textcomp}
\usepackage{tabu}
\usepackage{rotating}
\usepackage{multicol}
\usepackage{alltt}
\usepackage{longtable}[=v4.13]%
\usepackage{url}
\usepackage[
  margin=1.5cm,
]{geometry}

\special{papersize=210mm,297mm}
\setlength{\parindent}{0pt}
\tabulinesep=3pt
\pagestyle{plain}

\begin{document}

% First Page
\renewcommand\thesection{}
\renewcommand\thesubsection{}
\renewcommand\thesubsubsection{}

\begin{center}

  \topskip0pt
  \vspace*{\fill}
  \section{FORFAITINGA LĪGUMS}
  \subsection{(KAS LATVIJAS NORMATĪVO AKTU IZPRATNĒ IR CESIJAS LĪGUMS)}
  \vspace*{\fill}

  \vspace{8cm}

  \subsubsection{STARP}

  \vspace{1cm}

  \begin{tabu}{|r|X[2]|} \tabucline{}

    CEDENTS&{{.ESCo.Name}}\\\tabucline{}

    FORFAITINGA CESIONĀRS&{{index .Contract.Fields "client_name"}}\\\tabucline{}

  \end{tabu}

  \iffalse input forfaitingFields.front_page_free_text value="{{.Contract.Agreement.front_page_free_text}}" \fi {{.Contract.Agreement.front_page_free_text}}, \iffalse input forfaitingFields.front_page_date value="{{.Contract.Agreement.front_page_date}}" type="date" \fi {{.Contract.Agreement.front_page_date}}

\end{center}

\pagebreak

% Preample Page
\begin{center}
	\section{PREAMBULA}
\end{center}

\vspace{5mm}

Cedents ir noslēdzis energoefektivitātes pakalpojuma līgumu (šā Līguma 1.pielikumā pievienotajā formā),
kas noslēgts
{{index .Contract.Fields "date"}} starp {{index .Contract.Fields "client_name"}}
(kā \textbf{“Pasūtītāju”} jeb \textbf{“Cedenta klientu“}) un Cedentu (kā \textbf{“Izpildītāju”}),
saskaņā ar kuru Cedentam ir jānodrošina finansējums un jāuzņemas visi atbilstošie darbi, lai pilnībā,
pienācīgi un atbilstoši īstenotu energoefektivitātes paaugstināšanas pasākumus, par kuriem panākta
iepriekšēja vienošanās, kā atlīdzību saņemot regulārus maksājumus par panākto garantēto enerģijas
patēriņu (turpmāk – “Energoefektivitātes pakalpojuma līgums”), un Energoefektivitātes pakalpojuma
līgumā definētajiem jēdzieniem un frāzēm (ja vien konteksts nenosaka citādāk) ir tāda pati nozīme kā šajā
Līgumā;;\par

\vspace{5mm}

Cedenta atlīdzības saskaņā ar Energoefektivitātes pakalpojuma līgumu maksājama saskaņā ar atrunāto
mēneša maksājumu grafiku  {{date_diff .Project.ConstructionFrom .Project.ConstructionTo}}
gadu termiņā, kas sastāv no šādiem elementiem, kas detalizēti aprakstīti
Energoefektivitātes pakalpojuma līgumā, prot:\par

\vspace{5mm}

\begin{enumerate}
\item{\textbf{Maksa par enerģiju:} saskaņā ar šobrīd spēkā esošo Enerģijas tarifu par Enerģijas patēriņu un
sadalīta vienādos ikmēneša maksājumos, kam piemērojama norēķinu korekcija reizi gadā, ņemot
vērā: atsauces periodā dominējošos faktiskos laika apstākļus un garantētā enerģijas ietaupījuma
mērījumus un pārbaudi.;}

\item{\textbf{Maksa par renovāciju:} saistībā ar Ēkas renovācijas izmaksām maksājamā summa, kas indeksēta
atbilstoši EURIBOR likmei;;}

\item{\textbf{Ekspluatācijas un apkopes maksa:} par ekspluatācijas un apkopes izdevumiem pienākošais
maksājums, kam piemērojama ikgadēja indeksācija atbilstoši {{.Contract.Fields.country_adj}} patēriņa cenu indeksam
attiecīgajā gadā, kuru publicē {{.Contract.Fields.country_issuer}}.}

\item{\textbf{Jebkādi aprēķinātie nodokļi} (piemēram, PVN) par pakalpojumu veikšanu}

\item{\textbf{Mājsaimniecības karstais ūdens:} summa, kas maksājama saskaņā ar šobrīd spēkā esošo Enerģija
tarifu, kas aprēķināt par faktisko karstā ūdens patēriņu mājsaimniecībā.
turpmāk tekstā saīsināti saukti par
\textbf{"Debitoru parādiem"}}
\end{enumerate}

Cedents vēlas Forfaitinga cesionāram nodot savus no Energoefektivitātes pakalpojuma izrietošos
Debitoru parādus pilnā apmērā ar nosacījumu, ka apmierinoši ir izpildīti Energoefektivitātes pakalpojuma
līgumā noteiktie darbi un par cesiju ir rakstveidā paziņots Ēkas attiecīgajiem iedzīvotājiem.\par

\pagebreak

% Specific Terms and Conditions of this agreement
\renewcommand\thesection{}
\renewcommand\thesubsection{\arabic{subsection}}
\renewcommand\thesubsubsection{\thesubsection.\arabic{subsubsection}.}

\renewcommand*{\theenumi}{\thesubsection.\arabic{enumi}}
\renewcommand*{\theenumii}{\theenumi.\arabic{enumii}}

\section{SPECIĀLIE NOTEIKUMI}
\vspace{5mm}

\subsection{LĪGUMA PUSES}

\textbf{“Cedents“ (saukts arī – “ESCO”)}

\vspace{2mm}

\begin{tabu}{|X|X[2]|} \tabucline{}
  Nosaukums                  &{{.ESCo.Name}}                                          \\\tabucline{}
  Reģistrācijas numurs   &{{.ESCo.VAT}}                                           \\\tabucline{}
  Adrese               &{{.ESCo.Address}}                                       \\\tabucline{}
  Tiesiskais pārstāvis  &{{.Contract.Fields.contractor_representative_name}}     \\\tabucline{}
\end{tabu}

\vspace{5mm}

\textbf{“Forfaitinga cesionārs“ (saukts arī – “Fonds” vai “LABEEF”)}

\vspace{2mm}

\begin{tabu}{|X|X[2]|} \tabucline{}
  Nosaukums                   &{{index .Contract.Fields "client_name"}}                 \\\tabucline{}
  Reģistrācijas numurs    &{{index .Contract.Fields "client_id"}}                   \\\tabucline{}
  Adrese                &{{index .Contract.Fields "client_address"}}              \\\tabucline{}
  Tiesiskais pārstāvis  &{{index .Contract.Fields "client_representative_name"}}  \\\tabucline{}
\end{tabu}

\vspace{5mm}

\subsection{LĪGUMA APJOMS}

\vspace{2mm}

\begin{enumerate}
\item{Ievērojot šā Līguma noteikumus un nosacījumus, Cedents cedē un nodod Forfaitinga
cesionāram, turpretī Forfaitinga cesionārs pērk un saņem uz šā Līguma pamata visu
Debitoru parādu kopējo summu, kas definēti un Cedenta klientam maksājami saskaņā ar
Energoefektivitātes pakalpojuma līgumu, tai skaitā apkalpošanas maksas, kas pastāv
starp Cedenta klientu un Cedentu saskaņā ar Energoefektivitātes pakalpojuma līgumu un
kas atbilst turpmāk norādītajiem kritērijiem:}

  \begin{enumerate}
  \item{Debitoru parādi ir radušies, īstenojot Energoefektivitātes paaugstināšanas
pasākumus, kurus Cedents ir sniedzis Cedenta klientam;}

  \item{Debitoru parādi pastāv ar fiksētu vai aprēķināmu vērtību, tiem ir piemērojama
līgumā noteiktā indeksācija, tie ir tiesiski realizējamā un piedzenamā formā,
atbilstoši Latvijas tiesību aktiem, tiem ir piemērojama Latvijas tiesu jurisdikcija, un
tie nav noprotestējami, tiem nav piemērojama regresa kārtība, pārsūdzības
tiesības, pretprasības, ieskaits, ķīla vai citas trešo personu tiesības;}

  \item{Energoefektivitātes pakalpojuma līgums un visi Debitoru parādus un to izpildi
apliecinošie dokumenti nav pretrunā ar jebkādām piemērojamajām obligātajām
tiesību normām, un visi šādi dokumenti ir iesniegti Forfaitinga cesionāram; un}

  \item{Debitoru parādu pārdošana, cesija un/vai ieķīlāšana nav aizliegtas (vienalga vai
saskaņā ar Energoefektivitātes pakalpojuma līguma, jebkura cita līguma vai
jebkādu piemērojamo tiesību aktu noteikumiem);
līdz ar:}

  \item{jebkādu aizturējumu, ķīlu, citu nodrošinājumu vai galvojumu cesiju, kas nodibināti
nolūkā nodrošināt ar šo Līgumu cedēto un nodoto Debitoru parādu samaksu; un}

  \item{Cedenta Ēkā uzstādītā Aprīkojuma ķīlu maksimālajā iespējamajā apmērā
(atbilstoši šā līguma vispārīgie noteikumi un nosacījumi tālāk aprakstītajam), lai
nodrošinātu šajā Līgumā ietvertās Forfaitinga cesionāra tiesības.}
  \end{enumerate}

\item{Šis Līgums tiek noslēgts ar atrunu, ka cesijai, nodošanai un atsavināšanai ir piemērojams
nosacījums, ka tās ir realizējamas tikai tad, ja izpildās turpmāk norādītie īpašie nosacījumi:}

  \begin{enumerate}
  \item{Projekts ir izstrādāts saskaņā ar Forfaitinga cesionāra Energoefektivitātes
pakalpojuma līguma Vispārīgajiem noteikumiem un nosacījumiem un pilnībā ievērojot {{index .Contract.Fields "client_name"}}
Finanšu un tehniskos noteikumus un vadlīnijas attiecībā uz energoefektivitātes
pasākumiem;}

  \item{ir pieejama vismaz viena kalendārā gada nepārtraukta komunālo maksājumu rēķinu
apmaksas vēsture saistībā ar attiecīgo Projektu, kas apliecina Energoefektivitātes pakalpojuma
līgumā Garantētā enerģijas patēriņa sasniegšanu atbilstoši Energoefektivitātes pakalpojuma
līgumā noteiktajai mērījumu un pārbaudes kārtībai;}

  \item{visas Energoefektivitātes paaugstināšanas pasākumu īstenošanas izmaksas ir pilnībā
apmaksātas attiecīgajiem apakšuzņēmējiem (ja tādi būtu);}

  \item{ir ticis pienācīgi nosūtīts un tiesīgi atzīts paziņojums par Debitoru parādu cesiju šā Līguma
2.pielikumā noteiktajā formātā, ar kuru Cedenta klients apstiprina Forfaitinga cesionāram šāda
paziņojuma saņemšanu;}

  \item{iegādāta tiesīgi noslēgta un spēkā esoša apdrošināšanas polise par visu Ēkas vērtību pēc
tās renovācijas ar minimālo apdrošināšanas segumu pret ugunsgrēku, zemestrīci, plūdiem, ūdens
nodarītiem bojājumiem, jebkādām citām dabas katastrofām, kas ietekmē Ēku, strukturāliem
bojājumiem, ko izraisījusi nosēšanās un krituši koki, ko apliecina attiecīgās polises vai cita
pārliecinoša dokumenta, kas apstiprina apdrošināšanas prēmijas samaksu, apliecinātas kopijas
iesniegšana. Apdrošināšanas polisei vajadzētu būt spēkā ne mazāk kā 3 gadus pēc tās
noformēšanas un noslēgtai atbilstoši noteikumiem un nosacījumiem, kas ietver attiecīgās nozares
standarta noteikumus, un noslēgtai ar apdrošinātāju, kuram saskaņā ar Latvijā piemērojamiem
atzītiem reitingiem ir piešķirts vismaz A+ reitings;}

  \item{Apmierinoši Forfaitinga cesionāra nolūkā pārbadīt Ēkā veikto renovācijas darbu un
īstenoto Energoefektivitātes pasākumu kvalitāti veiktās padziļinātās izpētes procesa, kas veikts
ne ātrāk kā pēc pirmās apkures sezonas beigām (šīs padziļinātās izpētes datuma nosacījums ir 20
(divdesmit) darba dienas pēc datuma, kad Cedents ir iesniedzis visus Forfaitinga cesionāra
padziļinātās izpētes nolūkā pieprasītos dokumentus), rezultāti. Puses atzīst, ka šis padziļinātās
izpētes nosacījums ir iekļauts vienīgi Forfaitinga cesionāra labā, lai Forfaitinga cesionāram ļautu
izpētīt Cedenta veikto darbu kvalitāti un saskaņā ar Energoefektivitātes pakalpojuma līgumu
panākto faktisko enerģijas patēriņa apjomu; un}

  \item{ir tiesīgi noslēgts un spēkā esošs Pārvaldības līgums (atbilstoši 8.punktā tālāk sniegtajai
definīcijai) un Trīspusējais līgums (atbilstoši 8.punktā tālāk sniegtajai definīcijai).}
  \end{enumerate}

\item{Līdz ar Nosacījumu izpildes protokola parakstīšanu Debitoru parādu nodošana tiek
pabeigta. Šis Līgums ir tiesiski saistošs, sākot no tā parakstīšanas brīža, bet no tā izrietošās
saistības, ja vien nav skaidri norādīts pretējais, uzskatāmas par atliktām līdz
iepriekšminētā Protokola izdošanai, kas apliecina no (2.21) līdz (2.2.7) punktam iepriekš
norādīto nosacījumu izpildi.}

\end{enumerate}

\subsection{AR ŠO LĪGUMU NODOTIE DEBITORU PARĀDI}

\begin{enumerate}
\item{Ar šo Līgumu Forfaitinga cesionāram nodoto Debitoru parādu kopējā summa ir visi
debitoru parādi, kuru maksājumus Cedentam ir tiesības saņemt no Cedenta klienta, uz
tiesīgi noslēgta Energoefektivitātes pakalpojuma līguma pamata (šādi debitoru parādi ir
nododami kā 1.2.punktā iepriekš definētie Debitoru parādi), un neierobežojot Forfaitinga
cesionāra tiesības pārjaunot jebkādas no Energoefektivitātes pakalpojuma līgumā
noteiktajām Cedenta tiesībām. Debitoru parādi, kas cedējami Fondam, rodas no šādām
maksājumu saistībām: \par

    \begin{tabu}{XX[2]}
      Cedenta klients:  & {{.BuildingOwner.Name}} \\
      juridiskā adrese:      & {{.BuildingOwner.Address}}
    \end{tabu}}

\item{Debitoru parādu nodošana saskaņā ar šo Līgumu veido tiesību uz visu to prasību pret Cedenta
klientu cesiju (Latvijas Civillikuma 1793.panta devītās daļas 1.punkta un saistīto normu izpratnē),
uz kuriem Cedents ir tiesīgs. Lai novērstu šaubas, visu esošo Cedenta prasību pret Cedenta klientu
cesija piešķir Forfaitinga cesionāram tiesības vienpersoniski vai ar trešo personu starpniecību
uzņemties visas atbilstošās faktiskās un juridiskās darbības visu šādu prasību pret katru no
Dzīvokļu īpašniekiem atsevišķi vai kopīgi kā Cedenta klientu realizācijai. Neierobežojot
iepriekšminēto, Cedents ar šo do savu piekrišanu un bez ierunām piekrīt pārjaunot Forfaitinga
cesionāram vai jebkurai no tā ieceltajām personām jebkuras no savām saistībām saskaņā ar
Energoefektivitātes pakalpojuma līgumu ne vēlāk kā 5 Darba dienu laikā no attiecīga Forfaitinga
cesionāra pieprasījuma saņemšanas dienas.}
\end{enumerate}

\subsection{DEBITORU PARĀDU PIRKUMA MAKSA}

\begin{enumerate}
\item{Debitoru parādu pirkuma maksa ir EUR
    \iffalse input forfaitingFields.receivable_price value="{{.Contract.Agreement.receivable_price}}" type="number" \fi {{.Contract.Agreement.receivable_price}} (\iffalse input forfaitingFields.receivable_price_words value="{{.Contract.Agreement.receivable_price_words}}" \fi {{.Contract.Agreement.receivable_price_words}}) (the \textbf{"Price"}) apmērā
    \iffalse input forfaitingFields.receivable_price_percent value="{{.Contract.Agreement.receivable_price_percent}}" type="number" \fi {{.Contract.Agreement.receivable_price_percent}} (\iffalse input forfaitingFields.receivable_price_percent_words value="{{.Contract.Agreement.receivable_price_percent_words}}" \fi {{.Contract.Agreement.receivable_price_percent_words}}) no pilnas vērtības debitoru parādiem, kas saistītas ar renovācijas maksa par EPC
līgumu.}

\item{Pirkuma maksas samaksa tiek veikta saskaņā ar 5.1.punktu pēc tā, kas iestājas vēlāk}

    \begin{enumerate}
    \item{Cedenta klients saņem paziņojumu par Cedenta veikto visu Debitoru parādu kopējās
summas nodošanu Forfaitinga cesionāram, un }

    \item{tiek saņemts 2.3.punktā iepriekš atrunātais Nosacījumu izpildes protokols.}
    \end{enumerate}
\end{enumerate}

\subsection{ATLĪDZĪBA PAR IZPILDI UN CITI VEICAMIE MAKSĀJUMI}

\begin{enumerate}
\item{Cedentu būs tiesības uz maksājumu no forfaiting pārņēmējs, kas atbilst:}

  \begin{enumerate}
  \item{ \iffalse input forfaitingFields.forfaiting_assignee_payment value="{{.Contract.Agreement.forfaiting_assignee_payment}}" type="number" \fi {{.Contract.Agreement.forfaiting_assignee_payment}} (\iffalse input forfaitingFields.forfaiting_assignee_payment_words value="{{.Contract.Agreement.forfaiting_assignee_payment_words}}" \fi {{.Contract.Agreement.forfaiting_assignee_payment_words}})
no debitoru saistīta renovācijai maksa par EPC vienošanās}

  \item{uz Ekspluatācijas un apkopes maksu, kas norādīta Energoefektivitātes pakalpojuma
līgumā un tiek maksāta vienu reizi ceturksnī, kamēr tiek sniegti pakalpojumi}
  \end{enumerate}

\item{Cedentu būs papildus ir tiesības uz atlīdzību darbības attiecībā uz katru gadu, kurā sasniegts
enerģijas taupīšanas atbilstoši EPC vienošanās, pārsniedz summu, garantēto enerģijas
ietaupījumu šim gadam atbilstoši EPC nolīguma (pēc mērījumu un pārbaudes saskaņā ar LABEEF
Finanšu un tehniskie noteikumi un vadlīnijas energoefektivitātes pasākumi). Šāds sniegums
maksu aprēķina katru gadu saskaņā ar šādu formulu: \par

    \begin{center}
      \[Pārsniedzošie rezultāti maksa = \iffalse input forfaitingFields.performance_fee value="{{.Contract.Agreement.performance_fee}}" type="number" \fi {{.Contract.Agreement.performance_fee}}\% P_G\]
    \end{center}

    kur: \par
    $P_G$: garantiju pārsniedzošie rezultāti Norēķinu periodā, EUR (bez PVN), kas aprēķināts
saskaņā ar EPC līgumu (3.pielikums šā nolīguma) 5.pielikums}

\item{Maksa sniegums aprēķināta iepriekšminētajā veidā kļūst jāsamaksā ar forfaiting pārņēmējam ar
Maksājumu par pirmo ceturksni periodā pēc termiņa beigām gadu, kurā tika panākta papildu
enerģijas ietaupījumu.}

\item{Garantijas neizpildes gadījumā, kur bilance starp Enerģijas ietaupījumu Norēķinu periodā un
Garantēto enerģijas ietaupījumu ir negatīva, Cedentam ir atsevišķi un bez ierunām pienākums
maksāt Forfaitinga cesionāram Energoefektivitātes pakalpojuma līgumā noteikto Kompensāciju.
Forfaitinga cesionārs patur tiesības atskaitīt Kompensāciju no maksājamās Ekspluatācijas un
apkopes maksas un Maksas par renovāciju (5.1.punkts).}

\item{Iepriekšminētie aprēķini veicami reizi gadā līdz ar Atlīdzības ikgadējo korekciju saskaņā ar
Energoefektivitātes pakalpojuma līguma 5. pielikumu. Iegūtā starpība starp Debitoru parādiem,
balstoties uz Garantēto enerģijas patēriņu, un Debitoru parādiem, kuru pamatā ir faktiskais gada
enerģijas maksājums, ir iekasējama un jāmaksā Forfaitinga cesionāram 30 dienu laikā pēc
aprēķinu veikšanas, kuros konstatēta novirze no Garantētā enerģijas patēriņa. Cedentam ir
jānokārto savs parāds ar pārskaitījumu uz Forfaitinga cesionāra norādīto bankas kontu
iepriekšminētajā termiņā. Iepriekšminētais maksājuma kavējums piešķir Forfaitinga cesionāram
tiesības uz kavējuma naudu \iffalse input forfaitingFields.late_payment_fee value="{{.Contract.Agreement.late_payment_fee}}" type="number" \fi {{.Contract.Agreement.late_payment_fee}} \%
    (\iffalse input forfaitingFields.late_payment_fee_words value="{{.Contract.Agreement.late_payment_fee_words}}" \fi {{.Contract.Agreement.late_payment_fee_words}}) apmērā no nesamaksātās
summas par katru kavējuma dienu, bet kopumā nepārsniedzot \iffalse input forfaitingFields.outstanding_amount value="{{.Contract.Agreement.outstanding_amount}}" type="number" \fi {{.Contract.Agreement.outstanding_amount}}
    \%(\iffalse input forfaitingFields.outstanding_amount_words value="{{.Contract.Agreement.outstanding_amount_words}}" \fi {{.Contract.Agreement.outstanding_amount_words}}) no nesamaksātās summas.}

\item{Neierobežojot iepriekšminēto, Forfaitinga cesionārs ir tiesīgs veikt savu prasījumu par
iepriekšminētajām summām ieskaitu no jebkuriem izpildes maksas maksājumiem, kas pienākas
saskaņā ar 6.1.punktu iepriekš, un/vai samazināt jebkura no ceturkšņa maksājumiem summu kā
aizstāvību, ieturējumu, iebildumu, pretprasību vai uz jebkuru citu savu no šā Līguma izrietošo
tiesību pamata.}
\end{enumerate}

\subsection{SAMAKSAS TERMIŅI}

\begin{enumerate}
\item{Cedentam ir jāizraksta rēķins par Maksu [katru tās maksājumu] 10 dienu laikā pēc 4. punktā
iepriekš minētā Nosacījumu izpildes protokola parakstīšanas starp Pusēm datuma saskaņā ar šajā
Līgumā iepriekš minētajiem noteikumiem. Maksājumi tiks veikti šādas iemaksas (iemaksu) veidā:{[}s{]}: \par

    {{index .Contract.Agreement "payment_first_amount"}} EUR – – kuru termiņš ir un kas maksājami ne vēlāk kā {{index .Contract.Agreement "payment_first_date"}}; \par
    {{index .Contract.Agreement "payment_second_amount"}} EUR – – kuru termiņš ir un kas maksājami ne vēlāk kā {{index .Contract.Agreement "payment_second_date"}}; and \par
    {{index .Contract.Agreement "payment_third_amount"}} EUR – – kuru termiņš ir un kas maksājami ne vēlāk kā {{index .Contract.Agreement "payment_third_date"}}.\par}

\item{Cedentu izdod faktūru par ceturkšņiem attiecībā uz maksājumiem, kas atbilst Uzkrāto nodevām
saskaņā raksts 5.1.}

\item{Cedentu izsniedz rēķinu par katru gadu pēc mērīšanai un pārbaudei, kā norādīts EPC līgumā
saskaņā 5.2 rakstu.}

\item{Ar Maksu vai izpildi saistītie maksājumi (saskaņā ar 4.3.1.punktu) vai citi atlīdzības maksājumi
(saskaņā ar 4.4.1.punktu) ir veicami uz turpmāk norādīto Cedenta bankas kontu (ja vien Cedents
10.lapa no 24
neizmaina savus norādījumus attiecībā uz maksājumiem, pirms Forfaitinga cesionārs uzsāk
attiecīgo maksājumu, nosūtot Cesionāram rakstisku paziņojumu, un tādā gadījumā maksājami ir
veicami uz šādā paziņojumā norādīto bankas kontu): \par

    \begin{tabu}{XX[2]}
      Saņēmēja vārds/nosaukums: 	& 	{{.FABankAcc.BeneficiaryName}} 	\\
      Bankas nosaukums un adrese: 	& 	{{.FABankAcc.BankNameAddress}} 	\\
      IBAN:			& 	{{.FABankAcc.IBAN}} 		\\
      SWIFT:			& 	{{.FABankAcc.SWIFT}} 		\\
    \end{tabu}}

\item{The Assignor shall pay any due invoice within 30 working days.}

\item{The Forfaiting Assignee shall issue an invoice on annual basis
    after Measurement and Verification as indicated in the Energy
    Performance Contract pursuant Art. 5.5.}

\item{The payments in respect of the breach of Guaranteed Energy
    Consumption shall be made to the following bank account of the
    Forfaiting Assignee (unless the Forfaiting Assignee changes its
    payment instructions prior to the Assignor initiating the relevant
    payment by giving a written notice, in which case payments shall
    be made to the bank account indicated in such notice):

    \begin{tabu}{XX[2]}
      Beneficiary Name: 	& 	{{.FABankAcc.BeneficiaryName}} 	\\
      Bank name and address: 	& 	{{.FABankAcc.BankNameAddress}} 	\\
      IBAN:			& 	{{.FABankAcc.IBAN}} 		\\
      SWIFT:			& 	{{.FABankAcc.SWIFT}} 		\\
    \end{tabu}}

\end{enumerate}

{{template "sign.tex"}}

% General Terms and Conditions
\section{ŠĀ LĪGUMA VISPĀRĪGIE NOTEIKUMI UN NOSACĪJUMI}

\begin{multicols}{2}

  \subsection{KAS ESCO IR JĀIESNIEDZ LABEEF – NEPIECIEŠAMIE
DOKUMENTI UN CITAS GARANTIJAS UN NOSACĪJUMI}

  \begin{enumerate}
  \item{Parakstot šo Līgumu, Cedents nodod Energoefektivitātes pakalpojuma
līgumu un visus dokumentus, kas apliecina nodotos Debitoru padomus
un ne vēlāk kā 5 dienu laikā pēc attiecīga Forfaitinga cesionāra
pieprasījuma saņemšanas visu informāciju, kas nepieciešama cedēto
Debitoru parādu iekasēšanai no Cedenta klienta un/vai no katra
Dzīvokļa īpašnieka, un jāiesniedz visi dokumenti un informācija, kas
nepieciešama visu no Energoefektivitātes pakalpojuma līguma izrietošo
prasījumu realizācijai. Cedentam ne retāk kā reizi ceturksnī ir
jāapstiprina, ka Forfaitinga cesionāram iesniegtie dokumenti un
informācija attiecībā uz nodotajiem Debitoru parādiem aptver visus
dokumentus un informāciju attiecībā uz nodotajiem Debitoru parādiem
vai, ja tā nav, jāiesniedz visi papildu dokumenti un informācija, lai
novērstu jebkādu iztrūkumu. Cedentam ir pienākums nekavējoties
rakstveidā informēt Forfaitinga cesionāru par jebkādiem apstākļiem,
kas liecinātu par Cedenta klienta kredītreitinga vai kredītspējas
pasliktināšanos.}

  \item{Cedentam ir atsevišķi jebkurā laikā jāsniedz Forfaitinga cesionāram
informācija par Cedenta finanšu uzskaiti, ciktāl tā ir saistīta ar
nodotajiem Debitoru parādiem, kā arī peļņas un zaudējumu pārskats
sešu mēnešu laikā pēc bilances datuma.}

  \item{Cedentam ir pienākums nekavējoties rakstveidā informēt Forfaitinga
cesionāru par jebkādu strīdu, pārsūdzību, garantijas prasību, garantijas
tiesību anulēšanu, realizāciju no Cedenta klienta puses vai tā saistību
neizpildi, kā arī nekavējoties jāpakļaujas jebkādam Cedenta klienta
likumīgam prasījumam pret Cedentu nomainīt, uzlabot vai uzstādīt
jebkādu trūkstošu iekārtu vai pakalpojumu vai novērst jebkādu
neizpildi. Neierobežojot iepriekšminētā vispārīgumu, iepriekšminētais
ietver, bez ierobežojuma jebkādas garantijas prasības vai jebkādas cita
rakstura prasības, kas izriet no Energoefektivitātes pakalpojuma
līguma, neaprobežojoties ar jebkādām prasībām, kas izriet no Cedenta
Ēkā uzstādīto Energoefektivitātes paaugstināšanas pasākumu
trūkumiem vai bojājumiem.}

  \item{Cedentam ir atsevišķs pienākums pēc pieprasījuma nodrošināt
apmierinošus pierādījumus jebkādu dokumentu īstumam, kas apliecina
nodotos Debitoru parādus un liecina, ka šādi dokumenti rada tiesiskas,
spēkā esošas, saistošas un realizējamas saistības bez jebkāda regresa
pret Forfaitinga cesionāru. Forfaitinga cesionāram ir 10 Darba dienu
laikā pēc to saņemšanas jāizpēta Cedenta iesniegtie dokumenti, un tas
var neņemt vērā vai atdot Cedentam jebkuru iesniegto dokumentu, kas
neatbilst spēkā esošas, saistošas un realizējamas saistības
apliecināšanas kritērijiem, un jāpieprasa, lai pienācīgi un nekavējoties
tiktu iesniegti šīm prasībām atbilstoši dokumenti.}

  \item{Cedents garantē Forfaitinga cesionāram, ka:}

    \begin{enumerate}

    \item{ tas ir pienācīgi dibināts, tiesīgi pastāvošs un ar labu
reputāciju saskaņā ar to jurisdikciju, kurā tas ir dibināts vai reģistrēts,
un tam ir visas tiesības un pilnvaras parakstīt, nodot un izpildīt savas
šajā Līgumā noteiktās saistības;}

    \item{tam parakstot, nododot un izpildot šo Līgumu, nerodas
jebkāda (i) tā dibināšanas dokumentu, (ii) tam piemērojamā jebkāda
likumdošanas akta, tiesību aktu vai jebkuras valsts iestādes noteikumu;
(iii) jebkāda tam piemērojamā sprieduma vai rīkojuma par pagaidu
noregulējumu noteikuma pārkāpums vai neatbilstība;}

    \item{šā Līguma parakstīšana, nodošana un izpilde tā vārdā ir
pienācīgi un tiesīgi pilnvarota, un šis Līgums pēc tā parakstīšanas kļūs
par tā tiesiskām, spēkā esošām un saistošām saistībām, kas
realizējamas pret to saskaņā ar attiecīgajiem Līguma noteikumiem;
izņemot šajā Līgumā noteikto, nekāds cits paziņojums, izņemot
paziņojumu Cedenta klientam par Debitoru parādu nodošanu, jebkādai
kompetentajai valsts iestādei vai jebkurai citai personai, tās piekrišana
vai saskaņojums, vai jebkura cita darbība no tās puses nav un nebūs
nepieciešama, lai parakstītu, nodotu un izpildītu šajā Līgumā noteiktos
tā pienākumus;}

    \item{ tam ir pienācīgas tiesības cedēt nodotos Debitoru parādus;}

    \item{pilnībā vai daļēji nodotie Debitoru parādi ir tiesiskas, spēkā
esošas un saistošas Cedenta klienta saistības, un tie ir piedzenami no
Cedenta klienta saskaņā ar attiecīgajiem noteikumiem;}

    \item{nodotie Debitoru parādi nav tikuši citādā veidā cedēti,
pārdoti, uz tiem nav tikusi nodibināta hipotēka, tie nav ieķīlāti vai
jebkādā citādā veidā apgrūtināti vai apgrūtināti ar jebkādām jebkuras
trešās personas tiesībām un ka nevienam no nodotajiem Debitoru
parādiem nav un nav sagaidāms ieskaits pret jebkādiem maksājumiem,
kas Cedentam varētu būt maksājami Cedenta klientam; un}

    \item{nekādi procesi nav (i) gaidāmi pret Cedentu vai, (ii) cik
zināms Cedentam, nedraud Cedentam jebkurā kompetentā valsts
iestādē, tiesā vai šķīrējtiesā, kas atsevišķi vai kopumā varētu būtiski un
nelabvēlīgi ietekmē (i) jebkādu saskaņā ar šo Līgumu veikto vai veicamo
Cedenta darbību vai (ii) nodotos Debitoru parādus.}
    \end{enumerate}
  \end{enumerate}

  \subsection{KAS FONDAM PIENĀKAS NO ESCO – CEDENTA
TIESĪBAS UN PIENĀKUMI}

  \begin{enumerate}
  \item{Cedents uzņemas atbildību pret Forfaitinga cesionāru, ja paziņojuma
par Debitoru parādu cesiju nosūtīšanas dienā vai pirms tam tas nav
atklājis jebkādu maksājumu prasījumu vai citu saistību pastāvēšanu, ka
var ietekmēt saskaņā ar šo Līgumu nododamo Debitoru parādu summu
vai spēkā esamību. Iepriekšminētais neierobežo to, ka Cedents nav
atbildīgs, nedz atsevišķi, nedz kopīgi ar Cedenta klientu, par jebkādu
Debitoru parādu un saskaņā ar šo Līgumu Forfaitinga cesionāram
nodoto saistībā ar Debitoru parādiem nodibināto nodrošinājumu
savlaicīgu samaksu vai piedziņu. Neņemot vērā neko pretēju, Cedents vēl aizvien ir atbildīgs par nododamo Debitoru parādu un attiecībā uz
tiem nodibināto nodrošinājumu patiesumu un realizējamību.}


  \item{Cedentam kopā ar Forfaitinga cesionāru ir kopīgi jāinformē Cedenta
klients vienā paziņojumā par Debitoru parādu nodošanu uz šā Līguma
pamata, un šādam paziņojumam ir jābūt 1.pielikumā pievienotajā
formātā.}

  \item{Cedents garantē Energoefektivitātes pakalpojuma līgumā norādītā
Garantētā enerģijas ietaupījuma apjomu un apņemas nodrošināt tā
līmeņa uzturēšanu visā Energoefektivitātes pakalpojuma līguma spēkā
esamības laikā, vienlaikus apņemoties veikt visus tādus papildu darbus,
kuri nerada būtiskus papildu izdevumus.}

  \item{Cedentam ir jānodrošina, lai visi komunālie pakalpojumi no Ēkas
atsevišķiem dzīvokļiem un uz tiem, no kuriem izriet Debitoru parādi,
nebūtu atslēgti vai pārtraukti jebkurā brīdī bez Cedenta klienta
iepriekšējas piekrišanas (tai skaitā, kad šāda piekrišana ir izteikta
mutiski, rakstiski vai nodota Pārvaldniekam pa e-pastu) un tiek
nekavējoties atjaunoti gadījumā, ja tie īslaicīgi nedarbojas jebkāda
veida regulāru pārbaužu vai trūkumu vai bojājumu novēršanas vai
nomaiņas dēļ. Šī garantija neattiecas uz gadījumiem, kad Cedents
neietekmē šādus pārtraukumus un/vai tie ir radušies tieši piegādātāju,
Forfaitinga cesionāra vai jebkādu trešo personu darbības vai
bezdarbības dēļ.}

  \item{Cedents piekrīt saglabāt savu atbildību un apņemas pildīt visus savus
pienākumus un saistības saskaņā ar Energoefektivitātes pakalpojuma
līgumu, kas ir tieši saistīti ar Cedenta kā Izpildītāja darbību vai
pienākumiem Energoefektivitātes pakalpojuma līguma ietvaros.
Iepriekšminētais neierobežo Cedenta saistības jebkāda īstenoto
Energoefektivitātes paaugstināšanas pasākumu vai renovācijas darbu
trūkumu vai bojājuma gadījumā, kas konstatēti var radušies šā Līguma
darbības laikā uzņemties remontu un/vai nomaiņu (atkarībā no tā, kas
no šiem abiem tiek noteikts kā kvalitātes un cenas ziņā efektīvākais
līdzeklis) par saviem līdzekļiem saprātīgā termiņā atbilstoši
piemērojamajiem nozares standartiem. Gadījumā, ja Cedents saprātīgā
laikā, bet ne ilgāk kā 20 darba dienas neievēro iepriekšminēto,
Forfaitinga cesionāram rodas tiesības izpildīt darbus par Cedenta
līdzekļiem un saņemt atlīdzību 20 dienu laikā pēc attiecīga rēķina
nosūtīšanas Cedentam (kuru var nosūtīt pa e-pastu).}

  \item{Cedentam nekavējoties pēc uzzināšanas ir jāziņo Forfaitinga
cesionāram par jebkādu kaitējumu vai izmaiņām, kas veiktas bez tā
ziņas vai atļaujas jebkurās Cedenta uzstādītajās iekārtās (ieskaitot, bet
neaprobežojoties ar skaitītājiem un piederumiem), vai jebkādiem
apstākļiem, kas ir un/vai varētu negatīvi ietekmēt Cedenta šajā periodā
12.lapa no 24 sagaidīto Enerģijas patēriņa samazināšanos (atsevišķi vai kopā jebkurā
periodā) par diviem (2) vai vairāk procentiem no Enerģijas patēriņa. Ja
Cedents neievēro iepriekšminēto, Cedents kļūst atbildīgs par visiem
zaudējumiem un kaitējumu, kas rodas tieši no bojājuma vai izmaiņām,
ja vien netiek konstatēts, ka ātrāks paziņojums Forfaitinga cesionāram
nebūtu tam devis pietiekami laika, lai mazinātu Forfaitinga cesionāram
radušos kaitējumu un zaudējumus.}

  \item{Cedents ar šo piekrīt, ka saskaņā ar šo Līgumu Forfaitinga cesionārs
saņem pienācīgu pilnvarojumu attiecībā uz, tiek pienācīgi iecelts un
tiesīgs ievākt, apstrādāt, uzglabāt, izmantot un nodot saskaņā ar
Energoefektivitātes pakalpojuma līgumu Cedenta klienta sniegtos
personas datus un nodot tos jebkurai trešajai personai, kurai var tikt
nodotas no šā Līguma izrietošās Forfaitinga cesionāra tiesības vai
pienākumi, tai skaitā, bet ne tikai jebkurai personai, kas ir atbildīga par
īstenoto Energoefektivitātes paaugstināšanas pasākumu veiktspējas
uzraudzībai domātas tiešsaistes platformas izstrādi, realizāciju,
ekspluatāciju un apkopi. Citiem mērķiem Forfaitinga cesionārs datus
var izmantot vienīgi pēc Cedenta klienta piekrišanas tam saņemšanas.
Lai novērstu šaubas un, neņemot vērā iepriekšminēto, jebkuri no
personas datiem vai ar Ēku vai Aprīkojumu saistītie dati paliek Cedenta
klienta īpašumā, neierobežojot Forfaitinga cesionāra tiesības ievākt,
apstrādāt, uzglabāt, izmantot un nodot tos, un, izbeidzot šo Līgumu, tie
ir pilnībā jāatdod Cedenta klientam (vai pēc Cedenta klienta izvēlas
neatgriezeniski jāiznīcina, šo datu glabātājam izdodot apliecinājumu
par šādu iznīcināšanu).}

  \item{Forfaitinga cesionārs atzīst un piekrīt, ka Cedents vai jebkura trešā
persona, kurai Cedents ir nodevis šajā Līgumā ietvertās tiesības un/vai
pienākumus, izmanto jebkādus anonīmus datus un informāciju saistībā
ar enerģijas patēriņu Ēkā vienalga, vai to ir sniedzis Cedenta klients vai
ieguvis Cedents, nolūkā atzīmēt un apkopot valsts, reģionāla vai ES
mēroga datubāzi vai ar mērķi Cedentam izmantot kā atsauci vai
jebkādam iekšējam mērķim, kam ir piekritis Cedenta klients.}

  \item{Ar šo cedentu piekrīt, ka saskaņā ar šo nolīguma forfaiting pārņēmējs
saņem pienācīgu pilnvarojumu attiecībā uz, ir pienācīgi piešķirtas un
tiesīga vākt, apstrādāt, uzglabāt, izmantot un nodot personas datus, ko
cedenta Klienta piešķirtas saskaņā ar EPC nolīgumu, un pārskaitīt
pēdējais jebkurai trešajai personai, kurai forfaiting pēcteča tiesības vai
saistības, kas izriet no šā nolīguma, var tikt piešķirts, tostarp bez
ierobežojumiem jebkuru personu, kas atbild par attīstību, ieviešanu,
darbību un uzturēšanu tiešsaistes platformu sekotu sniegumu īstenoto
renovācija darbojas. To datu izmantošana, ko forfaiting pēctecim
citiem mērķiem jāveic tikai saņemot cedenta Klienta piekrišanu šajā
sakarā. Lai izvairītos no šaubām un neraugoties uz iepriekš minēto,
kādu no personas datu vai datu, kas saistīti ar ēku vai istabai paliek
īpašumā cedenta Klientam neskarot labi no forfaiting cesionāru savākt,
apstrādāt, uzglabāt, izmantošanu un nodošanu to un atdod pēdējo (vai
pēc izvēles cedenta Klienta, pastāvīgi iznīcināta ar sertifikātu šādas
iznīcināšanas turētājs šādu datu izsniegtā) pilnā apmērā pēc šī līguma
izbeigšanas.}

  \item{Forfaiting pārņēmējs atzīst un piekrīt izmantošanas, ko cedents vai
jebkurai citai trešajai pusei tā piešķirta ar tiesībām un / vai saistības
saskaņā ar šo nolīgumu, par jebkādiem anonīmiem datiem un
informāciju, kas attiecas uz enerģijas patēriņa ēka, vai ar nosacījumu,
cedenta Klients vai iegūti, atsavinātājam, lai ar salīdzinošās
novērtēšanas un apkopošanu no valsts, reģionālā vai ES mēroga datu
bāzē mērķiem vai par lietošanas mērķiem, ko cedents kā atsauce vai
jebkuram iekšējam mērķim vienojās ar cedenta Klientam.}
  \end{enumerate}

  \subsection{KAS CEDENTAM (ESCO) PIENĀKAS NO FORFAITINGA
CESIONĀRA (FONDA) – FORFAITINGA CESIONĀRA
TIESĪBAS UN PIENĀKUMI}

  \begin{enumerate}
  \item{Pēc attiecīgo dokumentu saņemšanas no Cedenta Forfaitinga
cesionāram ir jāizpēta šie dokumenti, lai noteiktu, vai tie ir
nepieciešamie dokumenti, un nekavējoties, bet jebkurā gadījumā ne
vēlāk kā 10 (desmit) darba dienu laikā jāatdod Cedentam jebkāds
dokuments, kas acīmredzami neatbilst tā īstuma nosacījumiem un/vai
acīmredzami nerada spēkā esošu, tiesisku, saistošu vai izpildāmu
saistību. Ja, balstoties uz iepriekšminētajiem faktoriem, Forfaitinga
cesionārs konstatē, ka iesniegtie dokumenti nav apmierinoši vai ka
nepieciešami papildu dokumenti, Forfaitinga cesionāram ir rakstveidā
jāinformē par to Cedents, norādot savus iebildumus. Gadījumā, ja
Cedents nepiekrīt Forfaitinga cesionāra izteiktajiem iebildumiem, strīds
ir jāizskata cienījamam ekspertam ar pietiekamām profesionālām
zināšanām un vērā ņemamu atpazīstamību starptautisko finanšu jomā,
kuru Puses ir kopīgi iecēlušas un kura lēmums ir galīgs, saistošs un spēkā
no tā pieņemšanas brīža. Ja Puses nevienojas par ekspertu strīda
risināšanai, katrai Pusei rodas tiesības iecelt vienu profesionālu
ekspertu, un abiem šādiem ekspertiem ir kopīgi jāizvēlas trešais
eksperts, kas vadīs ekspertu grupu. Ekspertu grupas lēmums ir saistošs,
un Pusēm ir līdzvērtīgi jāsadala iepriekšminētā procesa administrācijas
izdevumi.}

  \item{Izņemot tālāk un 9.punktā noteikto, Forfaitinga cesionāram nav
regresa prasījuma tiesību pret Cedentu par Debitora parāda
nesamaksu, ja vien:}

    \begin{enumerate}
    \item{nav pieļauts Energoefektivitātes pakalpojuma līgumā
noteikto Cedenta saistību pārkāpums; un/vai}

    \item{darbība vai bezdarbība neietilpst Cedenta saskaņā ar
Energoefektivitātes pakalpojuma līgumu uzņemtajās saistībās attiecībā
uz Energoefektivitātes paaugstināšanas pasākumu īstenošanu un to
apkopi }
    \end{enumerate}

  \item{Iepriekšminētais neierobežo Forfaitinga cesionāra tiesības prasīt
jebkādu saskaņā ar šo Līgumu nodoto Debitoru parādu realizāciju un
piedziņu, izmantojot pēc saviem ieskatiem mērķim vispiemērotāko
ārpustiesas vai tiesas procesu pret Cedenta klientu, tai skaitā, bez
ierobežojuma – ikvienu Dzīvokļa īpašnieku. Cedentam ir jānodrošina
visa nepieciešamā sadarbība saistībā ar šādu piespiedu izpildi.}

  \item{Cedentam ir jāatlīdzina un jāpasargā Cesionārs no jebkādas atbildības
un/vai prasījuma, kas rodas Forfaitinga cesionāram vai pret to, kas izriet
no vai rodas būtiska jebkādu Cedenta apliecinājumu, garantiju vai
saistību, kas izriet no šā Līguma vai Energoefektivitātes pakalpojuma
līguma, pārkāpuma rezultātā.}

  \item{Forfaitinga cesionāram ir jāatlīdzina un jāpasargā Cedents no jebkādas
atbildības un/vai prasījuma, kas rodas Cedentam vai pret to, kas izriet no vai rodas būtiska jebkādu Forfaitinga cesionāra apliecinājumu,
garantiju vai saistību, kas izriet no šā Līguma vai Energoefektivitātes
pakalpojuma līguma, pārkāpuma rezultātā, tomēr ar nosacījumu, ka
kopējā summa, kuru Forfaitinga cesionāram ir pienākums maksāt
Cedentam, tiek aprobežota ar vēl nesamaksāto saskaņā ar 4.4.1.
punktu pienākošos maksājumu atlikumu.}

  \item{. Saskaņā ar 7.3.un 7.4.punktu iepriekš, ja jebkura trešā persona uzsāk
jebkādu procesu vai izsaka jebkādu prasību pret Cedentu vai Forfaitinga
cesionāru (turpmāk katrs saukts – “Puse”), par kuru šī Puse (turpmāk –
“Kompensāciju saņemošā puse”) ir tiesīga uz kompensāciju saskaņā ar
šo Līgumu, šādai Atlīdzību saņemošajai pusei ir nekavējoties rakstveidā
jāinformē otra Puse (turpmāk – “Kompensējošā puse”) par šādu
procesu vai prasību. Pusei, kas šādā procesā ir uzņēmusies aizstāvību,
ir jāsniedz otrai Pusei visu šāda procesa ietvaros iesniegto vai izsniegto
paziņojumu, lūgumu un citu dokumentu kopijas. Neviena Puse nedrīkst
noslēgt nekādu izlīgumu vai veikt grozījumus bez otras Puses
piekrišanas, kas (i) Kompensējošās puses gadījumā nedrīkst tikt
nepamatoti atteikta, ja izlīgums vai grozījumi ir aprobežoti ar finansiālas
kompensācijas izmaksu; (ii) Kompensāciju saņemošās puses gadījumā
nedrīkst tikt nepamatoti atteikta, ja izlīgums vai grozījumi paredz, ka
Kompensāciju saņemošajai pusei tiek uzlikts izpildes pienākums vai tā
uzņemas atbildību. }

  \item{Katra kompensācija atbilstoši šim Līgumam ir saistības turpinājums,
atsevišķi un neatkarīgi no citām Pušu līgumsaistībām un paliek spēkā
pēc šā Līguma izbeigšanas, un Pusei nav nepieciešami izdevumi vai veikt
maksājumus pirms no šā Līguma izrietošo tiesību uz kompensāciju
realizācijas.}

  \item{Forfaitinga cesionārs ir tiesīgs pieprasīt un saņemt no Cedenta
atbilstošu un savlaicīgu visu Energoefektivitātes pakalpojuma līgumā
noteikto Cedenta pienākumu un saistību izpildi, kas ir tieši saistīti ar
Cedenta kā Izpildītāja atbilstoši Energoefektivitātes pakalpojuma
līgumam darbību vai saistībām un/vai tieši saistīti ar
Energoefektivitātes paaugstināšanas pasākumu apkopi.
Iepriekšminētais neierobežo Forfaitinga cesionāra tiesības pieprasīt un
saņemt noteiktu Cedenta Energoefektivitātes pakalpojuma līgumā
noteikto saistību pārjaunojumu nekavējoties, bet ne vēlāk kā 5 Darba
dienu laikā pēc attiecīga Cedenta paziņojuma saņemšanas, savlaicīgi
atbilstoši nepieciešamībai saņemt visus dokumentus un/vai papildu
paskaidrojumus un skaidrojumus attiecībā uz Ēku.}

 \item{Forfaitinga cesionāram kā personai, kurai Debitoru parādi tiek cedēti, ir
tiesības ierosināt jebkādu tiesas vai ārpustiesas procesu, kas ir
nepieciešams, lai iekasētu, piedzītu un saņemtu Debitoru parādus. Ja
Forfaitinga cesionāram būs jārealizē savas tiesības, tas informēs
Cedentu, un Cedents atbilstoši nepieciešamībai sadarbosies.
Forfaitinga cesionārs iesniegs Cedentam jebkādu rakstisku nodomu
vēstuli vai rakstiskus apliecinājumus, kas apstiprina vai ir nepieciešami
tiesību realizācijai, un Cedents tos parakstīs, ja tas varētu būt
nepieciešams, lai īstenotu plānoto tiesību realizāciju.}
  \end{enumerate}

  \subsection{PĀRVALDNIEKS UN ĒKAS APSAIMNIEKOŠANA}

  \begin{enumerate}
  \item{Cedentam ir jānodrošina, lai jebkurā brīdī būtu formas un satura ziņā
Forfaitinga cesionāram pieņemams:}

    \begin{enumerate}
    \item{līgums starp personu, kurš rīkojas kā pārvaldnieks (lai
iekasētu maksājumus no Cedenta klienta, nodrošinātu saziņu starp
Cedentu un Cedenta klientu un informētu Forfaitinga cesionāru par
visiem ar Debitoru parādiem saistītiem jautājumiem) (turpmāk –
„Pārvaldnieks”), Cedentu un Cedenta klientu (turpmāk – „Pārvaldības
līgums”); un}

    \item{līgums starp Cedentu, Forfaitinga cesionāru un
Pārvaldnieku (turpmāk – "Trīspusējais līgums").}
    \end{enumerate}
  \end{enumerate}

  \subsection{ATTIECĪGIE LĪGUMĀ NOTEIKTIE TERMIŅI – LĪGUMA
SPĒKĀ ESAMĪBAS PERIODS}

  \begin{enumerate}

  \item{Debitoru parādu cesijas saskaņā ar šo Līgumu un visu šajā Līgumā
noteikto Pušu saistību nosacījums ir visu priekšnosacījumu izpilde, ko
apliecina 4.punktā iepriekš noteiktā Nosacījumu izpildes protokola
parakstīšana.}

  \item{Visi šā Līguma grozījumi, papildinājumi un izmaiņas ir veicami
rakstveidā pēc visu Pušu savstarpējas vienošanās un stājas spēkā tikai
pēc tam, kad Puses tos ir parakstījušas.}

  \item{Cedents vienpusēji izbeidz šo Līgumu, rakstveidā paziņojot par to
Forfaitinga cesionāram un Pārvaldniekam vismaz 30 (trīsdesmit) dienas
iepriekš, ja Cedenta klients ir izbeidzis Energoefektivitātes pakalpojuma
līgumu, neatkarīgi no izbeigšanas iemesla. Šādā gadījumā Forfaitinga
cesionārs ir tiesīgs saņemt no Cedenta kompensāciju turpmāk
norādītās kopsummas apmērā: }

    \begin{enumerate}
    \item{visas Forfaitinga cesionāra Cedentam saskaņā ar šo Līguma
samaksātās summas, plus likumīgie procenti, kas uzkrājušies saistībā ar
tām, sākot no samaksas Cedentam datuma; plus}
    \item{visas Forfaitinga cesionāra ieguldījumu izmaksas un tā
kapitāla izmaksas, ieskaitot, bet neaprobežojoties ar jebkādiem saistību
maksājumiem, procentu likmēm un citām papildu piemaksām, kas
Forfaitinga cesionāram varētu būt radušās, un plānoto ieņēmumu
summu, kuru Cedents būtu saņēmis, ja Energoefektivitātes
pakalpojuma līgums nebūtu izbeigts.}
    \item{tomēr nerēķinot dubultā.}
    \end{enumerate}

  \item{Līgums nevar tikt izbeigts pirms Energoefektivitātes pakalpojuma
līguma izbeigšanas, ja vien Cedents un Forfaitinga cesionārs nav
rakstveidā vienojušies par citiem noteikumiem, vai arī Līgumā nav noteikts citādi.}

  \item{Šā Līguma izbeigšana neatbrīvo Puses no savu attiecīgo šajā Līgumā
noteikto saistību izpildes, kuru izpildes termiņš ir iestājies pirms šā
Līguma izbeigšanas, ja vien Puses nav rakstveidā vienojušās par citiem
noteikumiem, vai arī šajā Līgumā nav noteikt citādi. }

  \item{Gadījumā, ja Energoefektivitātes pakalpojuma līgums ir izbeigts pirms
termiņa jebkādu apstākļu dēļ un a Forfaitinga cesionāram nav
izmaksāta kompensācija saskaņā ar šā Līguma (jo īpaši 9.3.punkta)
noteikumiem:}

    \begin{enumerate}
    \item{Forfaitinga cesionāram ir tiesības saņemt jebkādu summu, kuru
Cedenta klientam ir jāmaksā saskaņā ar Energoefektivitātes
pakalpojuma līguma noteikumiem šādas izbeigšanas rezultātā (un šīs
summas veido daļu no Debitoru parādiem);}

    \item{ievērojot Energoefektivitātes pakalpojuma līgumā noteikumus,
ja jebkura no Cedenta uzstādītā Aprīkojuma demontāžas vai trūkuma
rezultātā Ēkai neradīsies nekāds būtisks kaitējums, Forfaitinga
cesionāram ir tiesības demontēt Cedentam piederošo Aprīkojumu Ēkā
un izvest to no Ēkas, nemaksājot par to Cedenta klienta, ar nosacījumu,
ka (i) šāda demontāža ir tehniski iespējama, būtiski nesabojājot Ēku, un
(ii) Cedenta klients ir saskaņā ar Energoefektivitātes pakalpojuma
līgumu samaksājis summu, kas ir mazāka nekā izmaksas par
Energoefektivitātes pakalpojuma līgumā noteikto siltumenerģijas un
apkopes pakalpojumu sniegšanu (x) apmērā, un 80 \% no Cedenta
kopējo ieguldījumu summas (y) apmērā, īstenojot Energoefektivitātes
paaugstināšanas pasākumus; un/vai}

    \item{gadījumā, ja pirmstermiņa izbeigšanas iemesls ir Cedenta
pieļauta neizpilde, tad Cedentam ir jāmaksā Forfaitinga cesionāram
tāda summa, kas kopā ar jebkurām citām saskaņā ar šo 9.6.punktu
saņemtajām summām ir līdzvērtīga summai, kuru Forfaitinga cesionārs
būtu tiesīgs saņemt saskaņā ar 9.3.punktu, ja Cedents būtu izbeidzis šo
Līgumu. Iepriekšminētais neierobežo Cedenta pienākumu pēc
Forfaitinga cesionāra pieprasījuma pārjaunot Forfaitinga cesionāra
norādītai trešajai personai jebkuras vai visas savas Energoefektivitātes
pakalpojuma līgumā noteiktā saistības, kas Cedentam ir jāveic 5 Darba
dienu laikā pēc rakstiska paziņojuma saņemšanas.}
    \end{enumerate}

  \item{Papildus šā Līguma noteikumiem Puses var izbeigt šo Līgumu jebkurā
laikā uz savstarpējas vienošanās pamata par šā Līguma izbeigšanas
nosacījumiem. Šajā punktā minētā vienošanās par izbeigšanu ir
noformējama rakstveidā un stājas spēkā pēc tam, kad to ir parakstījušas
visas Puses.}
  \end{enumerate}

  \subsection{APDROŠINĀŠANA}

  \begin{enumerate}
  \item{Energoefektivitātes pakalpojuma līguma spēkā esamības laikā un par
Dzīvokļu īpašnieku līdzekļiem Cedentam ir pastāvīgi jāuztur
apdrošināšanas polise par Ēku ar minimālo apdrošināšanas segumu
pret ugunsgrēku, zemestrīci, plūdiem, ūdens nodarītiem bojājumiem,
jebkādām citām dabas katastrofām, kas ietekmē Ēku, strukturāliem
bojājumiem, ko izraisījusi nosēšanās un krituši koki, par summu, kas nav mazāka par Ēkas atjaunošanas vērtību. Apdrošināšanas līgumam ir
jābūt noslēgtam atbilstoši noteikumiem un nosacījumiem, kas ietver
attiecīgās nozares standarta noteikumus, ar apdrošinātāju, kuram
saskaņā ar Latvijā piemērojamiem atzītiem reitingiem ir piešķirts
vismaz A+ reitings. Cedentam ir jāiesniedz Forfaitinga cesionāram šādas
apdrošināšanas polises oriģināls vai jāiesniedz polises vai cita
pamatojoša dokumenta, kas apliecina apdrošināšanas prēmijas valūtu
un samaksu, kopija. Jebkāda polise, kurā Cedents ir minēts kā
apdrošināšanas atlīdzības saņēmējs saskaņā ar apdrošināšanas
segumu, ir pienācīgi jānodod piemērojamā un tiesiski saistošā kārtībā
Forfaitinga cesionāram, lai tādējādi Forfaitinga cesionārs kļūtu par
patieso atlīdzības saņēmēju saskaņā ar apdrošināšanas polisi.}

  \item{Neievērojot neko pretēju, Forfaitinga cesionārs ir tiesīgs par saviem
līdzekļiem papildus apdrošināt Ēku, Energoefektivitātes
paaugstināšanas pasākumus, no Energoefektivitātes pakalpojuma
līguma izrietošo Debitoru parādu iekasēšanu un to piedziņu vai
jebkādus citus aktīvus, kas uzskatāmi par būtiskiem Cesionāra
komercdarījumā, bez vajadzības informēt vai saņemt tam Cedenta vai
tā Klienta piekrišanu.}
  \end{enumerate}

  \subsection{PRASĪJUMU CESIJA }

  \begin{enumerate}
  \item{Forfaitinga cesionārs var brīvi cedēt, ieķīlāt, apgrūtināt vai atsavināt
trešajām personām savas šajā Līgumā noteiktās tiesības un/vai
prasījumus (vienalga, vai tie radušies cesijas brīdī vai jebkurus
turpmākus prasījumus) attiecībā uz jebkuriem no Cedenta nodotajiem
Debitoru parādiem (kuru cesija vai atsavināšana var tikt veikta vairāku
cesiju veidā, kur katra attiecas uz Debitoru parādiem, kas pienākas no
viena vai vairākiem Dzīvokļu īpašniekiem), un jebkāda šāda
atsavināšana var notikt kā saskaņā ar šo Līgumu forfaitinga ietvaros
nodoto Debitoru parādu cesija jebkurai trešajai personai}

  \item{Cedents nedrīkst cedēt Debitoru parādus citādi, kā vien saskaņā ar šo
Līgumu. Cedents nedrīkst cedēt nekādas savas tiesības saņemt
maksājumus par Pakalpojumiem.}
  \end{enumerate}

  \subsection{KAM PIEDER UZSTĀDĪTĀS IEKĀRTAS –
ĪPAŠUMTIESĪBAS UZ APRĪKOJUMU}

  \begin{enumerate}
  \item{Šis Līgums neierobežo nekādus Energoefektivitātes pakalpojuma
līguma Vispārīgo noteikumu un nosacījumu [16.punkta] noteikumus un
Cedenta īpašumtiesības uz Energoefektivitātes paaugstināšanas
pasākumu īstenošanas nolūkā Ēkā uzstādīto Aprīkojumu, kuru var
atdalīt no Ēkas, neradot būtisku kaitējumu Ēkai}

  \item{Ievērojot iepriekšminēto un kā nodrošinājumu no šā Līguma
izrietošajām Cedenta saistībām, Forfaitinga cesionāram ir vienīgi savā
vai trešo personu labā jānodibina īpaša ķīla pār Ēkā uzstādīto
Aprīkojumu ne vēlāk kā 10 darba dienu laikā pēc attiecīgā Aprīkojuma
pieņemšanas-nodošanas akta parakstīšanas (atbilstoši
14.lapa no 24 Energoefektivitātes pakalpojuma līgumā ietvertajai definīcijai).
Iepriekšminētajā termiņā Cedents apņemas veikt visas tiesiskās un
faktiskās darbības, kas nepieciešamas ķīlas nodibināšanai, tai skaitā
pienācīgu reģistrāciju Latvijas Komercķīlu reģistrā pēc visu atbilstošo
juridisko dokumentu un papildu vienošanos parakstīšanas to
atbilstošajā tiesiskajā formā.}
  \end{enumerate}

  \subsection{KAS NOTIEK ŠĀ LĪGUMA SAISTĪBU NEIZPILDES
GADĪJUMĀ – ATBILDĪBA}

  \begin{enumerate}
  \item{Cedents nav atbildīgs par jebkādiem prasījumiem, kas saistīti ar
Debitoru parādu cesiju, izņemot šādos gadījumos:}
    \begin{enumerate}
    \item{Cedents ir pārkāpis savas Energoefektivitātes pakalpojuma
līgumā noteiktās saistības, kā rezultātā Forfaitinga cesionāram ir
nodarīti zaudējumi, un šo pārkāpumu rakstveidā ir atzinis Cedents vai
tas ir konstatēts atbilstoši Energoefektivitātes pakalpojuma līgumā
noteiktajai Strīdu izskatīšanas kārtībai, un šī Strīdu risināšanas kārtība
līdz ar nepieciešamajiem grozījumiem – mutatis mutandis – ir iekļauta
šajā Līgumā; un/vai}

    \item{par Cedenta kompensāciju, kas katrā gadījumā ir skaidri
norādīta šajā Līgumā.}
    \end{enumerate}

  \item{Cedentam ir jāatlīdzina un jāpasargā Forfaitinga cesionārs no jebkādas
atbildības un/vai prasījumiem, kas Forfaitinga cesionāram var rasties
vai pret to var tikt izvirzīti uz jebkādu šajā Līgumā vai
Energoefektivitātes pakalpojuma līgumā ietverto Cedenta
apliecinājumu, garantiju un apņemšanos būtiska pārkāpuma pamata
vai no tā izrietoši, tomēr ar nosacījumu, ka kopējā summu, kuru
Cedentam ir pienākums maksāt Forfaitinga cesionāram, tiek
aprobežota ar nesamaksāto Debitoru parādu summu, kas saskaņā ar
grafiku ir maksājama un kuru nav iekasējis Forfaitinga cesionārs}

  \item{Forfaitinga cesionāram ir jāatlīdzina un jāpasargā Cedents no jebkādas
atbildības un/vai prasījumiem, kas Cedentam var rasties vai pret to var
tikt izvirzīti uz jebkādu šajā Līgumā vai Energoefektivitātes pakalpojuma
līgumā ietverto Forfaitinga cesionāra apliecinājumu, garantiju un
apņemšanos būtiska pārkāpuma pamata vai no tā izrietoši, tomēr ar
nosacījumu, ka kopējā summu, kuru Forfaitinga cesionāram ir
pienākums maksāt Cedentam, tiek aprobežota ar simt procentiem
(100\%) no atbilstoši šim Līgumam nodoto Debitoru parādu kopējās
vērtības.}

  \item{Forfaitinga cesionāram ir tiesības nekavējoties atkāpties no šā Līguma
jebkādu turpmāk norādīto svarīgo iemeslu dēļ:}

    \begin{enumerate}
    \item{gaidāma Cedenta maksātnespēja vai likvidācijas vai
administratīva procesa uzsākšana attiecībā uz tā mantu vai arī šāds
likvidācijas vai administratīvais process nav uzsākts tikai tāpēc, ka nav
pietiekamu līdzekļu likvidatora vai administratora apmaksai;}

    \item{Cedents 14 dienu laikā pēc jebkāda rakstiska pieprasījuma
nesniedz informāciju un dokumentus, kas nepieciešami nodoto
Debitoru parādu iekasēšanai;}

    \item{Cedents pārkāpj jebkādu būtisku šā Līguma saistību, kā
rezultātā Forfaitinga cesionāram rodas zaudējumi;}

    \item{Cedents nav atklājis faktus, kas tam bijuši zināmi vai kurus tam
būtu vajadzējis zināt, kas liedz veiksmīgi iekasēt nodotos Debitoru
parādus}
    \end{enumerate}

  \item{Izņemot 5.3.un 5.6.punktos noteikto, ja Forfaitinga cesionārs atkāpjas
no šā Līguma attiecībā uz visiem vai jebkuru daļu no nodotajiem
Debitoru parādiem, Cedentam ir jāatmaksā jebkāda summa, kas
samaksāta saistībā ar attiecīgajiem Debitoru parādiem saskaņā ar IV.punktu iepriekš (izņemot jebkuras summas, kuras Forfaitinga cesionārs
jau ir saņēmis saistībā ar šiem Debitoru parādiem), un tam vairs nav
tiesības uz jebkādiem turpmākiem maksājumiem no Forfaitinga
cesionāra saskaņā ar šiem punktiem par attiecīgajiem Debitoru
parādiem. Šī summa ir jāsamaksā 14 dienu laikā pēc atkāpšanās.
Forfaitinga cesionāram par Cedenta līdzekļiem ir jāatdod Cedentam
attiecīgie Debitoru parādi un saistībā ar tiem saņemtās summas (ja
tādas būtu), atskaitot izmaksas un izdevumus.
Izbeidzot šo Līgumu, Forfaitinga cesionārs ir tiesīgs piedzīt no Cedenta
jebkādus izdevumus par juridisko palīdzību, kas ir nepieciešama, lai
iekasētu pienākošos debitoru parādus vai paturētu jebkādus
pienākošos maksājumus, kamēr nav atrisināts strīds.}
  \end{enumerate}

  \subsection{NEPĀRVARAMAS VARAS APSTĀKĻI}

  \begin{enumerate}
  \item{Jebkura ārkārtas situācija vai iepriekš neparedzams notikums, kam ir
visas tālāk uzskaitītās pazīmes, ir uzskatāms par nepārvaramas varas
apstākļiem:}

    \begin{enumerate}
    \item{Puses nevar paredzēt un ietekmēt tos;}

    \item{tie traucē Pusēm pildīt Līguma saistības;}

    \item{tie nevar tikt uzskatīti par attiecīgās Puses pieļautu kļūdu vai
nolaidību; }

    \item{var pierādīt, ka tie ir, vai vienoties, ka tie ir nepārvarami,
lai gan attiecīgā Puse ir pielikusi saprātīgas pūles, lai mēģinātu tos novērst.
Nepārvaramas varas apstākļi, piemēram, ietver, bet ne aprobežojas
ar karadarbību, dabas katastrofām un valsts pārvaldes iestāžu
normatīvajiem aktiem. Neņemot vērā iepriekšminēto, Aprīkojuma
trūkumi, nolīgtās kvalitātes vai kvantitātes ziņā nepilnīgi
Pakalpojumi, Cedenta izmantotie, nodrošinātie vai uzstādītie
materiāli vai to ekspluatācijas kavējumi (ja to nav izraisījuši
nepārvaramas varas apstākļi), Cedenta klienta strīdi streiki un
finansiālas grūtības NAV uzskatāmi par nepārvaramas varas
apstākļiem.}
    \end{enumerate}

  \item{Puses nav atbildīgas par pilnīgu vai daļēju Līgumā noteikto saistību
neizpildi, ja tās cēlonis ir nepārvaramas varas apstākļi. Puse, kas
paļaujas uz nepārvaramas varas apstākļiem, ir jāiesniedz otrai Pusei
pierādījumi par to.}

  \item{Pusei, kuras saistību izpilde nepārvaramas varas apstākļu rezultātā ir
kļuvusi apgrūtināta vai neiespējama, ir nekavējoties jāinformē otra
Puse par radušos stāvokli, norādot apstākļu aprakstu, iespējamo
ilgumu, sagaidāmās sekas un potenciālo to risinājumu.}

  \item{Pusēm ir jāveic nepieciešamās darbības, lai mazinātu nepārvaramas
varas apstākļu sekas, un jāveic saprātīgas darbības, lai mazinātu
jebkādu radīto kaitējumu.}

  \item{Ja nepārvaramas varas apstākļi ilgst vairāk nekā 6 (sešus) mēnešus
pēc kārtas, un nav sagaidāms, ka tie beigsies vēl nākamos 3 (trīs)
mēnešus, gan Cedentam, gan Forfaitinga cesionāram ir tiesības
vienpusēji izbeigt Līgumu un saņemt 9.3.punktā iepriekš noteikto
kompensāciju.}
  \end{enumerate}

  \subsection{KONFIDECIALITĀTE UN PERSONAS DATU AIZSARDZĪBA}

  \begin{enumerate}
  \item{Šā Līguma Puses vienojas neizpaust trešajām personām citu personu
Konfidenciālo informāciju (kur „Konfidenciālā informācija” nozīmē,
ievērojot 15.5.un 15.6.punktu, informāciju, kas atbilst 15.3.un 15.4.
punktā noteiktajām kategorijām), kā arī neizpaust datus par citiem,
kas varētu tikt izmantoti konkurences vai nelikumīgu darbību
veikšanas nolūkā gan Līguma spēkā esamības laikā, gan trīs gadus pēc
Līguma izbeigšanās}

  \item{Cedents nedrīkst pieprasīt vai informēt jebkādu Forfaitinga cesionāra
un/vai jebkādas juridiskas personas klientu, potenciālo klientu vai
sadarbības partneri, ka Cedentam ir zināms, ka Forfaitinga cesionārs
vēlas izveidot attiecības, lai apcirptu, anulētu, atsauktu, ierobežotu,
samazinātu vai citādi ierobežotu viņu darījumus ar Forfaitinga
cesionāru.}

  \item{Šā Līguma noslēgšanas vai izpildes gaitā iegūtā informācija, kas nav
vispārpieejama trešajām personām un kuras izpaušana, kā zināms
saņēmējai personai vai vajadzētu zināt saņēmējai personai, var kaitēt
izpaudējas personas likumīgajām tiesībām vai interesēm, uzskatāma
par konfidenciālu.}

  \item{Papildus 11.3.punkta noteikumiem, turpmāk uzskaitītā informācija
uzskatāma par konfidenciālu informāciju:}

    \begin{enumerate}
    \item{jebkāda ierobežotas pieejamības informācija par šā Līguma
Pusēm, to klientiem vai sadarbības partneriem, ieskaitot, bet
neaprobežojoties ar jebkādu informāciju, kurai ir jebkuras Puses
komercnoslēpuma statuss;}

    \item{jebkāda informācija par šā Līguma Pušu darba organizāciju,
krājumiem, iekārtām nu tehnoloģijām.}
    \end{enumerate}

  \item{Līguma 15.4.punktā sniegtais Konfidenciālas informācijas
uzskaitījums nav izsmeļošs, un uzskatāms, ka jebkura konfidenciāla
informācija, kas tomēr tiek izpausta, atbilst Konfidenciālas
informācijas definīcijai.}

  \item{Informācija, kas kļūst publiski pieejama trešo personu darbības
rezultātā, nevienai no Pusēm nepārkāpjot Līguma noteikumus, nav
uzskatāma par Konfidenciālu informāciju.}

  \item{Šā Līguma Puses var izpaust Konfidenciālo informāciju saviem
konsultantiem, ar nosacījumu, ka tās uzņemas atbildību, ja šādas
trešās personas saņemto Konfidenciālo informāciju izmanto
nelikumīgi vai neievēro šajā Līgumā noteiktos konfidencialitātes
pienākumus.}

  \item{Ja šā Līguma Pusei, kas ir saņēmusi Konfidenciālo informāciju, ir
pienākums šādu Konfidenciālo informāciju izpaust saskaņā ar Latvijā
spēkā esošajiem normatīvajiem aktiem vai rīkojumiem (vai
jebkuriem tiem tai piemērojamiem tiesību aktiem), tad: (i) šādai
Pusei ir atļauts izpaust šo informāciju; un (ii) paziņojums par
izpaušanas faktu ir nekavējoties jādod atklājējai Pusei, ja vien
piemērojamo normatīvo aktu prasības nenosaka citādi.}

  \item{Cedents un Forfaitinga cesionārs reklāmas nolūkā un nolūkā
informēt sabiedrību ir tiesīgi atklāt vispārīgu informāciju par
savstarpējo sadarbību, tai skaitā izpaust informāciju, kas jau ir
publiski pieejama par Pusēm, sadarbības raksturu, panākto Enerģijas
ietaupījumu un procesa termiņiem atbilstoši Līgumam, ciktāl tas
nepārkāpj citu pušu likumīgās tiesības un intereses saistībā ar
konfidenciālās informācijas aizsardzību. Ja šā Līguma Pusei ir šaubas
par konkrētas informācijas raksturu, pirms tās izpaušanas šīs
informācijas raksturs ir jāsaskaņo ar to Pusi (Pusēm), kuras likumīgās
tiesības un intereses var aizskart šīs informācijas izpaušana, ja šī Puse
šo informāciju uzskata par pakļautu Līgumā noteiktajam
konfidencialitātes pienākumam.}

  \item{Iepriekšminētais punkts neierobežo Forfaitinga cesionāra tiesības
ievākt, apstrādāt, uzglabāt, pārveidot un izplatīt visus no Cedenta
ievāktos datus nolūkā uzlabot savu pakalpojumu kvalitāti un
izstrādāt, pārvaldīt un uzturēt Energoefektivitātes pakalpojuma
tiešsaistes platformu, lai atbalstītu visus posmus un iesaistītās
personas tipiska Energoefektivitātes pakalpojuma projekta ietvaros.}

  \item{Neierobežojot 6.8.punktu, Cedents ar šo piekrīt, ka Forfaitnga
cesionārs ir tiesīgs ievākt, apstādāt, uzglabāt, izmanto un nodot
Cedenta saskaņā ar šo Līgumu sniegtos personas datus, tam veicot
savas likumā noteiktās saistības, nolūkā sniegt savus pakalpojumus
vai nodot tos jebkurai trešajai personai, kurai var tikt cedētas no šā
Līguma izrietošās tiesības vai saistības, tai skaitā, bet ne tikai jebkurai
personai, kas ir atbildīga par tiešsaistes platformas izstrādi,
īstenošanu, pārvaldīšanu un uzturēšanu, lai uzraudzītu īstenoto
Energoefektivitātes pasākumu veiktspēju. Forfaitinga cesionārs
datus var izmantot citiem mērķiem pēc tam, kad ir saņēmis Cedenta
vai atbilstošos gadījumos Cedenta klienta piekrišanu tam. Neņemot
vērā iepriekšminēto, jebkuri personas dati vai ar Ēku vai Aprīkojumu
saistīti dati paliek Cedenta klienta īpašumā.}

  \item{Cedents atzīst un piekrīt, ka Forfaitinga cesionārs vai jebkura trešā
persona, kurai Forfaitinga cesionārs ir nodevis šajā Līgumā ietvertās
tiesības, izmanto jebkādus anonīmus datus un informāciju saistībā ar
enerģijas patēriņu Ēkā vienalga, vai to ir sniedzis Cedents vai ieguvis
Cesionārs nolūkā atzīmēt un apkopot valsts, reģionāla vai ES mēroga
datubāzi vai ar mērķi Cedenta klientam izmantot kā atsauci vai
jebkādam iekšējam mērķim, kam ir piekritis Pasūtītājs.}
  \end{enumerate}

  \subsection{KURŠ PARAKSTA ŠO LĪGUMU – PUŠU PĀRSTĀVĪBA}

  \begin{enumerate}
  \item{Visos ar šo Līgumu saistītajos jautājumos Puses pārstāv to likumīgie
pārstāvji (ja tās ir juridiskas personas) vai pienācīgi un tiesīgi
pilnvarotas personas, kurām ir pārstāvības tiesības attiecīgās šā
Līguma Puses vārdā uzņemties saistības.}

  \item{Pušu pilnvarotie pārstāvji ir pilnvaroti pārstāvēt attiecīgo Pusi
saistībā ar visu juridisko jautājumu kārtošanu saistībā ar šo Līgumu
atbilstoši Latvijā spēkā esošajiem normatīvajiem aktiem vai
gadījumā, ja Puse ir reģistrēta ārpus Latvijas, tās jurisdikcijas
normatīvajiem aktiem, kurā šī Puse ir reģistrēta.}

  \item{Puses ir tiesīgas jebkurā laikā atsaukt savu pilnvaroto pārstāvju
pilnvarojumu, rakstveidā paziņojot par to pārējām Pusēm un
vienlaikus pilnvarojot citu pārstāvi, kura rakstiskās pilnvaras
pārstāvēt attiecīgo Pusi ir jāiesniedz pārējām Pusēm. Nomainot savu
pārstāvi, Pusei ir jānodrošina, lai otra Puse tiktu informēta par
pārstāvja nomaiņu ne vēlāk kā pārstāvja nomaiņas dienā. Minētais
paziņojums kļūst par daļu no šā Līguma.}

  \item{Papildus iepriekšminētajam, Cedents ir tiesīgs, ievērojot
Energoefektivitātes pakalpojuma līguma noteikumus, piesaistīt
jebkuru trešo personu kā savu apakšuzņēmēju vai sadarbības
partneri nolūkā izpildīt noteiktus darbus vai īstenot pakalpojumus ar
mērķi izpildīt šajā Līgumā vai Energoefektivitātes pakalpojuma
līgumā ietvertās saistības, iepriekš rakstveidā paziņojot pa to
Cedenta klientam, un paziņojums nododams tai skaitā pa e{-}pastu.}
  \end{enumerate}

  \subsection{KĀ RISINĀMI STRĪDI – STRĪDU RISINĀŠANAS KĀRTĪBA}

  \begin{enumerate}
  \item{Pusēm ar šo Līgumu saistītas nesaskaņas ir jāatrisina savstarpēju
sarunu ceļā, pieliekot visas pūles, lai atrisinātu attiecīgās nesaskaņas.
Šim nolūkam Pusēm ir jāsniedz savlaicīga rakstiska atbilde uz jebkuru
otras Puses vēstuli saistībā ar nesaskaņām, kā arī jāvelta saprātīgs
laiks, ciktāl tas ir iespējams, lai attiecīgās nesaskaņas atrisinātu
personiski.}

  \item{Ja Puses nepanāk savstarpēju vienošanos, un starp Pusēm pastāv
strīds, nesaskaņas vai prasības, kas izriet no šā Līguma, ir saistītas ar
to vai tā darbības apturēšanu, izbeigšanu vai spēkā neesamību, tās ir
jāatrisina meditācijas procesā saskaņā ar biedrības “Ēku
saglabāšanas un energotaupības birojs”, vienotais reģistrācijas
numurs 40008198558, vai tā tiesību un saistību pārņēmēja
meditācijas noteikumiem, kas ir spēkā šā Līguma darbības laikā un
pastāv, kad pirmo reizi strīds tiek nodots izskatīšanai. Ja starp Pusēm
pastāv strīds par tehniskiem jautājumiem, jebkura Puse var pieprasīt,
lai strīdu par konstatētiem apstākļiem risinātu saskaņā ar biedrības
“Ēku saglabāšanas un energotaupības birojs”, vienotais reģistrācijas
numurs 40008198558, apstākļu izmeklēšanas komisijas darba
kārtību.}

  \item{Ja Puses nepanāk savstarpēju vienošanos pēc meditācijas procesa un/vai apstākļu izmeklēšanas procesa,
strīds ir izšķirams Latvijas vispārējās jurisdikcijas tiesā saskaņā ar piemērojamajiem Latvijā
spēkā esošajiem normatīvajiem aktiem. Pieteikums ir iesniedzams
tiesā pēc piekritības atbilstoši atbildētāja dzīvesvietai vai juridiskajai
adresei, tomēr, ja tā nav Latvijā, tad Rīgas pilsētas centra rajona tiesā
vai Rīgas apgabaltiesā atbilstoši prasības piekritības noteikumiem.}
  \end{enumerate}

  \subsection{CITI NOTEIKUMI}

  \begin{enumerate}
  \item{Jebkāds rakstisks paziņojums saistībā ar šo Līgumu uzskatāms par
saņemtu 7. (septītajā) dienā pēc paziņojuma nosūtīšanas dienas, ja
tas ir nosūtīts ierakstītā sūtījumā uz jaunāko tās Puses pilnvarotā
pārstāvja norādīto adresi, kurai tas ir adresēts. Paziņojuma
nosūtīšanas diena ir tā diena, kad paziņojums ir nodots pastā, un to
apstiprina pasta zīmogs. Pa elektronisko pastu uz norādīto vai Pušu
paziņoto e-pasta adresi uzskatāmi par rakstiskiem paziņojumiem ar
visām no tā izrietošajām sekām, un tie ir uzskatāmi par saņemtiem
no brīža, kad tie ir sasnieguši saņēmēja operētājsistēmu neatkarīgi
no tā, vai tie ir izlasīti vai ne.}

  \item{Šis Līgums ir sastādīts un no tā izrietošās tiesības un pienākumi ir
iztulkojami, kā arī Pušu darbība ir vērtējama saskaņā ar Latvijas
Republikas piemērojamajiem tiesību aktiem.}

  \item{Šā Līguma sadaļu vai punktu virsraksti nevar tikt izmantoti šā Līguma
interpretācijā.}

  \item{Šis Līgums satur visus noteikumus, apsolījumus, nosacījumus un
nolūku apliecinājumus starp Pusēm, un Puses garantē, ka nav nekādu
mutisku noteikumu, apsolījumu, vienošanos, nosacījumu un nolūku
apliecinājumu, izņemot tos, kas minēti šajā Līgumā. }

  \item{Ja šā Līguma spēkā esamības laikā spēkā stājas grozījumi Latvijas
normatīvajos aktos, kuru dēļ jebkādu šajā Līgumā ietverto saistību
izpilde kļūst pilnīgi vai daļēji neiespējama, vai izmainās jebkuras
Puses pienākumu izpildes nosacījumi, tas neietekmē pārējo šā
Līguma noteikumu spēkā esamību, un Pusēm ir jāapņemas veikt
nepieciešamos grozījumus šajā Līgumā, kas vislabāk īsteno šā Līguma
sākotnējo nolūku un mērķi, un saimniecisko ietekmi.}

  \item{Pušu reorganizācija, kā arī Pušu akcionāru/dalībnieku (īpašnieku) vai
vadības struktūru pārstāvju (valdes locekļu) vai Cedenta klienta
nomaiņa (Ēkas Dzīvokļu īpašnieki atsavina īpašumtiesības uz
Dzīvokļiem un/vai izīrē tos īstermiņā vai ilgtermiņā) nav pamats šā
Līguma izbeigšanai vai šajā Līgumā noteikto saistību neizpildei. Ja
16.lapa no 24 jebkura no Pusēm ir iesaistīta uzņēmumu apvienošanās vai iegādes
procesā, attiecībā uz to ir sākts likvidācijas vai bankrota process, vai
Ēkas Dzīvokļu īpašnieki mainās, šis Līgums paliek spēkā, un tā
noteikumi ir saistoši attiecīgās Puses tiesību un saistību
pārņēmējiem. Katrai Pusei ir nekavējoties, bet ne vēlāk kā 5 (piecu)
dienu laikā pēc šādām izmaiņām, jāinformē pārējās Puses par šajā
Līgumā norādītās Puses adreses maiņu vai citām izmaiņām tās
tiesiskajā statusā vai pārvaldībā.}

    \item{Līgums ir sastādīts un parakstīts 2 (divos) eksemplāros [angļu]
[latviešu] valodā. Visiem eksemplāriem ir vienāds juridisks spēks.
Katra Puse patur 1 (vienu) šā Līguma eksemplāru. Puses ar saviem
parakstiem apliecina, ka ir sapratušas šā Līguma saturu, nozīmi un
sekas; tās atzīst, ka šis Līgums ir pareizs, savstarpēji izdevīgs un ka tās
labprātīgi vēlas parakstīt to.}
  \end{enumerate}

  \subsection{PIELIKUMI}

  \begin{enumerate}
  \item{Šā Līguma parakstīšanas brīdī šim Līgumam ir pievienoti šādi  pielikumi:}
   \begin{enumerate}
	\item{PIELIKUMS:PAZIŅOJUMS PAR DEBITORU PARĀDU CESIJU}
	\item{PIELIKUMS:CEDENTA KLIENTS/ PASŪTĪTĀJS}
	\item{ PIELIKUMS:ENERGOEFEKTIVITĀTES PAKALPOJUMA LĪGUMS}
	\item{PIELIKUMS:PILNVAROTIE PĀRSTĀVJI}
	\item{PIELIKUMS:DEFINĪCIJAS}
    \end{enumerate}

  \item{Visi pielikumi uzskatāmi par šā Līguma neatņemamu sastāvdaļu un ir
Pusēm saistoši. Pretrunu gadījumā starp šā Līguma un jebkuru
pielikumu terminiem, šā Līguma noteikumi ir noteicošie, izņemot tos
pielikumus, kas parakstīti pēc šā Līguma parakstīšanas. Pēc šā Līguma
parakstīšanas parakstītie pielikumi ir noteicošie salīdzinājumā ar šā
Līguma noteikumiem}

  \item{Parakstot šo Līgumu, tiek apliecināts, ka Puses ir iepriekš
apspriedušas Līguma un tam pievienoto pielikumu noteikumus, un
visām Pusēm tie ir skaidri un saprotami.}
  \end{enumerate}

\end{multicols}

\vspace{5mm}

{{template "sign.tex"}}

\begin{center}
  \section{ANNEXES}
\end{center}

\vspace{1cm}

\subsection{1. PIELIKUMS.PAZIŅOJUMS PAR DEBITORU PARĀDU CESIJU}

\vspace{2mm}

\begin{tabu}{|X|} \tabucline{}

  Kam: {{.Contract.Fields.client_name}} \par

  \begin{flushright}
    {{index .Contract.Agreement "place-of-forfaiting-agreement"}}, {{index .Contract.Agreement "date-of-forfaiting-agreement"}}
  \end{flushright}

  \begin{alltt}  Debitoru parādu [nodošana] [cesija]



    Godātie kungi/dāmas



    % TODO: @edimov Fix manager-name to be input %
    Mēs atsaucamies uz Energoefektivitātes pakalpojuma līgumu Nr.  {{.Contract.Fields.date}},datumā parakstīts starp Jums un mums, un informējam Jūs,
ka saskaņā ar {{.Contract.Agreement.front_page_date}} Forfaitinga līgumu mēs esam [nodevuši] [cedējuši] visus no Energoefektivitātes pakalpojuma līguma izrietošos
Debitoru parādus, tai skaitā Atlīdzību, kas aprēķināta un maksājama atbilstoši Energoefektivitātes pakalpojuma līguma 5.pielikumā sniegtajai
definīcijai Forfaitinga cesionāram,   {{index .Contract.Fields "client_name"}},  {{index .Contract.Fields "client_id"}}  {{index .Contract.Fields "client_address"}} (turpmāk – „Forfaitinga
cesionārs"). Visi Atlīdzības maksājumi kopā ar atlikumiem un naudassodiem, un citam summām atbilstoši Energoefektivitātes pakalpojuma
līgumā noteiktajam turpmāk ir maksājami Forfaitinga cesionāram, lai dzēstu Jūsu Energoefektivitātes pakalpojuma līgumā paredzētās
saistības. Forfaitinga cesionārs ir nozīmējis Jūsu esošo Pārvaldnieku, {{index .Contract.Agreement "manager-name"}} (the 'Manager'), (turpmāk – „Pārvaldnieks"), tā vārdā iekasēt visas maksājamās
summas.
  \end{alltt}

  \vspace{5mm}

  Mēs lūdzam Jūs pieņemt zināšanai Forfaitinga līgumu un saistīto debitoru parādu pārdošanu un informēt Forfaitinga cesionāru par savu
akceptu, izmantojot pievienoto veidni.

  \vspace{1cm}

    Ar cieņu,

  \vspace{1cm}

   {{.ESCo.Name}}, Assignor

  \vspace{1cm} \\\tabucline{}

\end{tabu}

\pagebreak

\begin{tabu}{|X|} \tabucline{}

  {{.ESCo.Name}}, Cedents\par

  \begin{flushright}
     {{index .Contract.Agreement "place-of-forfaiting-agreement"}},  {{index .Contract.Agreement "date-of-forfaiting-agreement"}} \par
  \end{flushright}

\begin{alltt}

 Debitoru parādu [nodošana] [cesija]


Godātie kungi/dāmas,


Mēs, {{index .Contract.Agreement "assignors-client-name"}}, atsaucamies uz iepriekšminēto paziņojumu, ar kuru mēs tiekam informēti un
uzzinām kā {{index .Contract.Agreement "assignor-name"}}(turpmāk – “Cedents") klients, ka Cedents ir
 {{index .Contract.Agreement "forfaiting-assignee"}} (turpmāk – "Forfaitinga cesionārs") nodevis savas likumīgās tiesības uz debitoru parādiem, kas
tam pienākas no mums saistībā ar mums un Cedentu {{index .Contract.Agreement "date-of-energy-contract"}} noslēgto Energoefektivitātes pakalpojuma līgumu (turpmāk –
“Energoefektivitātes pakalpojuma līgums”) (šādu nodošanu veicot saskaņā ar  {{index .Contract.Agreement "date-of-forfaiting-agreement"}}
starp Cedentu un Forfaitinga cesionāru noslēgto
Forfaitinga līgumu). Parakstot šo paziņojumu, mēs apstiprinām Cedentam un Forfaitinga cesionāram, ka mēs esam pienācīgi informēti par
iepriekšminēto Atlīdzības nodošanu un cesiju un atzīstam Forfaitinga cesionāru par Atlīdzības un visu citu debitoru parādu kreditoru, kas
pienākas no mums saskaņā ar Energoefektivitātes pakalpojuma līguma noteikumiem. Kamēr Forfaitinga cesionārs nebūs mūs informējis
citādi, mēs visas šādas summas maksāsim [ ] (turpmāk – "Pārvaldnieks") vai citai Forfaitinga cesionāra norādītai personai.
\end{alltt}

  \vspace{2mm}

  Mēs tāpat atzīstam, ka Forfaitinga cesionārs var izmantot Energoefektivitātes pakalpojuma līgumā noteiktās Cedenta tiesības jaunu izpildītāju
nomainīt ar Cedentu.

  \vspace{2cm}

  \begin{flushright}
    {\_\_\_\_\_\_\_\_\_\_\_\_\_\_} \par
    Vārds: \par
    Pilnvarojums: \par
  \end{flushright}

  \vspace{1cm} \\\tabucline{}

\end{tabu}

{{template "sign.tex"}}

\subsection{2. PIELIKUMS.CEDENTA KLIENTS / PASŪTĪTĀJS}

\iffalse input forfaitingFields.annex2_tarea value="{{.Contract.Agreement.annex2_tarea}}" type="textarea" \fi {{.Contract.Agreement.annex2_tarea}}

\pagebreak

\subsection{3. PIELIKUMS.ENERGOEFEKTIVITĀTES PAKALPOJUMA LĪGUMS}

\url{ {{.Attachments.signed_epc }} }

\iffalse attachment value="signed epc" \fi

\pagebreak

\subsection{4. PIELIKUMS.PILNVAROTIE PĀRSTĀVJI}

\iffalse input forfaitingFields.annex4_tarea value="{{.Contract.Agreement.annex4_tarea}}" type="textarea" \fi {{.Contract.Agreement.annex4_tarea}}

\pagebreak

\subsection{5. PIELIKUMS.DEFINĪCIJAS}


\renewcommand*{\theenumi}{\roman{enumi}}

Puses vienojas, ka šā Līguma izpratnē tālāk norādītajām definīcijām ir tālāk aprakstītā nozīme: \par

\begin{longtabu}{|X|X[2]|} \tabucline{}

  % Include `\hline` after every row to be sure that
  % if this is the last row of the page, to ensure upper and lower
  % bold line and remove the "empty" boxes

   Līgums & Šis forfaitinga līgums, kas noslēgts starp Cedentu un Forfaitinga cesionāru. \\\tabucline{}

   \hline

   Dzīvoklis & Katrs Ēkā esošs dzīvokļa īpašums normatīvo aktu izpratnē. \\\tabucline{}

   \hline

   Dzīvokļa īpašnieks  & Fiziska vai juridiska persona, uz kuras vārda Dzīvoklis ir ierakstīts Zemesgrāmatā, vai cita persona, kurai saskaņā ar
normatīvajiem aktiem pieder īpašumtiesības uz attiecīgo Dzīvokli. \\\tabucline{}

   \hline

   Cedenta klients &  {{index .Contract.Fields "client_name"}} turpmāk tekstā saukts arī “Pasūtītājs”.\\\tabucline{}

   \hline

   Sākotnējie rādītāji
(patēriņš/apjoms/ tarifs) & Sākotnējie rādītāji (patēriņš/apjoms/tarifs) apraksta apstākļus ēkā pirms Energoefektivitātes paaugstināšanas
pasākumu īstenošanas un ietver pamata gada datus par enerģiju un ēkas apstkākļus pamata gadā. \\\tabucline{}

   \hline

   Pamata gada dati par enerģiju & Enerģijas patēriņš gados, kas izraudzīti kā izejas pozīcija un apraksta ēkas energoefektivitāti pirms energoefektivitātes
paaugstināšanas pasākumu īstenošanas. \\\tabucline{}

   \hline

  Pamata gada apstākļi & Apstākļu kopums, kas bija iemesls enerģijas izlietojumam/pieprasījumam pamata gadā, jo īpaši istabas temperatūra,
āra temperatūra, apkures sezonas ilgums. \\\tabucline{}

   \hline

   Ēka & Daudzdzīvokļu dzīvojamā ēka {{index .Contract.Fields "client_address"}}. \\\tabucline{}

   \hline

   Darba diena & Oficiāla darba diena, kas nav izsludināta kā oficiāla brīvdiena vai oficiāla atpūtas diena atbilstoši Latvijas tiesību aktu
normām. \\\tabucline{}

   \hline

   Komforta standarti  & Nepieciešamais minimālais komforta līmenis, kas norādīts Energoefektivitātes pakalpojuma līguma 2. pielikumā
(“Komforta standarti”) \\\tabucline{}

   \hline

   Aprīkojuma pieņemšana un
nodošana & Pēc šajā Līgumā nolīgto Energoefektivitātes paaugstināšanas pasākumu īstenošanas pabeigšanas Ēkā Pasūtītāja
pārstāvis ir jāuzaicina uz Aprīkojuma pieņemšanu un nodošanu. Pasūtītāja un Izpildītāja pārstāvjiem ir jāparaksta
divpusējs akts, kas apliecina izpildītos darbus un detalizēti (tai skaitā ar foto materiāliem) apraksta stāvokli, kādā
Aprīkojums tiek nodots Pasūtītaāja valdījumā. \\\tabucline{}

   \hline

   Energoefektivitātes
paaugstināšanas pasākumi & Energoefektivitātes pakalpojuma līgumā norādītās darbības, kuru rezultātā tiek panākts pārbaudāms un izmērāms vai
aprēķināms Enerģijas ietaupījums. Energoefektivitātes paaugstināšanas pasākumi un siltumenerģijas piegādes pasākumi
ietver visus plānošanas, tehniskos un organizatoriskos pasākumus, kurus īsteno Izpildītājs, pamatojoties uz
Energoefektivitātes pakalpojuma līguma noteikumiem, tai skaitā, bet ne tikai pasākumu īstenošanas projekta sagatavošana,
Enerģijas ietaupījuma aizsardzības, mērījumu un pārbaudes pasākumi un optimizācijas nolūkā īstenojami pasākumi, un
uzturēšanas pasākumi, kā arī visas saistītās un atbilstošās darbības, ko Izpildītājs pamatoti uzskata par nepieciešamām
saskaņā ar šādu līdzīgu Energoefektivitātes paaugstināšanas pasākumu īstenošanas standartiem.\\\tabucline{}

   \hline

   Enerģijas tarifs & Maksa par Enerģijas vienību, kas tiek mērīta mWh un ko nosaka Enerģijas piegādātājs ēka atrašanās vietā, kas ir spēkā
noteiktu laika periodu un kuru apstiprina attiecīgais Regulators/saskaņā ar Līgumu.\\\tabucline{}

   \hline

   Garantētais enerģijas patēriņš & Pēc klimata korekcijām garantētais enerģijas daudzums [•] MWh apmērā, kas patērējams pēc Energoefektivitātes
paaugstināšanas pasākumu īstenošanas un kas veido  \% palielinājumu Ēkas energoefektivitātes ziņā salīdzinājumā ar
Sākotnējo apjomu. Lai novērstu šaubas, Izpildītājs garantē, ka koriģētais Ēkas enerģijas patēriņš nepārsniegtu Izpildītāja
garantētos apjomus, un atsevišķi apņemas pielikt papildu, izmaksu ziņā saprātīgas pūles, lai vēl vairāk samazinātu enerģijas
patēriņu visā šā Līguma darbības laikā.\\\tabucline{}

   \hline

   Aprīkojums & Izpildītāja uzstādītās vai citādi Ēkā ieguldītās iekārtas, sistēmas, piederumi un citi Energoefektivitātes paaugstināšanas
pasākumi nolūkā uzlabot vai saglabāt tās energoefektivitāti vai pārbaudīt, kontrolēt, izmērīt, uzturēt vai remontēt tās
saskaņā ar Energoefektivitātes pakalpojuma līgumu.\\\tabucline{}

   \hline

   Atlīdzība & Periodisks maksājums, kuru Pasūtītājs maksā Izpildītājam par Līgumā noteikto Pakalpojumu sniegšanu Pakalpojumu
sniegšanas periodā, kas ietver Maksu par enerģiju, Maksu par renovāciju un Ekspluatācijas un apkopes maksu.\\\tabucline{}

   \hline

   Apkures sezona &Ikgadējs periods laikposmā no 30. oktobra līdz 1. aprīlim, kurā Izpildītājam ir jānodrošina apkures sistēmas darbība
(siltumapgāde) Ēkas Dzīvokļos un jāgarantē atbilstība Līgumā noteiktajiem Komforta standartiem.\\\tabucline{}

   \hline

   LABEEF &Nozīmē Latvian Baltic Energy Efficiency Facility AS, kas darbojas kā Latvijas Komercreģistrā pienācīgi reģistrēta akciju
sabiedrība ar uzņēmuma reģistrācijas Nr. 40103960646.\\\tabucline{}

   \hline

   Pakalpojumi&Saskaņā ar Energoefektivitātes pakalpojuma līgumu Cedenta īstenotie Energoefektivitātes paaugstināšanas pasākumi,
Enerģijas piegāde un Aprīkojuma apkope.\\\tabucline{}

   \hline

   Pakalpojumu sniegšanas periods & Periods, kurā Cedentam ir pienākums sniegt savam Pasūtītājam Pakalpojumus saskaņā ar Energoefektivitātes
pakalpojuma līgumu.\\\tabucline{}

   \hline

   Pārvaldnieks & Fiziskā vai juridiskā persona, kas saskaņā ar Latvijas Dzīvojamo māju pārvaldīšanas likuma piemērojamajiem
noteikumiem un uz pārvaldīšanas līguma pamata veic Pasūtītāja uzdotās pārvaldīšanas darbības Ēkā, kā arī šajā Līgumā
24.lapa no 24 un Energoefektivitātes pakalpojuma līgumā noteiktos pienākumus. Lai novērstu šaubas, pārvaldīšanas līgums, atbilstoši
kuram Pārvaldnieks sniedz savus pakalpojumus, atsauces nolūkā ir jāpievieno šim Līgumam.\\\tabucline{}

   \hline

  Puse & Cedents un Cesionārs katrs atsevišķi.\\\tabucline{}

   \hline

   Puses & Cedents un Cesionārs kopā.\\\tabucline{}

   \hline

   Projekts & Projekts nozīmē visu daudzdzīvokļu dzīvojamo māju vai publisko ēku visaptverošu renovāciju, kas īstenota uz pievienotā
Energoefektivitātes pakalpojuma līguma pamata un ir daļa no šā Līguma, pateicoties kurai tiek panākta minimālais
Enerģijas patēriņa apjoms.\\\tabucline{}

   \hline

   PVN & Pievienotās vērtības nodoklis, kas maksājams saskaņā ar Latvijas normatīvajiem aktiem un Līguma noteikumiem.\\\tabucline{}

 \end{longtabu}

 \vspace{5mm}

 Ja vien konteksts nenosaka citādi, lietots šajā Līgumā, vienskaitlis ietver daudzskaitli, un daudzskaitlis var ietvert vienskaitli. Jebkura dzimuma lietošana var būt
attiecināta uz visiem dzimumiem. Ja vien nav norādīts citādi, atsauces uz Punktiem, Nodaļām vai apakšnodaļām ir atsauces uz šā Līguma Punktiem, Nodaļām un
apakšnodaļām. Ja vien konteksts nenosaka citādi, apzīmējums “ieskaitot/tai skaitā” nozīmē “ieskaitot, bet neaprobežojoties”. \par

\vspace{5mm}

{{ template "sign.tex"}}

\end{document}
