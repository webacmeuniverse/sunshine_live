\section{Annex 6 {-} Rules of determination of a balance payment after measurement and evaluation of the result}

\subsection{At a value of EC less than one, AAEC exceeds GAEC, and the Contractor pays the Client a compensation.}

\subsection{The compensation shall be calculated by the following
  formula: CP = VAAEC – VGAEC, where}

\begin{enumerate}
\item CP is the compensation payment.
\item VAAEC is the sum of the products of the achieved consumption of
  every one of the energy carriers at their price, determined in the
  specific terms of the contract.
\item VGAEC is the sum of the products of the guaranteed annual
  consumption of every one of the energy carriers by their price,
  determined in the specific terms of the contract.
\end{enumerate}

\subsection{The compensation means a sanction for non-achievement of the guaranteed energy saving. The parties can choose the compensation to be paid with the Client’s obligations from the next period.}

\subsection{At the value of EC more than one, the Client shall follow to pay the Contractor a premium payment.}

\subsection{The premium payment shall be calculated by the following formula: PP= VGAEC – VAAEC, where}
\begin{enumerate}
	\item PP is the premium payment
\end{enumerate}

\subsection{The premium payment means a bonus for exceeding the guaranteed economy.}

\subsection{An invoice for premium payment shall be issued by the Contractor within 10 (ten) business days after calculation of EC and sending of the annual monitoring report to the Client.}
