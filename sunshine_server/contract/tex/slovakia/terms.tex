\begin{multicols}{2}
[\section{VŠEOBECNÉ PODMIENKY}]
\section{VŠEOBECNÉ PODMIENKY}

\subsection{DEFINÍCIE POJMOV}
\begin{enumerate}


\item\textbf{Zmluva}: táto Zmluva o poskytovaní energetických služieb
uzavretá medzi Objednávateľom a Dodávateľom vrátane Osobitných podmienok
a ich Príloh, ako aj Všeobecných podmienok, vypracovaná a spravovaná
prostredníctvom Platformy pre Zmluvy o poskytovaní energetických služieb
\href{http://www.sharex.lv}{http://sunshineplatform.eu/}.

\item\textbf{Byt}: zákonne vyčlenená autonómna nehnuteľnosť v bytovom dome.

\item\textbf{Vlastník bytu}: osoba, ktorá nadobudla byt a ktorej vlastnícke
právo k bytu je registrované v Katastri nehnuteľností.

\item\textbf{Bankový deň:} deň, počas ktorého v mieste, kde sa podľa
ustanovení Zmluvy majú vykonávať bankové prevody, vykonávajú komerčné
banky všeobecné bankové transakcie.

\item\textbf{Základná hodnota}: spotreba tepelnej energie a teplej úžitkovej
vody v Budove vyjadrená ako ročná priemerná hodnota počas Základného
obdobia.

\item\textbf{Základné obdobie}: vzájomne dohodnuté časové obdobie
predstavujúce fungovanie Budovy pred implementáciou Opatrení.

\item\textbf{Budova}: bytový dom s viacerými bytmi, v ktorom Dodávateľ dodáva
Renovačné práce a poskytuje Služby na základe tejto Zmluvy.

\item\textbf{Pracovný deň}: oficiálny pracovný deň, ktorý nie je oficiálnym
sviatkom alebo oficiálnym dňom pracovného pokoja v súlade so zákonmi
Slovenskej republiky.

\item\textbf{Objednávateľ:} Vlastník (-ci) bytu v Budove alebo oprávnená
osoba alebo právnická osoba konajúca v mene Vlastníka (-kov) bytu.

\item\textbf{Štandardy komfortu}: súbor podmienok a parametrov vnútornej
klímy, ktoré garantuje Dodávateľ Objednávateľovi na základe tejto
Zmluvy.

\item\textbf{Dátum začatia}: dátum, ktorým sa začína Obdobie výstavby.

\item\textbf{Dátum uvedenia do prevádzky}: dátum, v ktorý Zmluvné strany
podpíšu Protokol o odovzdaní a prevzatí Opatrení, ako aj dátum, ktorým
sa začína Doba poskytovania služieb podľa Zmluvy.

\item\textbf{Obdobie výstavby}: obdobie plánované Dodávateľom na
implementáciu Opatrení. Obdobie výstavby sa začína Dňom začatia a končí
Dňom uvedenia do prevádzky.

\item\textbf{Dodávateľ}: právnická osoba, ktorá uzatvára túto Zmluvu, vykoná
Renovačné práce a poskytne Služby na základe ustanovení Zmluvy.

\item\textbf{Odovzdávací a preberací protokol}: protokol vypracovaný
Dodávateľom v súlade s predpismi a Normami platnými v Slovenskej
republike vzťahujúcimi sa na konečné uvedenie Opatrení realizovaných
Dodávateľom v Budove do prevádzky.

\item\textbf{Poplatok za teplú úžitkovú vodu}: poplatok zaplatený
Objednávateľom Dodávateľovi za skutočnú spotrebu teplej úžitkovej vody
vo výške podľa aktuálnej Ceny tepelnej energie.

\item\textbf{Energia}: produkt určitej hodnoty - palivo, tepelná energia,
obnoviteľná energia, elektrina alebo akýkoľvek iný druh energie.

\item\textbf{Energetická analýza a energetický audit}: činnosti vykonávané s
cieľom získať informácie o štruktúre spotreby energie v budovách alebo
skupinách budov, postupoch alebo zariadeniach, a tiež zmerať a overiť
možnosti ekonomicky uskutočniteľných úspor energie, pričom zistenia
vyplývajúce z týchto činností sa uvedú do správy.

\item\textbf{Garantovaná spotreba energie}: množstvo tepelnej energie
spotrebovanej v Budove na vykurovanie priestorov a~tepelnú stratu
cirkulácie po Dobu poskytovania služieb dosiahnuté na základe
Garantovanej úspory energie poskytnutej Dodávateľom a použitej na
stanovenie Paušálnej spotreby tepelnej energie.

\item\textbf{Služba (služby) energetickej efektívnosti}: súbor činností
vykonávaných Dodávateľom vrátane implementácie Opatrení v rámci Budovy,
prevádzky a údržby implementovaných Opatrení, analýz údajov o spotrebe
energie, monitorovania a hodnotenia spotreby energie, ktoré súvisia
predovšetkým s dosiahnutím Garantovanej úspory energie.

\item\textbf{Úspora energie}: objem ušetrenej energie, ktorý sa stanoví
meraním a overovaním spotreby pred a po implementácii jedného alebo
viacerých opatrení energetickej hospodárnosti a dosiahne v Budove
implementáciou Opatrení a zabezpečením energetickej hospodárnosti.

\item\textbf{Garantovaná úspora energie}: minimálna výška Úspory energie
vyplývajúca z implementácie Opatrení a z poskytovania Služieb
energetickej efektívnosti, garantovaná Dodávateľom v Zmluve a určená na
základe Plánu meraní a overovaní.

\item\textbf{Cena energie}: poplatok za energetickú jednotku v~mieste,
v~ktorom sa nachádza Budova.

\item\textbf{Platforma pre Zmluvy o poskytovaní energetických služieb
--http://sunshineplatform.eu/}: viacstranná online platforma určená pre
všetkých zúčastnených na uzatváranie Zmlúv o poskytovaní energetických
služieb, ktorú nájdete na
\href{http://www.sharex.lv}{http://sunshineplatform.eu/}, ktorej účelom
je podporiť vývoj a riadenie projektov renovácie Budov na základe Zmlúv
o poskytovaní energetických služieb.

\item\textbf{Poplatok (-y)}: opakujúce sa mesačné Poplatky, ktoré
Objednávateľ platí Dodávateľovi za poskytovanie služieb špecifikovaných
v Zmluve po Dobu poskytovania služieb, zahŕňajúce Poplatok za tepelnú
energiu, Poplatok za teplú úžitkovú vodu, Poplatok za renováciu a
Poplatok za prevádzku a údržbu spolu s prípadnými splatnými daňami
(napríklad DPH).

\item\textbf{Paušálna spotreba tepelnej energie}: množstvo tepelnej energie
vypočítané Dodávateľom na účely účtovania pevne stanoveného mesačného
množstva tepelnej energie Objednávateľovi za každé Vyúčtovacie obdobie
po Dobu poskytovania služieb.

\item\textbf{Finančný príspevok}: podiel investičných nákladov na Renovačné
práce financovaný Dodávateľom priamo z vlastného kapitálu alebo
poskytnutý nepriamo ako financovanie treťou stranou, v súvislosti s
ktorými si Dodávateľ účtuje Poplatok za renováciu.

\item\textbf{Poplatok za tepelnú energiu}: poplatok zaplatený Objednávateľom
Dodávateľovi za energiu spotrebovanú Budovou po Dobu poskytovania
služieb, s výhradou úprav a~zostatku, vyúčtovaný raz ročne na konci
Vyúčtovacieho obdobia, berúc do úvahy skutočné poveternostné podmienky
počas Vyúčtovacieho obdobia a Meranie a overovanie Garantovanej úspory
energie.

\item\textbf{Dodávka tepla}: dodávka tepelnej energie do Budovy pre potreby
vykurovania priestorov a prípravy teplej úžitkovej vody.

\item\textbf{Vykurovacia sezóna}: obdobie v roku, v ktorom je Dodávateľ
povinný dodržiavať garantované Štandardy komfortu stanovené v tejto
Zmluve, počnúc 1. októbrom a končiac 30. aprílom každého vyúčtovacieho
roka po Dobu poskytovania služieb.

\item\textbf{Faktúra}: faktúra vystavená Objednávateľovi (Vlastníkom bytu
alebo zástupcom Vlastníkov bytu) za prijaté Služby a ďalšie platby
splatné Dodávateľovi vyplývajúce zo Zmluvy v súlade so všetkými
príslušnými zákonnými požiadavkami vyplývajúcimi z právnych predpisov
platných v Slovenskej republike.

\item\textbf{IPMVP}: Medzinárodný protokol o meraní a overovaní výkonu
vypracovaný EVO (Efficiency Valuation Organisation), (1629 K Street NW,
Suite 300, Washington, DC 20006, USA), ktorý sa používa na účely merania
a overovania v rámci Zmluvy.

\item\textbf{(SK) BEEF}: Slovak Energy Efficiency Facility Jsc., akciová
spoločnosť riadne zapísaná v Obchodnom registri Slovenskej republiky pod
číslom \ldots\ldots\ldots\ldots\ldots{} .

\item\textbf{Latentný stav}: vady a~nedostatky Budovy alebo priestorov
susediacich s Budovou, o ktorých Objednávateľ nemal vedomosť a Dodávateľ
ich nemohol identifikovať pomocou primeraných pozorovaní a bežných
prehliadok v čase uzatvorenia Zmluvy.

\item\textbf{Správca}: fyzická alebo právnická osoba, ktorá v súlade s
platnými ustanoveniami zákona Slovenskej republiky o správcoch bytových
domov a na základe zmluvy o výkone správy vykonáva správcovské a
údržbárske činnosti požadované Objednávateľom a stanovené touto Zmluvou.

\item\textbf{Opatrenia} tiež označované ako Opatrenia energetickej účinnosti:
také opatrenia, ktoré vedú k dosiahnutiu overiteľného, merateľného alebo
odhadnuteľného zvýšenia energetickej hospodárnosti, ako aj iné stavebné
a inštalačné práce zamerané na modernizáciu a zhodnotenie Budovy
z~konštrukčných aj estetických dôvodov.

\item\textbf{Meranie a overovanie}: proces a činnosti vykonávané s cieľom
určiť Úspory energie pripadajúce na Budovu v dôsledku implementovaných
Opatrení a poskytovaných Služieb.

\item\textbf{Poplatok za prevádzku a údržbu}: poplatok zaplatený
Objednávateľom Dodávateľovi za Služby spojené s Prevádzkou a údržbou
Opatrení, ktorý podlieha ročnej indexácii v súlade so zákonom č. 18/1995
Z. z. o cenách v znení neskorších predpisov.

\item\textbf{Manuál prevádzky a údržby}: príručka obsahujúca harmonogram
údržby Opatrení implementovaných na základe tejto Zmluvy a prevádzkových
činností, na ktoré sa vzťahuje táto Zmluva.

\item\textbf{Zmluvné strany}: Objednávateľ a Dodávateľ spoločne.

\item\textbf{Zmluvná strana}: Objednávateľ a Dodávateľ samostatne.

\item\textbf{Harmonogram platieb}: dokument vypracovaný Dodávateľom pre
Objednávateľa, v ktorom je uvedený Poplatok za renováciu určený na
splácanie Finančného príspevku a vypočítaný pre každé obdobie výpočtu
úroku v súlade s touto Zmluvou.

\item\textbf{Riadne fungovanie}: fungovanie Opatrení takým spôsobom, aby sa
zabezpečilo dosiahnutie ich úplnej funkčnosti a efektívnosti, vrátane
všetkých potrebných údržbárskych činností vykonávaných Dodávateľom na
jeho náklady.

\item\textbf{Regulačný orgán}: Komisia pre verejné služby alebo iný príslušný
orgán ustanovený na základe zákonov a iných právnych predpisov platných
v Slovenskej republike, ktorý schvaľuje tarify za obchodovanie s
tepelnou energiou v samosprávnom celku, v~ktorom sa Budova nachádza.

\item\textbf{Poplatok za renováciu}: poplatok indexovaný na Euribor, ktorým
spláca Objednávateľ Dodávateľovi Finančný príspevok poskytnutý
Dodávateľom.

\item\textbf{Renovačné práce}: činnosti vykonávané Dodávateľom potrebné na
implementáciu Opatrení v Budove vrátane inžinieringu, obstarávania,
dodávok, inštalácie, spustenia, uvedenia do prevádzky a financovania
Opatrení.

\item\textbf{Doba poskytovania služieb}: obdobie, počas ktorého Dodávateľ
poskytuje Objednávateľovi Služby. Doba poskytovania služieb začína
Dátumom uvedenia do prevádzky.

\item\textbf{Vyúčtovacie obdobie}: obdobie jedného kalendárneho roka, ktoré
sa opakuje každoročne po Dobu poskytovania služieb.

\item\textbf{Vyhlásenie}: dokument podpísaný Zmluvnými stranami ako dôkaz
potvrdzujúci rozličné parametre prítomné v~Budove, zaznamenané v čase
podpísania takéhoto dokumentu.

\item\textbf{DPH}: daň z pridanej hodnoty splatná v súlade so zákonmi a inými
právnymi predpismi platnými v Slovenskej republike, ako aj s
ustanoveniami tejto Zmluvy.
\end{enumerate}
\subsection{AKCEPTÁCIA ZMLUVNÝCH PODMIENOK }

\begin{enumerate}
\def\labelenumi{\arabic{enumi}.}
\item
  Objednávateľ si je vedomý toho, že Dodávateľ má potrebnú kvalifikáciu,
  skúsenosti a schopnosti na vykonávanie Renovačných prác a poskytovanie
  Služieb pre Objednávateľa. Z tohto dôvodu Objednávateľ oprávňuje
  Dodávateľa a~zaväzuje sa poskytnúť Dodávateľovi všetku súčinnosť, aby
  na svoje náklady vykonal všetky právne a faktické kroky na plnenie
  Zmluvy bez potreby udelenia výslovného plnomocenstva v prospech
  Dodávateľa.
\item
  Dodávateľ sa zaväzuje vykonať Renovačné práce pre Objednávateľa a
  poskytovať Objednávateľovi Služby za podmienok stanovených v tejto
  Zmluve. Dodávateľ potvrdzuje, že sa presvedčil o povahe, stave a
  umiestnení Budovy, ako aj o všetkých ďalších záležitostiach, ktoré by
  mohli akýmkoľvek spôsobom ovplyvniť plnenie jeho povinností podľa
  Zmluvy. Akékoľvek neoboznámenie sa Dodávateľa s Budovou alebo s
  akýmkoľvek stavom Budovy a jej okolia podľa tohto bodu ho nezbavuje
  zodpovednosti za plnenie jeho povinností podľa tejto Zmluvy.
\item
  Zmluva potvrdzuje, že rozpočet zahrnutý v Osobitných podmienkach
  zahŕňa všetky stavebné práce, materiál a vybavenie, ktoré sú potrebné
  na vykonanie Renovačných prác v súlade s technickými špecifikáciami a
  podmienkami stanovenými touto Zmluvou.
\item
  Pojmy použité v tejto Zmluve, Osobitných podmienkach a ich Prílohách,
  ako aj v týchto Všeobecných podmienkach majú na všetky účely významy
  uvedené v článku 1 týchto Všeobecných podmienok.
\item
  V prípade nesúladu medzi Všeobecnými podmienkami a Osobitnými
  podmienkami a ich Prílohami majú prednosť ustanovenia Osobitných
  podmienok a ich Príloh.
\end{enumerate}

\subsection{BEZPEČNOSŤ, KVALITA A KOMFORT}

\begin{enumerate}
\def\labelenumi{\arabic{enumi}.}
\item
  Služby poskytované Dodávateľom podľa tejto Zmluvy musia:

  \begin{enumerate}
  \def\labelenumii{\arabic{enumii}.}
  \item
    byť dodávané s najvyššou úrovňou schopností a starostlivosti, aká sa
    očakáva od skúsených a profesionálnych dodávateľov pravidelne
    vykonávajúcich práce a služby rovnakého alebo podobného rozsahu a
    zložitosti ako sú práce a služby špecifikované v tejto Zmluve;
  \item
    byť poskytované s použitím materiálu a vybavenia vhodnej kvality,
    a~tiež nového a vhodného na daný účel;
  \item
    dodržiavať stavebné právne predpisy a všetky ďalšie príslušné
    zákonné pravidlá, nariadenia alebo normy účinné v Slovenskej
    republike v čase poskytovania Služieb;
  \item
    byť vykonávané tak, aby spôsobovali Objednávateľovi a ostatným
    užívateľom Budovy čo najmenšie ťažkosti pri užívaní Budovy.
  \end{enumerate}
\item
  Štandardy komfortu musia po Dobu poskytovania služieb na základe tejto
  Zmluvy spĺňať alebo prekračovať úroveň uvedenú v Osobitných
  podmienkach.
\item
  Dodávateľ negarantuje po dobu, počas ktorej sú okná v Byte v Budove
  otvorené, ako aj 2 hodiny po ich zatvorení, úroveň vnútornej teploty
  dohodnutú v Osobitných podmienkach pre konkrétny Byt, v ktorom boli
  okná otvorené.
\item
  Dodávateľ zabezpečí primeranú technickú požiadavku na vetranie v
  Bytoch podľa príslušných predpisov a noriem Slovenskej republiky.
\item
  Dodávateľ prijme všetky potrebné opatrenia na zaistenie bezpečnosti a
  ochrany zdravia zamestnancov pri práci v súlade so zákonom č. 311/2001
  Z. z., Zákonník práce, v znení neskorších predpisov, zákonom č.
  124/2006 Z. z. o bezpečnosti a ochrane zdravia pri práci a všetkými
  príslušnými predpismi a normami platnými v Slovenskej republike.
\item
  Dodávateľ zavedie vhodné opatrenia na ochranu všetkých osôb pred
  usmrtením alebo úrazom, ktoré môžu byť spôsobené zlyhaním alebo hrubou
  nedbanlivosťou Dodávateľa, jeho zamestnancov, zástupcov alebo
  subdodávateľov počas Obdobia výstavby a Doby poskytovania služieb.
  Dodávateľ je taktiež povinný chrániť celú Budovu pred vznikom škôd
  súvisiacich s implementáciou Opatrení.
\item
  Dodávateľ zabezpečí, aby neboli žiadne inžinierske siete poskytované
  Budove kedykoľvek bez predchádzajúceho upozornenia odpojené alebo
  prerušené z dôvodu zlyhania alebo nedbanlivosti Dodávateľa. Akékoľvek
  inžinierske siete prerušené alebo odpojené z dôvodu zlyhania alebo
  nedbanlivosti Dodávateľa musia byť Dodávateľom okamžite obnovené na
  náklady Dodávateľa. Dodávateľ nenesie zodpovednosť za prípady, keď sú
  tieto prerušenia mimo kontroly Dodávateľa a/alebo sú dôsledkom konania
  alebo opomenutia zo strany spoločností vykonávajúcich ich údržbu,
  energetických a vodovodných spoločností alebo akýchkoľvek tretích
  strán, ktoré s ním nie sú v zmluvnom vzťahu.
\item
  Dodávateľ počas Obdobia výstavby zabezpečí náležitú ochranu Budovy
  pred vplyvmi počasia, aby zabránil presakovaniu zrážkovej vody a
  poškodeniu Budovy. Presakovanie podzemných vôd a Zásah vyššej moci sú
  vylúčené.
\item
  Dodávateľ sa riadi Európskym kódexom správania pre uzatváranie Zmlúv o
  poskytovaní energetických služieb (www. http://transparense.eu/),
  ktorý predstavuje súbor hodnôt a princípov považovaných za základné
  pre úspešné, profesionálne a transparentné plnenie Zmlúv o poskytovaní
  energetických služieb v európskych krajinách.
\end{enumerate}

\subsection{Garancie}

\begin{enumerate}
\def\labelenumi{\arabic{enumi}.}
\item
  Dodávateľ je po Dobu poskytovania služieb povinný poskytnúť
  Objednávateľovi Garantovanú úsporu energie ako súčasť tejto Zmluvy,
  ktorá bude podliehať každoročnému Meraniu a overovaniu.
\item
  Dodávateľ je po Dobu poskytovania služieb povinný garantovať Štandardy
  komfortu stanovené touto Zmluvou.
\item
  Dodávateľ zabezpečí po Dobu poskytovania služieb na svoje vlastné
  náklady Riadne fungovanie Opatrení inštalovaných alebo zavedených
  Dodávateľom pre vykurovací systém, systém zásobovania domácností
  teplou vodou, systém ventilácie a ochladzovania vzduchu, ako aj uzly a
  potrubia v súlade s ich špecifikáciami a bežným opotrebením počas
  celej doby platnosti tejto Zmluvy a, okrem iného, v prípade potreby aj
  opravu alebo výmenu týchto Opatrení.
\item
  Dodávateľ je povinný po Dobu poskytovania služieb na svoje náklady
  garantovať efektivitu a účinnosť izolačných materiálov nainštalovaných
  alebo použitých Dodávateľom v súlade s ich špecifikáciami a obvyklým
  opotrebením počas celej doby platnosti tejto Zmluvy a, okrem iného, v
  prípade potreby aj ich opravu alebo výmenu.
\item
  Dodávateľ zabezpečí na konci Doby poskytovania služieb Riadne
  fungovanie všetkých implementovaných Opatrení v súlade s ich
  špecifikáciami a obvyklým opotrebením, so zreteľom na riadnu údržbu.
  Dodávateľ poskytne Objednávateľovi na konci Doby poskytovania služieb
  všetky príručky na používanie, starostlivosť a údržbu, záznamy,
  pokyny, ďalšiu dokumentáciu, softvér, licencie na práva duševného
  vlastníctva, špeciálne nástroje a protokoly a postupy vhodné alebo
  potrebné na nepretržitý riadny výkon Opatrení zavedených na
  dosiahnutie Štandardov komfortu v súlade s touto Zmluvou.
\item
  Dodávateľ predloží Objednávateľovi pred začatím Obdobia výstavby
  záruku za plnenie vystavenú úverovou inštitúciou alebo poisťovňou,
  ktorou zaručí splnenie svojich povinností, a to vo výške 10\% z
  celkových Investičných nákladov (bez DPH) nasledovne:

  \begin{enumerate}
  \def\labelenumii{\arabic{enumii}.}
  \item
    ak je Dodávateľ všeobecnou stavebnou spoločnosťou, poskytuje takúto
    záruku za plnenie Dodávateľ v prospech Objednávateľa v súlade s
    ustanoveniami tejto Zmluvy;
  \item
    ak si Dodávateľ obstará všeobecnú stavebnú spoločnosť, takúto záruku
    za plnenie poskytne všeobecná stavebná spoločnosť v prospech
    Dodávateľa na základe uzatvorenej zmluvy o dielo na zhotovenie
    stavebných prác medzi Dodávateľom a všeobecnou stavebnou
    spoločnosťou;
  \item
    v prípade, že Dodávateľ neposkytne počas Obdobia výstavby originál
    takejto záruky za plnenie, ktorou zaručí vykonávanie činností počas
    Obdobia výstavby, Dodávateľ nie je oprávnený začať stavebné práce;
  \item
    záruka za plnenie musí byť platná počas celého Obdobia výstavby. V
    prípade predĺženia Obdobia výstavby Dodávateľ predĺži túto záruku o
    rovnaké časové obdobie.
  \end{enumerate}
\item
  Dodávateľ predloží Objednávateľovi najneskôr 10 dní od podpísania
  Odovzdávacieho a preberacieho protokolu záruku za plnenie vystavenú
  úverovou inštitúciou alebo poisťovňou, ktorou zaručí splnenie svojich
  povinností, a to vo výške najmenej 5\% z celkových Investičných
  nákladov (bez DPH) nasledovne:

  \begin{enumerate}
  \def\labelenumii{\arabic{enumii}.}
  \item
    ak je Dodávateľ všeobecnou stavebnou spoločnosťou, poskytuje takúto
    záruku za plnenie Dodávateľ v prospech Objednávateľa v súlade s
    ustanoveniami tejto Zmluvy;
  \item
    ak si Dodávateľ obstará všeobecnú stavebnú spoločnosť, takúto záruku
    za plnenie poskytuje všeobecná stavebná spoločnosť v prospech
    Dodávateľa na základe uzatvorenej zmluvy o dielo na zhotovenie
    stavebných prác medzi Dodávateľom a všeobecnou stavebnou
    spoločnosťou;
  \item
    záruka musí byť platná 36 mesiacov.
  \end{enumerate}
\item
  Objednávateľ má právo použiť záruku za plnenie uvedenú v bodoch 4.6. a
  4.7. na účely vyrovnania finančných záväzkov Dodávateľa alebo
  regulačných opatrení.
\item
  Záruka za plnenie uvedená v tomto článku musí byť vystavená úverovou
  inštitúciou alebo poisťovňou registrovanou v Slovenskej republike
  alebo v ktoromkoľvek inom členskom štáte Európskej únie alebo
  Európskeho hospodárskeho priestoru, ktorá poskytuje služby na území
  Slovenskej republiky v súlade s postupmi stanovenými právnymi aktmi
  Slovenskej republiky.
\end{enumerate}

\subsection{PRÁVA A POVINNOSTI DODÁVATEĽA}

\begin{enumerate}
\def\labelenumi{\arabic{enumi}.}
\item
  Dodávateľ má potrebnú odbornú kvalifikáciu, skúsenosti a schopnosti v
  oblasti dodávok vybavenia, materiálu a služieb podľa tejto Zmluvy.
\item
  Dodávateľ je povinný zabezpečiť si všetky potrebné povolenia a súhlasy
  od orgánov štátnej správy, mestských inštitúcií a~iných úradov na
  vykonávanie Renovačných prác a poskytovanie Služieb. Objednávateľ je
  povinný poskytnúť Dodávateľovi všetku súčinnosť, vynaložiť všetko
  úsilie na splnenie povinností Dodávateľa a udeliť Dodávateľovi
  požadované plnomocenstvo v súlade s bodom 6.4 týchto Podmienok.
\item
  Dodávateľ začne s realizáciou stavebných a inštalačných prác
  súvisiacich s Opatreniami v Dátum začatia a ukončí ich v stanovenom
  Období výstavby. Dodávateľ informuje Objednávateľa o predbežnom Dátume
  začatia, a to najneskôr do 20 Pracovných dní po podpísaní tejto
  Zmluvy.
\item
  Dodávateľ písomne informuje Objednávateľa najmenej 10 Pracovných dní
  vopred o Dátume začatia stavebných a inštalačných prác súvisiacich s
  Opatreniami, aby umožnil Objednávateľovi odstrániť odpad, opustený
  majetok a akékoľvek iné predmety zo spoločných priestorov Budovy
  (vrátane schodísk, suterénnych priestorov, podkrovia, strechy,
  priestorov na skladovanie uhlia / dreva a plynu, elektrických a
  telekomunikačných panelov a miestností kotolní). Ak Objednávateľ
  nevyprázdni spoločné priestory včas, má Dodávateľ právo zabezpečiť
  vyprázdnenie spoločných priestorov Budovy a vystaviť Objednávateľovi
  faktúru na úhradu týchto prác ako náhradu vzniknutých výdavkov.
  Objednávateľ je povinný uhradiť takúto faktúru bezodkladne, najneskôr
  do 20 Pracovných dní.
\item
  Počas Obdobia výstavby je Dodávateľ povinný zabezpečiť všetku pracovnú
  silu potrebnú na implementáciu Opatrení, vrátane nevyhnutného dohľadu,
  nástrojov, materiálov a vybavenia vhodnej povahy, kvality a množstva.
\item
  Po Dobu poskytovania služieb je Dodávateľ povinný zabezpečiť všetku
  pracovnú silu potrebnú na prevádzku a údržbu Opatrení, vrátane
  nevyhnutného dohľadu, nástrojov, materiálov a vybavenia vhodnej
  povahy, kvality a množstva.
\item
  Počas Obdobia výstavby Dodávateľ zabezpečí dodávku elektriny so
  samostatným meraním a uhradí elektrinu spotrebovanú na realizáciu a
  inštaláciu Opatrení, ak je to možné. Dodávateľ má právo na prístup do
  spoločného systému dodávky elektrickej energie v Budove.
\item
  Dodávateľ po ukončení stavebných a inštalačných prác Opatrení pred
  Dátumom uvedenia do prevádzky riadne vyčistí stavenisko (spoločné
  priestory Budovy, okná, vchody a okolie).
\item
  Dodávateľ vyzve Objednávateľa na uvedenie Opatrení realizovaných v
  Budove do prevádzky. Dodávateľ pripraví pre Objednávateľa na konci
  Obdobia výstavby protokol o odovzdaní a prevzatí Budovy (Odovzdávací a
  preberací protokol).
\item
  Dodávateľ počas Doby poskytovania služieb písomne informuje
  Objednávateľa v prípade, že Vlastníci bytu alebo iné tretie strany
  nesúvisiace s Dodávateľom uskladnili a ponechali v spoločných
  priestoroch Budovy odpad a/alebo predmety, ktoré by mohli spôsobiť
  Dodávateľovi problémy vo vykonávaní činností spojených s prevádzkou a
  údržbou podľa Zmluvy. Ak Objednávateľ nevyprázdni tieto priestory v
  súlade s touto Zmluvou, Dodávateľ má právo zabezpečiť vyprázdnenie
  priestorov a vystaviť Objednávateľovi faktúru ako náhradu za vykonanie
  takéhoto vyprázdnenia. Objednávateľ je povinný uhradiť takúto faktúru
  bezodkladne, najneskôr do 20 Pracovných dní.
\item
  Dodávateľ počas Doby poskytovania služieb písomne oznámi
  Objednávateľovi každú zistenú vadu, krádež, vandalizmus alebo sabotáž
  týkajúcu sa Opatrení.
\item
  Dodávateľ zabezpečí dostatočné Parametre tepelnej energie počas
  Vykurovacej sezóny spôsobom vhodným na splnenie Štandardov komfortu
  podľa tejto Zmluvy. Dodávateľ je povinný poskytnúť Objednávateľovi
  príslušné požiadavky na vykurovanie a akúkoľvek s tým súvisiacu
  súčinnosť. Dodávateľ na základe tejto Zmluvy nezodpovedá za prerušenie
  alebo nedostatočné Dodávky tepelnej energie do Budovy v prípadoch,
  ktoré sú mimo kontrolu Dodávateľa, vrátane zlyhania dodávok tepelnej
  energie zo strany dodávateľa tepelnej energie alebo z dôvodu Zásahu
  vyššej moci.
\item
  Dodávateľ predloží Objednávateľovi pred začiatkom Obdobia výstavby
  projekt renovácie Budovy v súlade so zákonom č. 321/2014 Z. z. o
  energetickej efektívnosti a všetkými príslušnými normami a
  nariadeniami platnými v Slovenskej republike, pričom je povinný
  zosúladiť ho s požiadavkami Objednávateľa a stavebného dozoru.
\item
  Dodávateľ je počas Obdobia výstavby povinný informovať a vyzývať
  Objednávateľa na týždenné monitorovacie stretnutia týkajúce sa stavu
  stavebných prác. Dodávateľ následne predloží Objednávateľovi každý
  mesiac 2 kópie správy o postupe a stave stavebných prác. Tieto správy
  môžu byť predložené elektronicky s využitím Platformy pre Zmluvy o
  poskytovaní energetických služieb
  -\href{http://www.sharex.lv}{http://sunshineplatform.eu/}.
\item
  V prípade, že je Dodávateľ sprostredkovateľom medzi Objednávateľom a
  dodávateľom tepelnej energie, je Dodávateľ povinný zaplatiť v mene
  Objednávateľa faktúry splatné teplárenskej spoločnosti, a to po
  prijatí zodpovedajúcej zložky Poplatku za tepelnú energiu splatnej
  Dodávateľovi na základe tejto Zmluvy. Bez ohľadu na vyššie uvedené,
  faktúry za kúrenie počas Obdobia výstavby je povinný uhrádzať
  Objednávateľ.
\item
  Dodávateľ má právo postúpiť alebo zadať výkon prác a služieb
  stanovených v Zmluve tretím stranám (subdodávateľom). Dodávateľ v
  plnom rozsahu zodpovedá za subdodávateľov, pokiaľ ide o povinnosti
  vyplývajúce z tejto Zmluvy.
\item
  Dodávateľ má právo upraviť Garantovanú úsporu energie v prípade zmeny
  v užívaní Budovy (článok 17). Akákoľvek zmena musí byť dohodnutá
  Zmluvnými stranami prostredníctvom písomného dodatku k tejto Zmluve.
\item
  Dodávateľ má právo poveriť na svoje náklady riadne kvalifikovaného a
  skúseného nezávislého odborníka, o ktorom Objednávateľa písomne
  informoval (Objednávateľ je oprávnený namietať proti osobe odborníka
  iba v prípade, že by nekonal s náležitou starostlivosťou), aby
  vyhodnotil súlad navrhovaných Opatrení s právnymi predpismi platnými v
  Slovenskej republike alebo rozhodnutiami miestnych samospráv, ktoré sú
  pre Objednávateľa záväzné v prípade, že si Objednávateľ uplatní svoje
  právo veta. Stanovisko tohto odborníka je pre Zmluvné strany záväzné.
\item
  Dodávateľ je povinný najneskôr do 5 Pracovných dní informovať
  Objednávateľa o zmene adresy uvedenej v Zmluve alebo o iných zmenách v
  jeho právnej forme, riadení a~právnom postavení, a to najmä ak je
  Dodávateľ predmetom akéhokoľvek zlúčenia alebo akvizície, likvidácie
  alebo konkurzného konania.
\item
  Prevádzka a údržba energetického zariadenia vrátane zaškolenia
  používateľov, monitorovania a prevádzky systému musí byť vykonávaná v
  súlade s § 17 ods. 3 zákona č. 321/2014 Z. z. o energetickej
  efektívnosti a všetkými príslušnými normami a nariadeniami platnými v
  Slovenskej republike uvedenými v Osobitných podmienkach a Zmluve o
  poskytovaní energetických služieb, ktoré tvoria prílohu k týmto
  Všeobecným podmienkam.
\end{enumerate}

\subsection{PRÁVA A POVINNOSTI OBJEDNÁVATEĽA}

\begin{enumerate}
\def\labelenumi{\arabic{enumi}.}
\item
  Objednávateľ prijal platné a vykonateľné rozhodnutie, ktorým sa táto
  Zmluva stáva záväznou pre všetkých Vlastníkov bytov v súlade so
  zákonom č. 182/1993 Z. z. o vlastníctve bytov a nebytových priestorov
  v znení neskorších predpisov, ktorí sú každý jednotlivo povinní
  dodržiavať jej ustanovenia bez ohľadu na to, či je daný Byt prenajatý
  a či sú Byty využívané skutočnými Vlastníkmi bytov alebo nie.
\item
  Objednávateľ (každý Vlastník bytu) je povinný informovať nájomcov a
  všetkých ostatných užívateľov Bytu alebo bežných užívateľov o
  príslušných povinnostiach vyplývajúcich z tejto Zmluvy.
\item
  Objednávateľ je povinný čo najskôr po obdržaní písomnej žiadosti
  Dodávateľa poskytnúť informácie a dokumenty požadované Dodávateľom na
  realizáciu Renovačných prác a na poskytovanie Služieb uvedených v
  Zmluve. Objednávateľ nenesie zodpovednosť za neposkytnutie akýchkoľvek
  dokumentov, ktoré by mohli byť relevantné, ale neboli dostatočne jasne
  špecifikované Dodávateľom.
\item
  Objednávateľ je povinný poskytnúť Dodávateľovi včasnú súčinnosť pri
  zabezpečení potrebných povolení, schválení alebo iných dokumentov
  týkajúcich sa úspešného plnenia Zmluvy od orgánov štátnej správy,
  mestských inštitúcií a agentúr, okrem iného vrátane osvedčenia a/alebo
  poskytnutia potrebných dokumentov, udeliť Dodávateľovi požadované
  plnomocenstvo a poskytnúť mu dostupné informácie. Na účely úspešného
  plnenia Zmluvy Objednávateľ riadne splnomocní Dodávateľa v požadovanej
  forme na uskutočnenie akýchkoľvek faktických alebo právnych úkonov vo
  vzťahu k príslušným orgánom. Objednávateľ však nenesie zodpovednosť za
  neposkytnutie akýchkoľvek a všetkých takýchto informácií, pokiaľ ich
  Dodávateľ jasne nešpecifikuje a pokiaľ tieto informácie nemá
  Objednávateľ primerane k dispozícii. Dodávateľ je povinný bez
  zbytočného odkladu informovať Objednávateľa o~skutočnosti, že mu
  neposkytol požadované informácie alebo potrebnú súčinnosť. Dodávateľ
  sa nebude považovať za Zmluvnú stranu, ktorá je v omeškaní s plnením
  svojich povinností, ak mu Objednávateľ neposkytol potrebnú súčinnosť.
\item
  Objednávateľ nebude zdržiavať, ani neodmietne udeliť súhlas s
  Renovačnými prácami, implementáciou Opatrení počas Obdobia výstavby a
  ich údržbou počas Doby poskytovania služieb; naopak, Objednávateľ bude
  konať v dobrej viere, aby uľahčil ich implementáciu a údržbu, ako aj
  dosiahnutie Garantovanej úspory energie.
\item
  Objednávateľ má právo do 10 Pracovných dní po podpísaní Odovzdávacieho
  a preberacieho protokolu reklamovať kvalitu alebo spôsob vykonania
  implementovaného Opatrenia. Po tejto lehote sa všetky Opatrenia
  implementované Dodávateľom budú považovať za akceptované a Obdobie
  výstavby za ukončené.
\item
  Objednávateľ je oprávnený uplatniť právo veta alebo odmietnuť
  implementáciu Opatrenia plánovaného v rámci Renovačných prác, ak bez
  akýchkoľvek pochybností preukáže, že príslušné Opatrenie porušuje
  zákony a nariadenia platné v Slovenskej republike alebo rozhodnutia
  miestnych samospráv záväzné pre Objednávateľa.
\item
  Objednávateľ poskytne Dodávateľovi alebo akejkoľvek inej osobe
  poverenej Dodávateľom počas Obdobia výstavby a počas Doby poskytovania
  služieb prístup do Budovy vrátane každého Bytu v rámci Budovy na účely
  poskytovania Služieb podľa Zmluvy. Objednávateľ zabezpečí prístup do
  Budovy kedykoľvek počas Pracovných dní (medzi 8:00 a 20:00) a v
  mimoriadnych situáciách i mimo pracovnú dobu a tiež cez víkendy a
  sviatky.
\item
  Objednávateľ pred Dátumom začatia zabezpečí, aby spoločné priestory
  (vrátane schodísk, suterénnych priestorov, podkrovia, strechy,
  skladovacích priestorov na uhlie / drevo a plyn, elektrických a
  telekomunikačných panelov a miestností kotolní) neobsahovali odpad,
  opustený majetok a akékoľvek predmety, ktoré sa tam nachádzajú, a to
  tak, že sa dohodne s Dodávateľom na ich odstránení alebo odvoze do
  zberného dvora alebo doručení známemu vlastníkovi.
\item
  Objednávateľ počas Doby poskytovania služieb zabezpečí, aby spoločné
  priestory Budovy, ako sú schody, suterénne priestory a podkrovné
  priestory, boli udržiavané čisté a v dobrom prevádzkovom stave, a to
  vykonávaním pravidelného upratovania a čistenia.
\item
  Objednávateľ nebude zasahovať do Opatrení inštalovaných Dodávateľom
  bez písomného súhlasu a oprávnenia Dodávateľa alebo v rozpore s
  prevádzkovými pokynmi poskytnutými Dodávateľom, najmä ak by mal takýto
  zásah negatívny vplyv na úroveň Úspor energie. Zásahy Objednávateľa do
  nastavenia vykurovacieho systému, systému teplej úžitkovej vody a
  ventilačného systému sa budú považovať za podstatné porušenie
  povinností Objednávateľa vyplývajúcich z tejto Zmluvy a budú slúžiť
  ako právoplatný a dostatočný dôvod na Ukončenie Zmluvy zo strany
  Dodávateľa.
\item
  Objednávateľ prijme všetky primerané opatrenia s cieľom zabezpečiť,
  aby nikto v Budove nezasahoval, ani nemanipuloval s nastavením
  vykurovacieho systému, systému teplej úžitkovej vody a ventilačného
  systému, ani akýmkoľvek spôsobom nepoškodzoval a nesabotoval
  Opatrenia.
\item
  Objednávateľ je povinný bezodkladne po zistení (do 1 Pracovného dňa)
  písomne oznámiť Dodávateľovi akékoľvek poškodenie, zmenu alebo zásah
  do Opatrenia inštalovaného Dodávateľom.
\item
  Objednávateľ je povinný písomne informovať o všetkých okolnostiach,
  ktoré majú alebo by mohli mať negatívny vplyv na Úsporu energie. Ak
  Objednávateľ neinformuje Dodávateľa tak, ako je uvedené vyššie, táto
  skutočnosť nezbavuje Dodávateľa zodpovednosti za dosiahnutie
  Garantovanej úspory energie, pokiaľ sa nezistí, že Objednávateľ mal v
  úmysle znížiť úroveň Úspory energie.
\item
  Objednávateľ je povinný najneskôr do 20 Pracovných dní písomne oznámiť
  Dodávateľovi a v prípade potreby s ním koordinovať vykonanie
  stavebných, inštalačných a údržbárskych prác, ktoré nie sú súčasťou
  tejto Zmluvy a majú potenciálny vplyv na spotrebu energie v Budove,
  okrem iného vrátane (i) zväčšenia plochy Budovy, (ii) ďalšej
  modernizácie Budovy, (iii) výmeny alebo inštalácie nových / iných
  radiátorov a/alebo tepelných konvektorov a/alebo vykurovacích telies a
  (iv) inštalácie novej jednotky na výrobu tepla. V týchto prípadoch je
  Dodávateľ oprávnený revidovať Garantovanú úsporu energie stanovenú v
  Zmluve.
\item
  Objednávateľ (Vlastník bytu) je povinný informovať Dodávateľa o
  prípadnej renovácii Bytu a časovom harmonograme, ak by takáto
  renovácia mohla mať vplyv na energetickú náročnosť Budovy, okrem iného
  o (i) výmene vykurovacích telies a/alebo tepelných konvektorov a/alebo
  vykurovacích telies, (ii) výmene okien, (iii) zväčšení vykurovanej
  plochy bytu vrátane plochy balkóna / lodžie a (iv) inštalácii systémov
  mechanického vetrania. V týchto prípadoch je Dodávateľ oprávnený
  revidovať Garantovanú úspory energie stanovenú v Zmluve.
\item
  Objednávateľ je počas Vykurovacej sezóny oprávnený otvárať okná v
  Bytoch Budovy najviac na 10 minút denne s cieľom zabezpečiť cirkuláciu
  čerstvého vzduchu, aby sa zbavil (i) prachu alebo silného zápachu
  čistiacich prostriedkov, ktorý v Byte pretrváva po upratovaní, a (ii)
  silného zápachu, ktorý v~Byte pretrváva po varení.
\item
  Objednávateľ je počas Vykurovacej sezóny oprávnený zo zdravotných
  dôvodov osôb nachádzajúcich sa v Byte otvárať okná Bytu v Budove
  kedykoľvek.
\item
  Objednávateľ zabezpečí, aby boli počas Vykurovacej sezóny všetky okná
  spoločných priestorov zatvorené.
\item
  Objednávateľ zabezpečí, aby počas Vykurovacej sezóny neboli ponechané
  otvorené žiadne vchodové dvere Budovy.
\item
  Ak sa zmenia Vlastníci bytov v Budove, Objednávateľ je povinný písomne
  oznámiť takúto zmenu Dodávateľovi, a to bezodkladne, najneskôr však do
  5 Pracovných dní po takejto zmene.
\item
  Objednávateľ (každý Vlastník bytu) zabezpečí, aby v prípade
  akéhokoľvek prevodu vlastníckych práv k jeho Bytu a bez ohľadu na to,
  z akých dôvodov alebo na akom právnom základe sa tento prevod
  vykonáva, podpísal nový Vlastník bytu (nadobúdateľ) Listinu o
  pristúpení alebo akýkoľvek iný právny dokument potvrdzujúci prevzatie
  práv a povinností pôvodného Vlastníka bytu vyplývajúcich z tejto
  Zmluvy novým Vlastníkom bytu. Nedodržanie vyššie uvedeného bude mať za
  následok, že pôvodný Vlastník bytu spoločne s Objednávateľom sú
  zodpovední za splnenie akýchkoľvek povinností stanovených touto
  Zmluvou, ako aj za akékoľvek ich prípadné porušenie zo strany
  nadobúdateľa. Pôvodný Vlastník bytu spoločne s Objednávateľom sú
  zodpovední za všetky škody, ktoré vzniknú porušením povinnosti v
  súlade s týmto bodom Podmienok.
\item
  Objednávateľ je povinný najneskôr do 5 Pracovných dní písomne
  informovať Dodávateľa o zmene Správcu. Nový Správca je povinný riadne
  sa oboznámiť s ustanoveniami tejto Zmluvy.
\end{enumerate}

\subsection{POSTUP PRI VYÚČTOVANÍ}
\begin{enumerate}
\def\labelenumi{\arabic{enumi}.}
\item
  Objednávateľ je povinný platiť Dodávateľovi mesačné poplatky uvedené v
  tejto Zmluve.
\item
  Lehota na vzájomné vysporiadanie úhrad medzi Zmluvnými stranami je
  jeden kalendárny mesiac. Vyúčtovacie obdobie pre výpočet Rozdielu
  medzi úhradami Faktúr vystavovaných na základe paušálnej Spotreby
  tepelnej energie a skutočne nameranou Spotrebou tepelnej energie a
  výsledkami Merania a overenia Garantovanej úspory energie je jeden
  rok.
\item
  Dodávateľ alebo ním poverená tretia strana zastupujúca Dodávateľa
  vypočíta každý mesiac sumu, ktorú má uhradiť Objednávateľ na základe
  Zmluvy. Celková suma všetkých vypočítaných Poplatkov splatných
  Objednávateľom Dodávateľovi za Služby poskytnuté podľa tejto Zmluvy sa
  považuje za vzájomné vysporiadanie úhrad.
\item
  Každých 12 mesiacov, vždy k~dátumu výročia začatia Doby poskytovania
  služieb, vykoná Dodávateľ na základe výsledkov Merania a overenia
  Garantovanej úspory energie ročné zúčtovanie.
\item
  Dodávateľ alebo ním poverená tretia strana vystaví najneskôr do 10.
  dňa každého mesiaca Faktúru s podrobnými údajmi o všetkých zložkách
  Poplatkov výslovne uvedených v Zmluve a doručí ju Objednávateľovi
  alebo Správcovi zastupujúcemu Objednávateľa. Splatnosť Faktúry je 30
  dní.
\item
  Dodávateľ zabezpečí, aby Faktúry pre každého Vlastníka bytu za
  poskytnuté Služby obsahovali jasné a komplexné informácie a aby v nich
  boli uvedené samostatne Poplatok za renováciu a Poplatok za prevádzku
  a údržbu splatné Dodávateľovi.
\item
  Prvá úhrada Poplatkov sa vykoná 1 mesiac po podpísaní Odovzdávacieho a
  preberacieho protokolu. Dovtedy je Objednávateľ povinný uhrádzať
  všetky svoje existujúce náklady na energie a komunálne výdavky v
  lehote ich splatnosti.
\item
  Objednávateľ (každý Vlastník bytu) zaplatí Poplatky Dodávateľovi
  (alebo tretej osobe určenej Dodávateľom) priamo alebo prostredníctvom
  urýchľovacej platby Dodávateľovi alebo ním určenému Správcovi na
  základe Faktúr vystavených Dodávateľom alebo ním určeným Správcom za
  všetky inžinierske siete a ďalšie prevádzkové náklady na údržbu
  Budovy, ktoré zahŕňajú Poplatky splatné Dodávateľovi. Objednávateľ
  uhradí Poplatky podľa zaužívaných postupov Dodávateľa alebo ním
  určeného Správcu, najneskôr však do 15 dní odo dňa prijatia Faktúry, a
  to prevodom potrebných finančných prostriedkov na bankový účet určený
  Dodávateľom alebo ním určeného Správcu.
\item
  Dodávateľ alebo ním určený Správca spravuje informácie spojené s
  vyúčtovaním vyplývajúcim z tejto Zmluvy, okrem iného prostredníctvom:

  \begin{enumerate}
  \def\labelenumii{\arabic{enumii}.}
  \item
    zaznamenávania všetkých informácií o Faktúrach vystavených každému
    Vlastníkovi bytu, vrátane ich súm;
  \item
    vedenia záznamov o úhradách Faktúr a neustálej aktualizácie výšky
    dlhu každého Vlastníka bytu, ak taký dlh existuje.
  \end{enumerate}
\item
  Objednávateľ na požiadanie Dodávateľa alebo niektorého z jeho
  postupníkov poskytne Dodávateľovi alebo takémuto postupníkovi (podľa
  situácie) aktuálny prehľad úhrad vykonaných Vlastníkmi bytov v Budove
  a zoznam dlžníkov.
\end{enumerate}

\subsection{DOBA PLATNOSTI ZMLUVY}

\begin{enumerate}
\def\labelenumi{\arabic{enumi}.}
\item
  Doba platnosti tejto Zmluvy začína dátumom jej uzatvorenia, pričom
  zostáva plne platná a účinná až do konca ukončenia Doby poskytovania
  služieb, s výhradou predčasného ukončenia špecifikovaného v tejto
  Zmluve.
\item
  Dobu platnosti tejto Zmluvy možno predĺžiť na základe písomného
  súhlasu Zmluvných strán prostredníctvom písomných Dodatkov k tejto
  Zmluve. Zmluvné strany majú predovšetkým právo predvídať alebo odsunúť
  Dátum začatia a Dátum uvedenia do prevádzky na základe písomného
  súhlasu Zmluvných strán prostredníctvom písomných Dodatkov k tejto
  Zmluve.
\item
  Dodávateľ začne s realizáciou stavebných a inštalačných prác
  súvisiacich s Opatreniami v Dátum začatia, pričom ich ukončí v
  stanovenom Období výstavby. Na konci Obdobia výstavby sú Objednávateľ
  a Dodávateľ povinní podpísať Odovzdávací a preberací protokol.
\item
  Doba poskytovania služieb a platobné podmienky začínajú platiť
  podpísaním Odovzdávacieho a preberacieho protokolu.
\item
  V prípade neplnenia alebo nedbanlivosti Objednávateľa, napríklad ak
  Dodávateľovi nie sú poskytnuté všetky náležité dokumenty a/alebo
  prístup do Budovy a/alebo v prípade Zásahu vyššej moci, nie je
  Dodávateľ v omeškaní a Dátum začatia, Obdobie výstavby a Doba
  poskytovania služieb sa považujú za automaticky odsunuté alebo
  predĺžené o dobu omeškania Objednávateľa.
\end{enumerate}

\subsection{LATENTNÝ STAV}

\begin{enumerate}
\def\labelenumi{\arabic{enumi}.}
\item
  Dodávateľ môže požadovať adekvátne zvýšenie Poplatkov vyplývajúcich zo
  Zmluvy o poskytovaní energetických služieb, ak by počas realizácie
  prác vznikla potreba vykonať akúkoľvek činnosť, ktorá nie je zahrnutá
  v rozpočte, za predpokladu, že v čase uzatvorenia Zmluvy o poskytovaní
  energetických služieb nebolo možné tieto činnosti predpokladať (ďalej
  len „Latentný stav``).
\end{enumerate}

9.2. Objednávateľ je oprávnený odstúpiť od Zmluvy s okamžitou
platnosťou, ak Dodávateľ požaduje zvýšenie Poplatkov stanovených Zmluvou
o poskytovaní energetických služieb o viac ako 10\% z rozpočtovanej
ceny. V takom prípade je Objednávateľ povinný uhradiť Dodávateľovi časť
ceny zodpovedajúcu čiastkovému vykonaniu prác podľa rozpočtu.

9.3. Omeškanie spôsobené Latentným stavom môže byť dôvodom na predĺženie
Obdobia výstavby, ak v dôsledku Latentného stavu Dodávateľ musí:

\begin{itemize}
\item
  vykonať ďalšie práce;
\item
  použiť ďalšie materiály; alebo
\item
  vynaložiť ďalšie náklady (vrátane, ale nielen, nákladov na omeškanie
  alebo prerušenie prevádzky), ktoré Dodávateľ v čase uzatvorenia Zmluvy
  primerane nepredpokladal a nemohol predpokladať ani pri uplatnení
  náležitej odbornej starostlivosti a osvedčených priemyselných
  postupov.
\end{itemize}

9.4. Objednávateľ hradí všetky skutočné výdavky vzniknuté v súvislosti s
Latentným stavom a odsúhlasené Zmluvnými stranami. Ak si Objednávateľ
neželá, aby Dodávateľ postupoval podľa oznámenia, musí ho bezodkladne
informovať, aby nepokračoval, pričom Dodávateľ musí takejto požiadavke
vyhovieť. Objednávateľ a Dodávateľ môžu rokovať a dohodnúť sa na nejakom
inom spôsobe prekonania Latentného stavu, okrem iného vrátane vykonania
ďalších nevyhnutných prác inými subjektmi a ich úhrady Objednávateľom.

\subsection{MERANIE A OVEROVANIE A SPRÁVA ÚDAJOV}

\begin{enumerate}
\def\labelenumi{\arabic{enumi}.}
\item
  Dodávateľ musí vykonať všetky činnosti spojené s Meraním a overovaním
  v súlade s platným plánom Merania a overovania vypracovaným na základe
  IPMVP (Medzinárodného protokolu o meraní a overovaní výkonu)
  dostupného na Platforme pre Zmluvy o poskytovaní energetických služieb
  -- \href{http://www.sharex.lv}{http://sunshineplatform.eu/}.
\item
  Všetky činnosti týkajúce sa Merania a overovania musia byť jasne a
  úplne zverejnené a transparentné pre všetky Zmluvné strany.
\item
  Dodávateľ je počas Doby poskytovania služieb povinný predkladať
  Objednávateľovi Výročnú správu. Táto Výročná správa musí obsahovať
  výpočet Poplatkov, Činnosti spojené s prevádzkou a údržbou vykonávané
  Dodávateľom a informáciu o tom, či bola na základe Merania a
  overovania počas Vyúčtovacieho obdobia dosiahnutá Garantovaná úspora
  energie. Správa musí poskytovať dostatočné informácie o Úspore energie
  vyplývajúcej z implementovaných Opatrení a o výpočte vykonanom na
  určenie Úspory energie. Výročnú správu zašle Dodávateľ Objednávateľovi
  každý rok najneskôr do 20 Pracovných dní od skončenia Vyúčtovacieho
  obdobia. Správa môže byť zaslaná aj prostredníctvom Platformy pre
  Zmluvy o poskytovaní energetických služieb --
  \href{http://www.sharex.lv}{http://sunshineplatform.eu/}.
\item
  Ak má Objednávateľ námietky voči záverom uvedeným vo Výročnej správe,
  musí o tom písomne informovať Dodávateľa do 15 Pracovných dní od
  prijatia Správy alebo od prijatia Oznámenia z Platformy pre Zmluvy o
  poskytovaní energetických služieb --
  \href{http://www.sharex.lv}{http://sunshineplatform.eu/}. Objednávateľ
  uvedie Dodávateľovi dôvody svojich námietok. Dodávateľ urobí v
  priebehu nasledujúcich 15 Pracovných dní od prijatia námietok potrebné
  zmeny a doplnenia a v prípade potreby ich oznámi Objednávateľovi a
  následne písomne informuje Objednávateľa.
\item
  Akákoľvek neoprávnená manipulácia alebo zásah Objednávateľa do
  Opatrení realizovaných v Budove, ktoré vedú k zníženiu úrovne Úspory
  energie, sa zohľadní pri Meraní a overovaní Garantovanej úspory
  energie podľa Zmluvy a slúži na opätovné nastavenie Garantovanej
  úrovne na pomernom základe.
\item
  Objednávateľ berie na vedomie a súhlasí s tým, že Dodávateľ alebo
  akákoľvek iná tretia strana určená Dodávateľom, ktorej boli pridelené
  práva a povinnosti vyplývajúce z tejto Zmluvy, môžu používať:

  \begin{enumerate}
  \def\labelenumii{\arabic{enumii}.}
  \item
    akékoľvek anonymné údaje a informácie týkajúce sa spotreby energie
    Budovou, či už poskytnuté Objednávateľom alebo získané Dodávateľom,
    na účely porovnávania a zostavenia celoštátnej, regionálnej alebo
    medzinárodnej databázy alebo na účely použitia Dodávateľom ako
    referencia alebo na akýkoľvek interný účel dohodnutý s
    Objednávateľom;
  \item
    osobné údaje poskytnuté Objednávateľom alebo Správcom na účely
    poskytovania Služieb a~prenášať ich tretej strane, ktorej môžu byť
    pridelené práva alebo povinnosti vyplývajúce z tejto Zmluvy, vrátane
    akejkoľvek strany vykonávajúcej odkúpenie pohľadávok vyplývajúcich z
    tejto Zmluvy (forfaiting) alebo spravujúcej alebo zodpovednej za
    vývoj, implementáciu, prevádzku a údržbu Platformy pre Zmluvy o
    poskytovaní energetických služieb --
    \href{http://www.sharex.lv}{http://sunshineplatform.eu/} ,
    prostredníctvom ktorej sa sleduje výkonnosť implementovaných
    Opatrení.
  \end{enumerate}
\item
  Dodávateľ je oprávnený ľubovoľne primerane dlho, podľa vlastného
  uváženia, sám alebo prostredníctvom ním poverených osôb, inštalovať,
  prevádzkovať, opravovať a zavádzať systém energetického manažmentu
  alebo obvyklé meracie prístroje a mať prístup k týmto nainštalovaným
  zariadeniam kedykoľvek v primeranom čase v súlade so Zmluvou.
\item
  Dodávateľ je oprávnený inštalovať v Bytoch merače tepla, ak dôjde k
  sťažnosti na nedodržanie Štandardov komfortu. Pokiaľ Vlastníci bytov v
  Budove nesúhlasia s inštaláciou uvedených meračov vo svojich Bytoch
  alebo neposkytnú primeraný prístup k takémuto meraču, Dodávateľ
  nenesie zodpovednosť za údajné nesprávne plnenie Zmluvy vo vzťahu k
  týmto Bytom.
\item
  Údaje zhromaždené meracími prístrojmi Dodávateľa majú informatívny
  charakter a v prípade sporov ich nemožno považovať za jediný základ na
  zistenie porušenia Zmluvy alebo dodržania Štandardov komfortu.
\end{enumerate}

\subsection{RIEŠENIE SPOROV}

\begin{enumerate}
\def\labelenumi{\arabic{enumi}.}
\item
  Akékoľvek nezhody sa budú Zmluvné strany snažiť urovnať predovšetkým
  priateľsky prostredníctvom rokovaní. Za týmto účelom poskytnú Zmluvné
  strany písomné potvrdenie prijatia akéhokoľvek listu od druhej
  Zmluvnej strany týkajúceho sa sporu a venujú primeraný čas na
  vyriešenie sporu buď osobne alebo prostredníctvom zástupcov Zmluvných
  strán v spore, ktorým musí byť člen vyššieho manažmentu.
\item
  Ak má Objednávateľ (alebo ktorýkoľvek Vlastník bytu) sťažnosti na
  Dodávateľa (týkajúce sa napríklad Štandardov komfortu alebo výšky
  Úspory energie alebo všeobecne implementovaných Opatrení a
  poskytovaných Služieb energetickej efektívnosti), musí o tom priamo
  alebo prostredníctvom Správcu písomne informovať Dodávateľa. Dodávateľ
  je povinný overiť a vypracovať Vyjadrenie k reklamácii a v prípade
  zistených porúch vzniknuté problémy odstrániť. Ak problém pretrváva
  dlhšie ako 20 Pracovných dní od oznámenia, Objednávateľ zabezpečí, aby
  sa stretol výbor zložený z riadne oprávnených zástupcov Dodávateľa,
  Správcu a Objednávateľa a na základe sťažnosti a skutočností zistených
  na mieste vypracoval návrh Vyjadrenia k reklamácii alebo vykonal
  akékoľvek iné kroky potrebné na zistenie skutočností v súlade s
  pravidlami na riešenie sporov (mediácie) dostupnými na Platforme pre
  Zmluvy o poskytovaní energetických služieb --
  \href{http://www.sharex.lv}{http://sunshineplatform.eu/}.
\item
  Ak má Dodávateľ sťažnosti na Objednávateľa (napríklad na poškodenie
  inštalovaného zariadenia), je povinný o tom informovať Objednávateľa a
  Správcu. Objednávateľ preverí a prípadne identifikuje páchateľa,
  vypracuje Vyjadrenie k sťažnosti a napraví vzniknutý problém. Ak
  problém pretrváva dlhšie ako 30 dní od oznámenia, Dodávateľ zabezpečí,
  aby by sa stretol výbor zložený z riadne oprávnených zástupcov
  Dodávateľa, Správcu a Objednávateľa a na základe sťažnosti a
  skutočností zistených na mieste vypracoval návrh Vyjadrenia k
  reklamácii alebo vykonal akékoľvek iné kroky potrebné na zistenie
  skutočností v súlade s pravidlami mediácie dostupnými na Platforme pre
  Zmluvy o poskytovaní energetických služieb --
  \href{http://www.sharex.lv}{http://sunshineplatform.eu/}.
\item
  Postup zisťovania skutočností v prípade sporov je nasledovný:

  \begin{enumerate}
  \def\labelenumii{\arabic{enumii}.}
  \item
    Skutočné Štandardy komfortu (teplota okolia jednotlivých Bytov v
    Budove) sa považujú za riadne zaznamenané, ak merania teploty
    vykonáva nezávislý certifikovaný energetický audítor (podľa zákona
    č. 321/2014 Z. z. o energetickej efektívnosti) v súlade s Vyhláškou
    č. 152/2005 Z. z. Vyjadrenie sa vypracuje na základe meraní
    vykonaných nezávislým certifikovaným energetickým audítorom;
  \item
    Všeobecné problémy s implementovanými Opatreniami, ako napríklad
    nefunkčné zariadenie a/alebo vady a poškodenia Opatrení, alebo s
    výpočtom Úspory energie, sa považujú za riadne zaznamenané, ak sú
    nahlásené nezávislým odborníkom, ako je certifikovaný energetický
    audítor v súlade so zákonom č. 321/2014 Z. z. o energetickej
    efektívnosti;
  \item
    Všetky Zmluvné strany musia byť informované najmenej 5 Pracovných
    dní pred akýmkoľvek meraním uskutočneným treťou stranou. Oprávnení
    zástupcovie Zmluvných strán majú právo zúčastniť sa na procese
    merania potrebného na vypracovanie Vyjadrenia. Neprítomnosť
    Oprávneného zástupcu ktorejkoľvek zo Zmluvných strán nepredstavuje
    prekážku pre vypracovanie a podpísanie Vyjadrenia Zmluvnými
    stranami;
  \item
    Podpísanie Vyjadrenia ktoroukoľvek zo Zmluvných strán sa nebude
    považovať za uznanie porušenia tejto Zmluvy a/alebo za vzdanie sa
    práv a povinností Zmluvných strán vyplývajúcich z tejto Zmluvy.
    Náklady nezávislých tretích strán vzniknuté v súvislosti s~vyššie
    uvedeným sa rozdelia rovnomerne medzi Zmluvné strany;
  \item
    Dodávateľovi, Správcovi a Vlastníkovi bytu, ktorý podal sťažnosť, sa
    doručí po jednej kópii podpísaného Vyjadrenia.
  \end{enumerate}
\item
  Ak sa Zmluvným stranám nepodarí dosiahnuť dohodu, Zmluvné strany začnú
  formálny proces mediácie v súlade s pravidlami mediácie dostupnými na
  Platforme pre Zmluvy o poskytovaní energetických služieb --
  \href{http://www.sharex.lv}{http://sunshineplatform.eu/} platnými
  počas doby platnosti Zmluvy a v čase sporu. Ak dôjde medzi Zmluvnými
  stranami k sporu o technických záležitostiach, ktorákoľvek Zmluvná
  strana môže požiadať o vyriešenie sporu v súlade s procesnými
  pravidlami vyšetrovacieho výboru, ktoré sú k dispozícii na Platforme
  pre Zmluvy o poskytovaní energetických služieb
  --\href{http://www.sharex.lv}{http://sunshineplatform.eu/}. Zmluvné
  strany sa dohodli, že náklady na mediačný proces si rozdelia v pomere
  50:50.
\item
  Ak sa Zmluvným stranám nepodarí dosiahnuť prostredníctvom mediačného
  a/alebo vyšetrovacieho procesu (zisťovanie skutočností) vzájomnú
  dohodu, bude spor predložený na riešenie všeobecným súdom Slovenskej
  republiky v súlade so zákonmi a právnymi predpismi platnými v
  Slovenskej republike. Návrh sa podáva na súde podľa jurisdikcie miesta
  bydliska alebo sídla odporcu.
\end{enumerate}

\subsection{ÚDRŽBA OPATRENÍ IMPLEMENTOVANÝCH DODÁVATEĽOM}

\begin{enumerate}
\def\labelenumi{\arabic{enumi}.}
\item
  Dodávateľ je povinný vymeniť, opraviť alebo vykonať generálnu opravu
  zariadenia (alebo akejkoľvek jeho časti) nainštalovaného ako súčasť
  Renovačných prác po uplynutí doby ich životnosti (stanovenej v Manuáli
  prevádzky a údržby) počas Doby poskytovania služieb stanovenej
  Zmluvou.
\item
  Dodávateľ zavedie postupy údržby Opatrení, ktoré zodpovedajú alebo
  prekračujú požiadavky a odporúčania výrobcu na ich príslušnú údržbu,
  ako aj požiadavky a odporúčania uvedené v Osobitných podmienkach.
\end{enumerate}

\subsection{POISTENIE}

\begin{enumerate}
\def\labelenumi{\arabic{enumi}.}
\item
  Na začiatku Obdobia výstavby je Dodávateľ povinný poistiť Budovu s
  písomným súhlasom Objednávateľa na sumu nie nižšiu, ako je hodnota
  obnovy Budovy, s minimálnym poistným krytím pre prípad požiaru,
  zemetrasenia, povodne, poškodenia vodou, iných živelných pohrôm s
  dopadom na Budovu, ako aj pre prípad poškodenia konštrukcie
  spôsobeného prepadom pôdy a spadnutými stromami. Na takéto poistenie
  sa vzťahujú nasledovné ustanovenia:

  \begin{enumerate}
  \def\labelenumii{\arabic{enumii}.}
  \item
    Dodávateľ uzavrie toto poistenie s poisťovňou s hodnotením minimálne
    A+ podľa príslušných hodnotení platných pre Slovenskú republiku.
  \item
    Dodávateľ predloží Objednávateľovi kópiu poistnej zmluvy a doklady
    potvrdzujúce zaplatenie poistného pred začatím Obdobia výstavby.
  \item
    Pre prípad výplaty poistného plnenia v minimálnej výške dostatočnej
    na kompenzáciu hodnoty renovácie Budovy musí byť označený ako
    príjemca Objednávateľ.
  \item
    Stavebné práce v Budove sa nezačnú, kým Dodávateľ nepredloží platne
    uzavretú poistnú zmluvu.
  \item
    Dodávateľ je povinný mať uzavretú poistnú zmluvu po dobu platnosti
    Zmluvy a na žiadosť Objednávateľa predložiť Objednávateľovi originál
    takejto poistnej zmluvy alebo kópiu potvrdenia o~poistení, prípadne
    iné nespochybniteľné doklady potvrdzujúce menu a platenie poistného,
    alebo umožniť prístup k predmetnému dokumentu prostredníctvom
    Platformy pre Zmluvy o poskytovaní energetických služieb --
    \href{http://www.sharex.lv}{http://sunshineplatform.eu/}.
  \item
    Dodávateľ zabezpečí svoje vlastné náklady poistenie Budovy počas
    celého Obdobia výstavby. Po dokončení Renovačných prác a po
    podpísaní Odovzdávacieho a preberacieho protokolu sa náklady na
    poistenie za zostávajúce obdobie platnosti Zmluvy rozdelia medzi
    Vlastníkov bytov v pomere k ploche Bytu nachádzajúceho sa v Budove a
    budú zahrnuté do faktúry Dodávateľa za prevádzku a údržbu. Správca
    menovaný Zmluvnými stranami zabezpečí, aby faktúry Objednávateľa
    vystavené Vlastníkom bytov obsahovali takéto náklady na poistenie.
  \end{enumerate}
\item
  Okrem toho je Dodávateľ počas Obdobia výstavby povinný mať uzavretú
  platnú zmluvu na poistenie občianskej a profesijnej zodpovednosti vo
  výške najmenej 110\% celkových investičných nákladov na Renovačné
  práce.
\end{enumerate}

\subsection{POSTÚPENIE POHĽADÁVOK }

\begin{enumerate}
\def\labelenumi{\arabic{enumi}.}
\item
  Dodávateľ má neobmedzené právo postúpiť tretím osobám svoje práva a
  nároky na ktorúkoľvek z pohľadávok voči Objednávateľovi vyplývajúcich
  z tejto Zmluvy. Dodávateľ je predovšetkým oprávnený postúpiť
  pohľadávky vyplývajúce z Poplatku za renováciu akémukoľvek
  Postupníkovi, ktorý s ním uzavrel zmluvu o financovaní, forfaitingu,
  postúpení pohľadávok alebo akúkoľvek inú zmluvu.
\item
  Postúpenie pohľadávok nezbavuje Dodávateľa jeho povinností a záväzkov
  podľa tejto Zmluvy. Postupník má však na základe tejto Zmluvy právo na
  prevzatie povinností vyplývajúcich z tejto Zmluvy v prípade, že si
  Dodávateľ neplní svoje povinnosti podľa tejto Zmluvy. Právo na
  prevzatie bude mať za cieľ len nahradiť Dodávateľa, ktorý neplní svoje
  povinnosti, iným subjektom schopným plniť všetky povinnosti a záväzky
  podľa tejto Zmluvy v prospech Objednávateľa a Postupníka.
\item
  V prípade takého postúpenia zašle Dodávateľ Objednávateľovi Oznámenie
  o~postúpení do 5 Pracovných dní od uzavretia zmluvy o postúpení.
\item
  Táto Zmluva je pre Objednávateľa osobná, pričom Objednávateľ nie je
  oprávnený postúpiť ju, ani previesť bez predchádzajúceho oznámenia
  Dodávateľovi.
\item
  Ak dôjde k akémukoľvek zlúčeniu alebo akvizícii Dodávateľa, Zmluva
  zostáva v platnosti a jej ustanovenia sú záväzné pre právnych
  nástupcov i prevodcu Dodávateľa.
\end{enumerate}

\subsection{VLASTNÍCKE PRÁVO K OPATRENIAM INŠTALOVANÝM V BUDOVE AKO SÚČASŤ RENOVAČNÝCH PRÁC}

\begin{enumerate}
\def\labelenumi{\arabic{enumi}.}
\item
  Ak Dodávateľ poskytol Finančný príspevok na vykonanie Rekonštrukčných
  prác, vlastníkom Opatrení, ktoré možno oddeliť od Budovy bez
  spôsobenia materiálnych škôd, je Dodávateľ. Ak sú Renovačné práce plne
  financované Objednávateľom, vlastníkom Opatrení je Objednávateľ.
\item
  Objednávateľ je povinný prijať primerané opatrenia, aby zabezpečil, že
  žiaden z Vlastníkov bytov alebo iných návštevníkov neodstráni,
  nezaťaží bremenom (neprenajme si, ani nedá do prenájmu, okrem iného),
  nezaloží, ani nezničí, nepoškodí a nebude neoprávnene zasahovať do
  Opatrení implementovaných v rámci Renovačných prác nezávisle od
  Zmluvnej strany, ktorá je vlastníkom Opatrení.
\item
  Dodávateľ je oprávnený bez súhlasu Objednávateľa (konkrétne bez
  potreby získať súhlas každého z Vlastníkov bytov) založiť a zaťažiť
  bremenom Opatrenia (alebo ich časti) vo svoj výlučný prospech alebo v
  prospech tretích strán, ak:

  \begin{enumerate}
  \def\labelenumii{\arabic{enumii}.}
  \item
    je Dodávateľ ich vlastníkom;
  \item
    je technicky možné demontovať ich bez toho, aby došlo k podstatnému
    poškodeniu Budovy;
  \item
    je záloh a/alebo bremeno potrebné ako záruka za Finančný príspevok
    Dodávateľa podľa tejto Zmluvy. Dodávateľ nie je oprávnený založiť
    Opatrenia na získanie finančných zdrojov na iné účely, než je
    plnenie tejto Zmluvy; a
  \item
    doba platnosti zálohu a/alebo bremena nesmie presiahnuť dobu
    platnosti Zmluvy.
  \end{enumerate}
\item
  V prípade, že Dodávateľ je vlastníkom Opatrení, po prijatí všetkých
  platieb splatných na základe tejto Zmluvy Dodávateľovi sa vlastnícke
  právo ku všetkým Opatreniam implementovaných ako súčasť Renovačných
  prác podľa Zmluvy považuje za automaticky prevedené na Objednávateľa.
  Tento prevod sa zrealizuje za nerefundovateľnú cenu 1.00 EUR (jedno
  euro) splatnú vopred pri podpise tejto Zmluvy. Prevod vlastníckeho
  práva na Objednávateľa sa potvrdí Vyhlásením o prevode podpísaným
  Dodávateľom a Objednávateľom.
\end{enumerate}

\subsection{SOFTVÉR A PRÁVA DUŠEVNÉHO VLASTNÍCTVA}

\begin{enumerate}
\def\labelenumi{\arabic{enumi}.}
\item
  Dodávateľ zabezpečí, aby mal licenciu alebo vlastnil všetky práva
  duševného a priemyselného vlastníctva (PDV) k Opatreniam
  implementovaným v Budove, vrátane vybavenia, materiálov, systémov,
  softvéru alebo akejkoľvek inej veci alebo dokumentu dodaného
  Dodávateľom Objednávateľovi na základe tejto Zmluvy. Zmluvné strany sa
  dohodli, že Dodávateľ zostáva vlastníkom týchto práv a že tieto práva
  neprechádzajú na Objednávateľa. Dodávateľ udeľuje Objednávateľovi
  trvalú, neodvolateľnú, nevýhradnú bezplatnú licenciu (s právom na
  poskytovanie sublicencií) na využívanie uvedených práv duševného
  vlastníctva v súvislosti s používaním Budovy, nie však na akékoľvek
  iné využitie.
\item
  Objednávateľ nesmie upravovať, kopírovať, ani spätne analyzovať žiaden
  softvér, ani ho spájať s iným softvérom, ktorý mu Dodávateľ poskytol
  ako súčasť Renovačných prác. Počas doby platnosti tejto Zmluvy
  Dodávateľ poskytne Objednávateľovi užívateľské príručky, manuály,
  technické informácie a všetky aktualizácie a revízie poskytovaného
  softvéru.
\item
  Dodávateľ je povinný odškodniť Objednávateľa v prípade akýchkoľvek
  nárokov, za ktoré je Objednávateľ právne zodpovedný, uplatnených v
  súvislosti s akýmkoľvek porušením práv duševného vlastníctva tretích
  strán týkajúcich sa akejkoľvek časti PDV dodaných Dodávateľom.
  Povinnosť Dodávateľa odškodniť Objednávateľa v prípade takýchto
  nárokov je podmienená tým, že Objednávateľ:

  \begin{enumerate}
  \def\labelenumii{\arabic{enumii}.}
  \item
    okamžite písomne informoval Dodávateľa o takomto nároku;
  \item
    neuznal takýto nárok, ani neovplyvnil obhajobu Dodávateľa voči
    takémuto nároku alebo schopnosť Dodávateľa rokovať o uspokojivom
    vysporiadaní;
  \item
    poskytol Dodávateľovi možnosť viesť na náklady Dodávateľa obhajobu a
    akékoľvek rokovania o vysporiadaní nárokov; a
  \item
    poskytol Dodávateľovi (na náklady Dodávateľa) takú súčinnosť a
    informácie, ktoré môže Dodávateľ odôvodnene požadovať v prípade
    obhajoby a akýchkoľvek rokovaní o vysporiadaní nárokov.
  \end{enumerate}
\item
  Dodávateľ podľa svojej voľby buď nahradí alebo upraví časť porušujúcu
  práva duševného vlastníctva časťou, ktorá neporušuje práva, alebo
  zabezpečí Objednávateľovi právo používať takúto časť porušujúcu práva
  duševného vlastníctva. Opravné prostriedky ustanovené v tomto článku
  sú jediným a výlučným opravným prostriedkom v prípade porušenia práv
  duševného vlastníctva.
\end{enumerate}

\subsection{ZMENY V UŽÍVANÍ BUDOVY}

\begin{enumerate}
\def\labelenumi{\arabic{enumi}.}
\item
  Budova je opísaná v Osobitných podmienkach vrátane jej užívania,
  plochy a rozmerov. Ak sa akékoľvek okolnosti, na ktorých boli založené
  výpočty Dodávateľa, zmenia z podnetu Objednávateľa alebo s jeho
  súhlasom alebo povolením, nebude mať táto zmena vplyv na Dodávateľa a
  plnenie Zmluvy. Zmeny v užívaní Budovy a úpravy Budovy sa vyhodnotia z
  hľadiska ekonomických aspektov (najmä z~hľadiska zmien týkajúcich sa
  nákladov) a Zmluva sa podľa toho prispôsobí novým okolnostiam.
\item
  Zmeny v užívaní Budovy zahŕňajú:

  \begin{enumerate}
  \def\labelenumii{\arabic{enumii}.}
  \item
    zväčšenie alebo zmenšenie plochy povrchu Budovy;
  \item
    montáž, poškodenie alebo demontáž príslušného zariadenia alebo iných
    inštalácií, ak to má za následok podstatné zvýšenie alebo zníženie
    spotreby energie alebo iných technických parametrov Budovy;
  \item
    zmeny v užívaní Budovy (napríklad bytová časť sa zmení na obchody,
    prevádzky, reštaurácie a kancelárie alebo sa začnú užívať
    nevyužívané / neobývané byty) ovplyvňujúce energetickú náročnosť
    Budovy.
  \end{enumerate}
\end{enumerate}

\subsection{NAKLADANIE S ODPOJENÝMI A/ALEBO ODMONTOVANÝMI ZARIADENIAMI A MATERIÁLOM}

\begin{enumerate}
\def\labelenumi{\arabic{enumi}.}
\item
  Dodávateľ zabezpečí na svoje náklady likvidáciu odpadu generovaného v
  súvislosti s plnením tejto Zmluvy v súlade s príslušnými zákonmi a
  predpismi o likvidácii odpadu platnými v Slovenskej republike.
\item
  Dodávateľ je povinný písomne o tom informovať Objednávateľa najneskôr
  5 Pracovných dní pred prvou plánovanou likvidačnou činnosťou. Takéto
  oznámenie sa týka všetkého vybavenia, materiálu a iného majetku
  nainštalovaného v Budove a podliehajúceho demontáži a výmene za účelom
  realizácie a inštalácie Opatrení počas Obdobia výstavby.
\item
  Ak si Objednávateľ želá použiť akékoľvek zariadenie, materiál alebo
  iný majetok odpojený alebo demontovaný Dodávateľom počas Obdobia
  výstavby, je povinný to písomne oznámiť Dodávateľovi a zabezpečiť na
  svoje náklady jeho vyzdvihnutie a prepravu.
\end{enumerate}

\subsection{ZODPOVEDNOSŤ}

\begin{enumerate}
\def\labelenumi{\arabic{enumi}.}
\item
  Dodávateľ zodpovedá za včasnú implementáciu Opatrení v dohodnutom
  Období výstavby. Nesplnenie tejto povinnosti zo strany Dodávateľa
  oprávňuje Objednávateľa na zmluvnú pokutu vo výške 0,02\% z celkových
  plánovaných investičných nákladov za deň. Zmluvná pokuta však nesmie
  prekročiť 10\% z plánovaných investičných nákladov.
\item
  Objednávateľ je zodpovedný za včasné zaplatenie splatných nákladov a
  Poplatkov podľa tejto Zmluvy. Dodávateľ je oprávnený požadovať náhradu
  za omeškanie s úhradou. Náhrada za omeškanie s úhradou zodpovedá
  zmluvnej pokute vo výške 0,1\% z dlžnej sumy za deň. Úhradou zmluvnej
  pokuty nie je dotknutý nárok na náhradu škody vo výške presahujúcej
  zmluvnú pokutu.
\item
  Ak Objednávateľ neuhradí ktorúkoľvek dlžnú sumu a zo Zmluvy nevyplýva
  lehota dlhšia ako 90 dní, počas ktorej možno riadne a efektívne
  uplatniť Postupy riešenia sporov vzniknutých v súvislosti s touto
  Zmluvou, Dodávateľ je oprávnený vypovedať Zmluvu z dôvodu neplnenia
  povinností a porušenia Zmluvy zo strany Objednávateľa.
\item
  Dodávateľ je zodpovedný za dodržiavanie Štandardov komfortu v Budove
  stanovených touto Zmluvou. Ak je počas Vykurovacej sezóny vnútorná
  teplota v ktoromkoľvek z Bytov v priemere o 2 stupne Celzia (s ohľadom
  na presnosť prístrojového vybavenia) nižšia ako je teplota stanovená
  Štandardmi komfortu podľa tejto Zmluvy, Dodávateľ je povinný dať pokyn
  Správcovi na zníženie faktúry Objednávateľa pre príslušného Vlastníka
  bytu nasledovne:

  \begin{enumerate}
  \def\labelenumii{\arabic{enumii}.}
  \item
    zľava 5\% za každý stupeň Celzia z Poplatku za energiu za každý
    mesiac Vykurovacej sezóny, keď teplota klesla pod dohodnuté
    Štandardy komfortu;
  \item
    vnútorná teplota a to, či bola úroveň teploty nižšia ako Štandardy
    komfortu, sa určí v súlade s Postupmi riešenia sporov stanovenými
    touto Zmluvou;
  \item
    Dodávateľ neposkytne zľavu, ak došlo k poklesu vnútornej teploty v
    Byte (i) v dôsledku konania alebo opomenutia užívateľov alebo
    Vlastníkov bytu v rozpore s touto Zmluvou, (ii) v dôsledku neplnenia
    povinností Objednávateľa alebo (iii) z iných dôvodov, ktoré nemožno
    pripísať Dodávateľovi.
  \end{enumerate}
\item
  Objednávateľ zodpovedá za poškodenie, manipuláciu alebo nedovolené
  zasahovanie, vandalizmus, sabotáž, krádež (pokiaľ ich nezapríčinil
  Dodávateľ alebo osoby, za ktoré je Dodávateľ zodpovedný) Opatrení,
  najmä ak ovplyvnia úroveň Úspory energie, dohodnutých Štandardov
  komfortu alebo bezpečnosť osôb žijúcich v~Budove a užívajúcich Budovu.
  V takomto prípade je Objednávateľ povinný:

  \begin{enumerate}
  \def\labelenumii{\arabic{enumii}.}
  \item
    plne nahradiť Dodávateľovi náklady na obnovenie príslušného
    Opatrenia;
  \item
    zaplatiť Dodávateľovi zmluvnú pokutu vo výške 10\% z nákladov na
    obnovenie Opatrenia a súvisiacich administratívnych nákladov
    vzniknutých Dodávateľovi;
  \item
    náklady na obnovenie sa vypočítajú na základe platných trhových cien
    relevantných v čase výpočtov;
  \item
    zodpovednosť Objednávateľa za poškodenie, manipuláciu alebo
    zasahovanie do Opatrení sa určí v súlade s Postupmi riešenia sporov
    stanovenými touto Zmluvou.
  \end{enumerate}
\item
  Dodávateľ zaistí Objednávateľa voči zodpovednosti, nákladom, výdavkom,
  škodám alebo poplatkom, ktoré mu môžu vzniknúť v dôsledku reklamácie
  alebo sťažnosti, správnych alebo súdnych žalôb podaných voči
  Objednávateľovi orgánmi štátnej správy alebo tretími stranami,
  vyplývajúcim z konania Dodávateľa alebo akýchkoľvek potenciálnych
  porušení práv duševného vlastníctva týkajúcich sa Opatrení
  realizovaných Dodávateľom. Dodávateľ je povinný poskytnúť
  Objednávateľovi náhradu všetkých nákladov a výdavkov primerane
  požadovaných na náhradu všetkých priamych škôd vzniknutých v dôsledku
  konania Dodávateľa v rozpore s platnými právnymi predpismi. Náhrada
  musí byť riadne zdokumentovaná a splatná do 30 Pracovných dní od
  prijatia príslušnej požiadavky od Objednávateľa Dodávateľom s
  výslovným uvedením dlžnej sumy.
\item
  Poplatky za poskytovanie služieb všeobecného hospodárskeho záujmu
  (vrátane dodávky tepla) a sankcie uplatniteľné voči ktorejkoľvek
  Zmluvnej strane uvedené v zákonoch a iných právnych predpisoch
  platných v Slovenskej republike neobmedzujú povinnosti Zmluvných strán
  vyplývajúce z tejto Zmluvy vrátane zodpovednosti Zmluvných strán a
  úhrady príslušných zmluvných pokút, náhrad a penále podľa tejto
  Zmluvy.
\item
  Úhrada pokút, náhrad a penále nezbavuje Zmluvnú stranu povinností
  stanovených touto Zmluvou.
\end{enumerate}

\subsection{UKONČENIE ZMLUVY}

\begin{enumerate}
\def\labelenumi{\arabic{enumi}.}
\item
  V prípade, že ktorákoľvek zo Zmluvných strán poruší podstatné
  ustanovenie tejto Zmluvy, môže Zmluvná strana, ktorá si riadne plní
  svoje povinnosti, ukončiť túto Zmluvu s okamžitou platnosťou a
  požadovať od Zmluvnej strany, ktorá Zmluvu porušila, odškodnenie za
  neplnenie tejto Zmluvy.
\item
  Ukončenie Zmluvy pred Dátumom začatia a investovaním do stavebných a
  inštalačných prác:

  \begin{enumerate}
  \def\labelenumii{\arabic{enumii}.}
  \item
    V prípade jednostranného ukončenia Zmluvy zo strany Objednávateľa z
    dôvodu podstatného neplnenia alebo porušenia Zmluvy zo strany
    Dodávateľa má Objednávateľ nárok na náhradu vo výške 1\% z
    Investičných nákladov (bez DPH) plánovaných v Zmluve.
  \item
    V prípade jednostranného ukončenia Zmluvy zo strany Dodávateľa z
    dôvodu podstatného neplnenia alebo porušenia Zmluvy zo strany
    Objednávateľa má Dodávateľ nárok na náhradu vo výške 1\% z
    Investičných nákladov (bez DPH) plánovaných v Zmluve.
  \item
    V prípade jednostranného ukončenia Zmluvy zo strany Objednávateľa z
    iných obchodných alebo podnikateľských dôvodov, ktoré nie sú
    nevyhnutne spojené s touto Zmluvou, má Dodávateľ nárok na náhradu vo
    výške 1\% z Investičných nákladov (bez DPH) plánovaných v Zmluve.
  \item
    V prípade jednostranného ukončenia Zmluvy zo strany Dodávateľa z
    iných obchodných alebo podnikateľských dôvodov, ktoré nie sú
    nevyhnutne spojené s touto Zmluvou, má Objednávateľ nárok na náhradu
    vo výške 1\% z Investičných nákladov (bez DPH) plánovaných v Zmluve.
  \end{enumerate}
\item
  Ukončenie Zmluvy po vzniku investičných nákladov na stavebné a
  inštalačné práce súvisiace s Opatreniami, ktoré boli pokryté Finančným
  príspevkom Dodávateľa:

  \begin{enumerate}
  \def\labelenumii{\arabic{enumii}.}
  \item
    V prípade jednostranného ukončenia Zmluvy zo strany Objednávateľa z
    dôvodu závažného neplnenia alebo porušenia Zmluvy zo strany
    Dodávateľa nahradí Objednávateľ len zostatok Finančného príspevku
    poskytnutého Dodávateľom so zľavou 3\%, pričom takáto náhrada sa
    považuje za zodpovedajúcu kritériám výkonnosti stanoveným v Zmluve.
    Okrem vyššie uvedeného má Objednávateľ právo na kompletnú projektovú
    dokumentáciu s podrobným uvedením doteraz vykonaných prác spolu so
    všetkými povoleniami, licenciami alebo inými dokumentmi, ktoré
    Dodávateľ získal na základe tejto Zmluvy, ukončenie
    najnevyhnutnejších prác, všetky záruky poskytnuté výrobcami,
    akékoľvek sublicencie (a prevod akýchkoľvek licencií) potrebné na
    používanie príslušných práv duševného vlastníctva a softvéru
    (vrátane softvéru, ktorý bol nainštalovaný, a, podľa potreby,
    akejkoľvek sprievodnej dokumentácie, informácií súvisiacich s kódom,
    akýchkoľvek zdrojových kódov, dátových súborov, výpočtov,
    elektronických médií, výtlačkov alebo súvisiacich informácií),
    a~tiež na zaškolenie akýchkoľvek tretích strán, ktoré boli
    Objednávateľom výslovne určené v prípade, že bola dokončená
    realizácia Renovačných prác.
  \item
    V prípade jednostranného ukončenia Zmluvy zo strany Dodávateľa z
    dôvodu závažného neplnenia alebo porušenia Zmluvy zo strany
    Objednávateľa nahradí Dodávateľ len zostatok Finančného príspevku
    plus kompenzáciu vo výške 3\% zo sumy, ktorá má byť uhradená.
    Objednávateľ má právo na kompletnú projektovú dokumentáciu s
    podrobným uvedením doteraz vykonaných prác spolu so všetkými
    povoleniami, licenciami alebo inými dokumentmi, ktoré Dodávateľ
    získal na základe tejto Zmluvy, všetky záruky poskytnuté výrobcami,
    akékoľvek sublicencie (a prevod akýchkoľvek licencií) potrebné na
    používanie príslušných práv duševného vlastníctva a softvéru
    (vrátane softvéru, ktorý bol nainštalovaný, a podľa potreby,
    akejkoľvek sprievodnej dokumentácie, informácií súvisiacich s kódom,
    akýchkoľvek zdrojových kódov, dátových súborov, výpočtov,
    elektronických médií, výtlačkov alebo súvisiacich informácií).
  \item
    V prípade jednostranného ukončenia Zmluvy zo strany Objednávateľa z
    iných obchodných alebo podnikateľských dôvodov, ktoré nie sú
    nevyhnutne spojené s touto Zmluvou, má Dodávateľ nárok na náhradu
    zodpovedajúcu zostatku Finančného príspevku poskytnutého Dodávateľom
    plus kompenzáciu vo výške 3\% zo sumy, ktorá má byť uhradená.
  \item
    V prípade jednostranného ukončenia Zmluvy zo strany Dodávateľa z
    iných obchodných alebo podnikateľských dôvodov, ktoré nie sú
    nevyhnutne spojené s touto Zmluvou, má Dodávateľ nárok na náhradu vo
    výške zodpovedajúce nesplatenej istine Finančného príspevku
    poskytnutého Dodávateľom so zľavou 3\% z~takto vypočítanej sumy.
  \end{enumerate}
\item
  Dodávateľ alebo ktorýkoľvek z jeho postupníkov vystaví Objednávateľovi
  faktúru na vypočítanú náhradu, v ktorej budú jasne uvedené informácie
  použité na takýto výpočet na základe Harmonogramov platieb
  vypracovaných počas Doby poskytovania služieb alebo faktúr zaplatených
  počas vykonávania Renovačných prác pred dátum ukončenia Zmluvy.
  Objednávateľ zaplatí do 60 dní odo dňa vystavenia faktúry náhradu
  splatnú Dodávateľovi alebo niektorému z jeho príjemcov, postupníkov
  alebo iných subjektov, ktoré boli jednostranne označené ako právne
  oprávnené na všetky alebo niektoré z práv podľa tejto Zmluvy.
\item
  Predčasné ukončenie Zmluvy musí byť Zmluvnou stranou, ktorá ukončí
  Zmluvu, písomne oznámené (oznámenie o ukončení) najmenej 20 Pracovných
  dní vopred. V~prípade ukončenia Zmluvy z dôvodu neplnenia alebo
  porušenia Zmluvy ktoroukoľvek Zmluvnou stranou musí platné písomné
  oznámenie o ukončení obsahovať kroky podniknuté v rámci Postupov
  riešenia sporov stanovených touto Zmluvou a súvisiacou podpornou
  dokumentáciou.
\item
  Objednávateľ je oprávnený kedykoľvek požadovať od Dodávateľa
  a~Dodávateľ je oprávnený kedykoľvek predkladať Objednávateľovi výpočet
  sumy, ktorá má byť Dodávateľovi nahradená v prípade predčasného
  ukončenia Zmluvy.
\item
  Ukončenie Zmluvy vo všeobecnosti nezbavuje Zmluvné strany plnenia
  povinností stanovených Zmluvou, ktoré vznikli pred ukončením Zmluvy,
  pokiaľ sa Zmluvné strany písomne nedohodli na iných ustanoveniach
  alebo pokiaľ Zmluva neustanovuje inak. Najmä jednostranné ukončenie
  Zmluvy zo strany Objednávateľa v prípade závažného neplnenia alebo
  porušenia Zmluvy zo strany Dodávateľa samo osebe nezbavuje
  Objednávateľa povinnosti platiť faktúry vystavené za obdobia
  predchádzajúce dátumu ukončenia Zmluvy.
\item
  Reorganizácia, zmena akcionárov a/alebo vlastníkov, zmena manažmentu
  Zmluvných strán, vrátane zmien Vlastníkov bytov v Budove, neslúži ako
  dôvod na ukončenie Zmluvy alebo neplnenie povinností stanovených
  Zmluvou.
\item
  Okrem prípadov uvedených v Zmluve môžu Zmluvné strany kedykoľvek
  ukončiť Zmluvu na základe vzájomnej písomnej dohody obsahujúcej
  podmienky ukončenia.
\item
  Zmluvná strana, ktorá má nárok na náhradu, sa môže domáhať náhrady buď
  prostredníctvom výkonu svojich práv podľa tejto Zmluvy alebo v súlade
  s príslušnými zákonmi a inými právnymi predpismi platnými v Slovenskej
  republike. Oprávnená Zmluvná strana sa však nemôže dožadovať dvojitej
  náhrady za rovnaké neplnenie alebo porušenie.
\item
  Zmluvné strany sa môžu dohodnúť na demontáži Opatrení vlastnených
  alebo čiastočne vlastnených Dodávateľom z Budovy, ak dôjde k
  predčasnému ukončeniu Zmluvy z dôvodu akýchkoľvek okolností a ak bude
  dotknutá Zmluvná strana súhlasiť s hodnotou príslušných Opatrení a
  výpočtom náhrady. Možnosť demontovať Opatrenia za okolností
  stanovených v tomto bode nemá vplyv na nároky na náhradu škody, ako aj
  nákladov a výdavkov, na ktoré môže mať ktorákoľvek zo Zmluvných strán
  nárok v súvislosti s predčasným ukončením tejto Zmluvy.
\end{enumerate}

\subsection{ZÁSAH VYŠŠEJ MOCI}

\begin{enumerate}
\def\labelenumi{\arabic{enumi}.}
\item
  Akákoľvek mimoriadna situácia alebo udalosť, ktorá sa nedá vopred
  predvídať a ktorá sa vyznačuje všetkými nasledujúcimi znakmi, sa
  definuje ako zásah vyššej moci:

  \begin{enumerate}
  \def\labelenumii{\arabic{enumii}.}
  \item
    Zmluvné strany ju nie sú schopné predvídať a ovplyvniť;
  \item
    zasahuje do plnenia povinností Zmluvných strán;
  \item
    nemožno ju kvalifikovať ako chybu alebo nedbanlivosť spôsobenú
    ktoroukoľvek zo Zmluvných strán;
  \item
    dá sa preukázať alebo ju možno uznať za neprekonateľnú, aj keď
    Zmluvná strana (Zmluvné strany) vyvinula (-li) primerané úsilie, aby
    jej zabránili.
  \end{enumerate}
\item
  Zásah vyššej moci zahŕňa okrem iného vojnu, živelnú pohromu, pandémie
  a právne akty orgánov štátnej správy.
\item
  Za zásah vyššej moci sa NEPOVAŽUJÚ vady Opatrení, služby poskytnuté v
  inej ako v dohodnutej kvalite alebo množstve alebo použitie alebo
  inštalácia vybavenia alebo materiálov Dodávateľom v inej ako
  dohodnutej kvalite alebo množstve alebo omeškanie s prevádzkou
  Opatrení (pokiaľ neboli spôsobené zásahom vyššej moci), námietky
  Objednávateľa, štrajky, finančné ťažkosti alebo iné podobné okolnosti
  týkajúce sa Zmluvnej strany, ktorá sa odvoláva na zásah vyššej moci.
\item
  Zmluvné strany nezodpovedajú za úplné ani čiastočné neplnenie
  povinností vyplývajúcich zo Zmluvy, ak dôjde k zásahu vyššej moci.
  Zmluvná strana, ktorá sa odvoláva na zásah vyššej moci, je povinná
  poskytnúť druhej Zmluvnej strane dôkaz o takomto zásahu.
\item
  Zmluvná strana (ďalej len „Dotknutá Zmluvná strana``), ktorá nemôže
  plniť svoje povinnosti vyplývajúce z tejto Zmluvy, je povinná
  bezodkladne, najneskôr však do 3 Pracovných dní odkedy ho
  predpokladala alebo sa o ňom dozvedela, oznámiť druhej Zmluvnej strane
  výskyt Zásahu vyššej moci, opísať vzniknutú situáciu, ktorá nastane
  alebo nastala, s uvedením podrobností o zásahu, jeho možnom trvaní,
  predpokladaných dôsledkov a ich prípadného zmiernenia.
\item
  Zmluvné strany vykonajú spoločne a nerozdielne všetky potrebné úkony
  na zmiernenie dopadu zásahu vyššej moci a podniknú primerané kroky na
  elimináciu vzniknutých škôd.
\item
  Ak zásah vyššej moci trvá bez prerušenia dlhšie ako 6 mesiacov a v
  nasledujúcich 3 mesiacoch sa nepredpokladá jeho ukončenie, Dodávateľ
  alebo Objednávateľ je oprávnený jednostranne ukončiť Zmluvu.
\end{enumerate}

\subsection{DÔVERNOSŤ INFORMÁCIÍ}

\begin{enumerate}
\def\labelenumi{\arabic{enumi}.}
\item
  Informácie získané počas uzatvárania alebo plnenia Zmluvy, ktoré nie
  sú všeobecne dostupné tretím stranám a ktoré boli sprístupnené
  prijímajúcej Zmluvnej strane, pričom môžu poškodiť zákonné práva alebo
  záujmy Zmluvnej strany poskytujúcej takéto informácie, sa považujú za
  dôverné.
\item
  Zmluvné strany sa zaväzujú, že neposkytnú Dôverné informácie druhej
  Zmluvnej strany tretím stranám, ani nesprístupnia údaje druhej
  Zmluvnej strany, ktoré by mohli byť použité na účely konkurenčnej
  činnosti alebo páchania trestnej činnosti, a to počas platnosti Zmluvy
  plus 3 roky po jej ukončení.
\item
  Informácie, ktoré zverejnili tretie strany bez toho, aby Zmluvné
  strany porušili ustanovenia Zmluvy, sa nepovažujú za dôverné.
\item
  Zmluvné strany môžu sprístupniť dôverné informácie tretím stranám na
  účely plnenia povinností vyplývajúcich zo Zmluvy. Ak Zmluvné strany
  sprístupnia dôverné informácie v~súlade s~týmto ustanovením, sú
  povinné zabezpečiť, aby tretia strana dodržiavala rovnaké povinnosti
  týkajúce sa dôvernosti, ako sú stanovené v tejto Zmluve.
\item
  Sprístupnenie dôverných informácií požadované v súlade so zákonmi a
  inými právnymi predpismi platnými v Slovenskej republike sa nepovažuje
  za porušenie Zmluvy.
\item
  Na reklamné účely a na účely informovania širokej verejnosti sú
  Dodávateľ, všetci jeho postupníci a Objednávateľ oprávnení sprístupniť
  všeobecné informácie o vzájomnej spolupráci, okrem iného informácie o
  Zmluvných stranách, povahe spolupráce a dosiahnutej Úspore energie,
  ako aj údaje o Spotrebe energie, ktoré už sú verejne dostupné, a to do
  takej miery, do akej zverejnenie informácií neporušuje zákonné práva a
  záujmy druhej Zmluvnej strany týkajúce sa ochrany dôverných
  informácií. Ak má Zmluvná strana pochybnosti o povahe konkrétnych
  informácií pred ich sprístupnením, ich povahu musí určiť Zmluvná
  strana (Zmluvné strany), ktorej zákonné práva a záujmy by mohli byť
  sprístupnením takýchto informácií porušené, avšak len v prípade, ak
  táto Zmluvná strana považuje takúto informáciu za takú, ktorá
  nepodlieha povinnosti mlčanlivosti stanovenej touto Zmluvou.
\item
  Vyššie uvedené platí bez toho, aby bola dotknutá výslovná povinnosť
  Objednávateľa nepožadovať, ani neporadiť žiadnemu objednávateľovi,
  potenciálnemu objednávateľovi alebo obchodnému kontaktu Dodávateľa
  a/alebo stavebným spoločnostiam, aby obmedzili, zrušili, odvolali,
  limitovali, redukovali alebo inak znížili mieru služieb poskytovaných
  Dodávateľom.
\item
  Vyššie uvedené ustanovenia sa nedotýkajú práva Dodávateľa
  zhromažďovať, spracúvať, uchovávať, prenášať, prevádzať na
  nadobúdateľov svojich financujúcich partnerov a šíriť všetky údaje
  získané od Objednávateľa na účely zlepšenia kvality jeho Služieb, ako
  aj na účely vývoja, prevádzky a údržby online Platformy pre Zmluvy o
  poskytovaní energetických služieb --
  \href{http://www.sharex.lv}{http://sunshineplatform.eu/} podporujúcej
  všetky etapy a zúčastnené strany v bežnom Projekte Zmluvy o
  poskytovaní energetických služieb.
\end{enumerate}

\subsection{UZATVORENIE A ZMENY TEJTO ZMLUVY}

\begin{enumerate}
\def\labelenumi{\arabic{enumi}.}
\item
  Zmluva nadobúda platnosť dňom jej podpísania všetkými Zmluvnými
  stranami podľa týchto Všeobecných podmienok a zostáva v platnosti až
  do úplného splnenia všetkých povinností Zmluvnými stranami.
\item
  Všetky úpravy, doplnenia a zmeny Zmluvy môžu byť vykonané vo forme
  písomných dodatkov na základe vzájomnej dohody všetkých Zmluvných
  strán, ktoré nadobudnú účinnosť po ich podpísaní všetkými Zmluvnými
  stranami a budú tvoriť neoddeliteľnú prílohu k tejto Zmluve.
\item
  Všetky ostatné ustanovenia Podmienok alebo príslušných Príloh
  zostávajú v platnosti a účinnosti. Akékoľvek odchýlky sa uplatňujú iba
  na tú časť Zmluvy, pre ktorú boli dohodnuté.
\item
  Zmluva sa považuje za ukončenú po uhradení všetkých Poplatkov v plnej
  výške zo strany Objednávateľa a splnení všetkých povinností zo strany
  Dodávateľa.
\item
  Ak počas doby platnosti Zmluvy nadobudnú účinnosť zmeny a doplnenia
  zákonov a iných právnych predpisov platných v Slovenskej republike,
  ktoré úplne alebo čiastočne znemožnia plnenie akýchkoľvek povinností
  podľa tejto Zmluvy, nebude to mať vplyv na platnosť ostatných
  povinností stanovených touto Zmluvou. V takom prípade Zmluvné strany
  prijmú vhodné zmeny a doplnenia Zmluvy s úmyslom a účelom
  minimalizovať ekonomický dopad takýchto zmien na Zmluvné strany.
\end{enumerate}

\subsection{ZASTÚPENIE ZMLUVNÝCH STRÁN}

\begin{enumerate}
\def\labelenumi{\arabic{enumi}.}
\item
  V prípade tejto Zmluvy musia Zmluvné strany zastupovať ich legitímni
  zástupcovia (pokiaľ ide o právnické osoby) alebo osoby osobitne
  uvedené v tejto Zmluve. Objednávateľa alebo Dodávateľa sú oprávnené
  zastupovať iba osoby uvedené v Osobitných podmienkach.
\item
  Zmluva je vyhotovená a podpísaná v 3 origináloch v slovenskom jazyku s
  rovnakým právnym účinkom. Zmluvné strany svojím podpisom potvrdzujú,
  že rozumejú obsahu, zmyslu a dôsledkom Zmluvy, uznávajú, že táto
  Zmluva je správna, vzájomne výhodná a zahŕňa všetky ustanovenia,
  prísľuby, podmienky a vyjadrenia zámerov Zmluvných strán a že si ju
  dobrovoľne želajú plniť bez vyvíjania akejkoľvek nátlaku na vôľu
  ktorejkoľvek z nich.
\end{enumerate}

\subsection{OZNÁMENIA}

\begin{enumerate}
\def\labelenumi{\arabic{enumi}.}
\item
  Forma Oznámení: všetky oznámenia, požiadavky, nároky, žiadosti a iná
  komunikácia vyžadovaná alebo povolená na základe podmienok stanovených
  touto Zmluvou musia byť zaslané a doručované na adresy Zmluvných
  strán.
\item
  Spôsob zasielania Oznámení: všetky oznámenia musia byť doručované (i)
  osobne (ii) kuriérskou službou na nasledujúci deň, (iii) doporučenou
  poštou, (iv) faxom, (v) elektronickou poštou (e-mailom) so žiadosťou o
  Potvrdenie o doručení na adresu Zmluvnej strany uvedenú v tejto Zmluve
  alebo na akúkoľvek inú adresu, ktorú môže ktorákoľvek Zmluvná strana
  písomne určiť, (vi) prostredníctvom notifikačných služieb
  poskytovaných v rámci Platformy pre Zmluvy o poskytovaní energetických
  služieb -- \href{http://www.sharex.lv}{http://sunshineplatform.eu/},
  (vii) pokiaľ ide o Oznámenia s malým počtom informačných SMS správ s
  potvrdením o prijatí, ich zaslaním na mobilné číslo Zmluvnej strany
  uvedené v tejto Zmluve alebo na akékoľvek iné čísla, ktoré môže
  ktorákoľvek Zmluvná strana písomne určiť.
\item
  Prijatie Oznámení: všetky oznámenia sa považujú za prijaté po (i) ich
  prevzatí Zmluvnou stranou, ktorej bolo oznámenie zaslané, alebo (ii) v
  siedmy deň po odoslaní, podľa toho, čo nastane skôr.
\end{enumerate}

\end{multicols}
